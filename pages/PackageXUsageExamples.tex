% !TeX program = pdflatex
% !TeX root = PackageXUsageExamples.tex

\documentclass[../FeynHelpersManual.tex]{subfiles}
\begin{document}
\hypertarget{package-x usage examples}{
\section{Package-X usage examples}\label{package-x usage examples}\index{Package-X usage examples}}

The syntax of the FeynHelpers interface to Package-X is very simple.
There is essentially only one command \texttt{PaXEvalute} that does all
the job of evaluating Passarino-Veltman functions analytically.

Let us start with something simple, e.g.~the \(A_0\) function in the
standard normalization of Denner et al.~(used in LoopTools and many
other packages)

\begin{Shaded}
\begin{Highlighting}[]
\NormalTok{res}\ExtensionTok{=}\NormalTok{PaXEvaluateUVIRSplit}\OperatorTok{[}\NormalTok{A0}\OperatorTok{[}\FunctionTok{m}\SpecialCharTok{\^{}}\DecValTok{2}\OperatorTok{]]}
\end{Highlighting}
\end{Shaded}

A nice feature of Package-X is the ability to explicitly distinguish
between UV and IR poles, which can be important for renormalization or
matching calculations. To this aim we have 3 functions at our disposal
that essentially act in the same way as \texttt{PaXEvaluate} but label
the poles accordingly. These are \texttt{PaXEvaluateUVIRSplit},
\texttt{PaXEvaluateUV} and \texttt{PaXEvaluateIR}. For example,

\begin{Shaded}
\begin{Highlighting}[]
\NormalTok{PaXEvaluateUVIRSplit}\OperatorTok{[}\NormalTok{A0}\OperatorTok{[}\FunctionTok{m}\SpecialCharTok{\^{}}\DecValTok{2}\OperatorTok{]]}
\end{Highlighting}
\end{Shaded}

We can also abbreviate the singularity structure with
\texttt{SMP[\allowbreak{}"Delta_UV"]} using

\begin{Shaded}
\begin{Highlighting}[]
\NormalTok{res2}\ExtensionTok{=}\NormalTok{FCHideEpsilon}\OperatorTok{[}\NormalTok{res}\OperatorTok{]}
\end{Highlighting}
\end{Shaded}

or make it explicit again

\begin{Shaded}
\begin{Highlighting}[]
\NormalTok{FCShowEpsilon}\OperatorTok{[}\NormalTok{res2}\OperatorTok{]}
\end{Highlighting}
\end{Shaded}

To evaluate \texttt{FAD}-type integrals, we need to specify the loop
momentum explicitly. Furthermore, when we enter loop integrals as
\texttt{FAD*SPD}, we usually also imply that they have the standard
normalization with \(1/(2\pi)^D\). We do not have to write the prefactor
out explicitly, but we must of course include it when we are evaluating
our master integrals symbolically or numerically.

When employing \texttt{PaXEvaluate} this can be taken into account via
the option \texttt{PaXImplicitPrefactor}

\begin{Shaded}
\begin{Highlighting}[]
\NormalTok{PaXEvaluate}\OperatorTok{[}\NormalTok{FAD}\OperatorTok{[\{}\FunctionTok{q}\OperatorTok{,} \FunctionTok{m}\OperatorTok{\}],} \FunctionTok{q}\OperatorTok{,}\NormalTok{ PaXImplicitPrefactor }\OtherTok{{-}\textgreater{}} \DecValTok{1}\SpecialCharTok{/}\NormalTok{(}\DecValTok{2} \FunctionTok{Pi}\NormalTok{)}\SpecialCharTok{\^{}}\NormalTok{(}\DecValTok{4} \SpecialCharTok{{-}} \DecValTok{2}\NormalTok{ Epsilon)}\OperatorTok{]}
\end{Highlighting}
\end{Shaded}

When it comes to the evaluation of functions with complicated
kinematics, Package-X is sometimes a bit stubborn

\begin{Shaded}
\begin{Highlighting}[]
\NormalTok{PaXEvaluate}\OperatorTok{[}\NormalTok{PaVe}\OperatorTok{[}\DecValTok{0}\OperatorTok{,} \DecValTok{0}\OperatorTok{,} \DecValTok{1}\OperatorTok{,} \OperatorTok{\{}\NormalTok{SP}\OperatorTok{[}\FunctionTok{p}\OperatorTok{,} \FunctionTok{p}\OperatorTok{],}\NormalTok{ SP}\OperatorTok{[}\FunctionTok{p}\OperatorTok{,} \FunctionTok{p}\OperatorTok{],} \FunctionTok{m}\SpecialCharTok{\^{}}\DecValTok{2}\OperatorTok{\},} \OperatorTok{\{}\FunctionTok{m}\SpecialCharTok{\^{}}\DecValTok{2}\OperatorTok{,} \FunctionTok{m}\SpecialCharTok{\^{}}\DecValTok{2}\OperatorTok{,} \FunctionTok{m}\SpecialCharTok{\^{}}\DecValTok{2}\OperatorTok{\}]]}
\end{Highlighting}
\end{Shaded}

It knows that there is no simple way to express the full result for the
\(C_0\) function, so it prefers to keep in the implicit form. Using the
option \texttt{PaXC0Expand} we can obtain the (admittedly quite large
and complicated) result nonetheless

\begin{Shaded}
\begin{Highlighting}[]
\NormalTok{PaXEvaluate}\OperatorTok{[}\NormalTok{PaVe}\OperatorTok{[}\DecValTok{0}\OperatorTok{,} \DecValTok{0}\OperatorTok{,} \DecValTok{1}\OperatorTok{,} \OperatorTok{\{}\NormalTok{SP}\OperatorTok{[}\FunctionTok{p}\OperatorTok{,} \FunctionTok{p}\OperatorTok{],}\NormalTok{ SP}\OperatorTok{[}\FunctionTok{p}\OperatorTok{,} \FunctionTok{p}\OperatorTok{],} \FunctionTok{m}\SpecialCharTok{\^{}}\DecValTok{2}\OperatorTok{\},} \OperatorTok{\{}\FunctionTok{m}\SpecialCharTok{\^{}}\DecValTok{2}\OperatorTok{,} \FunctionTok{m}\SpecialCharTok{\^{}}\DecValTok{2}\OperatorTok{,} \FunctionTok{m}\SpecialCharTok{\^{}}\DecValTok{2}\OperatorTok{\}],}\NormalTok{ PaXC0Expand }\OtherTok{{-}\textgreater{}} \ConstantTok{True}\OperatorTok{]}
\end{Highlighting}
\end{Shaded}

Notice that if we are interested only in the UV-divergent piece, we can
get it rather easily without going through the full result.

\begin{Shaded}
\begin{Highlighting}[]
\NormalTok{PaXEvaluateUV}\OperatorTok{[}\NormalTok{PaVe}\OperatorTok{[}\DecValTok{0}\OperatorTok{,} \DecValTok{0}\OperatorTok{,} \DecValTok{1}\OperatorTok{,} \OperatorTok{\{}\NormalTok{SP}\OperatorTok{[}\FunctionTok{p}\OperatorTok{,} \FunctionTok{p}\OperatorTok{],}\NormalTok{ SP}\OperatorTok{[}\FunctionTok{p}\OperatorTok{,} \FunctionTok{p}\OperatorTok{],} \FunctionTok{m}\SpecialCharTok{\^{}}\DecValTok{2}\OperatorTok{\},} \OperatorTok{\{}\FunctionTok{m}\SpecialCharTok{\^{}}\DecValTok{2}\OperatorTok{,} \FunctionTok{m}\SpecialCharTok{\^{}}\DecValTok{2}\OperatorTok{,} \FunctionTok{m}\SpecialCharTok{\^{}}\DecValTok{2}\OperatorTok{\}],}\NormalTok{ PaXC0Expand }\OtherTok{{-}\textgreater{}} \ConstantTok{True}\OperatorTok{]}
\end{Highlighting}
\end{Shaded}

Notice that the UV-poles can be revealed using the built-in function
\texttt{PaVeUVPart}

\begin{Shaded}
\begin{Highlighting}[]
\NormalTok{PaVeUVPart}\OperatorTok{[}\NormalTok{PaVe}\OperatorTok{[}\DecValTok{0}\OperatorTok{,} \DecValTok{0}\OperatorTok{,} \DecValTok{1}\OperatorTok{,} \OperatorTok{\{}\NormalTok{SP}\OperatorTok{[}\FunctionTok{p}\OperatorTok{,} \FunctionTok{p}\OperatorTok{],}\NormalTok{ SP}\OperatorTok{[}\FunctionTok{p}\OperatorTok{,} \FunctionTok{p}\OperatorTok{],} \FunctionTok{m}\SpecialCharTok{\^{}}\DecValTok{2}\OperatorTok{\},} \OperatorTok{\{}\FunctionTok{m}\SpecialCharTok{\^{}}\DecValTok{2}\OperatorTok{,} \FunctionTok{m}\SpecialCharTok{\^{}}\DecValTok{2}\OperatorTok{,} \FunctionTok{m}\SpecialCharTok{\^{}}\DecValTok{2}\OperatorTok{\}]]}
\end{Highlighting}
\end{Shaded}

Furthermore, Package-X can expand coefficient functions in their
arguments, which is very useful for many applications. In
\texttt{PaXEvaluate} the corresponding option is called
\texttt{PaXSeries}

\begin{Shaded}
\begin{Highlighting}[]
\NormalTok{PaXEvaluate}\OperatorTok{[}\NormalTok{B0}\OperatorTok{[}\NormalTok{SPD}\OperatorTok{[}\NormalTok{p1}\OperatorTok{,}\NormalTok{ p1}\OperatorTok{],}\NormalTok{ m1}\SpecialCharTok{\^{}}\DecValTok{2}\OperatorTok{,}\NormalTok{ m2}\SpecialCharTok{\^{}}\DecValTok{2}\OperatorTok{],}\NormalTok{ PaXSeries }\OtherTok{{-}\textgreater{}} \OperatorTok{\{\{}\NormalTok{m1}\OperatorTok{,} \DecValTok{0}\OperatorTok{,} \DecValTok{2}\OperatorTok{\}\}]}
\end{Highlighting}
\end{Shaded}

By default Package-X tries to do expansion around the final results,
which is usually not what we want. With the option \texttt{PaXAnalytic}
the expansion will be done on the level of Feynman parameters, as one
would do it by hand

\begin{Shaded}
\begin{Highlighting}[]
\NormalTok{PaXEvaluate}\OperatorTok{[}\NormalTok{B0}\OperatorTok{[}\NormalTok{SPD}\OperatorTok{[}\NormalTok{p1}\OperatorTok{,}\NormalTok{ p1}\OperatorTok{],}\NormalTok{ m1}\SpecialCharTok{\^{}}\DecValTok{2}\OperatorTok{,}\NormalTok{ m2}\SpecialCharTok{\^{}}\DecValTok{2}\OperatorTok{],}\NormalTok{ PaXSeries }\OtherTok{{-}\textgreater{}} \OperatorTok{\{\{}\NormalTok{m1}\OperatorTok{,} \DecValTok{0}\OperatorTok{,} \DecValTok{2}\OperatorTok{\}\},}\NormalTok{ PaXAnalytic }\OtherTok{{-}\textgreater{}} \ConstantTok{True}\OperatorTok{]}
\end{Highlighting}
\end{Shaded}

We can also do a double expansion in \texttt{m1} and \texttt{m2}

\begin{Shaded}
\begin{Highlighting}[]
\NormalTok{PaXEvaluate}\OperatorTok{[}\NormalTok{B0}\OperatorTok{[}\NormalTok{SPD}\OperatorTok{[}\NormalTok{p1}\OperatorTok{,}\NormalTok{ p1}\OperatorTok{],}\NormalTok{ m1}\SpecialCharTok{\^{}}\DecValTok{2}\OperatorTok{,}\NormalTok{ m2}\SpecialCharTok{\^{}}\DecValTok{2}\OperatorTok{],}\NormalTok{ PaXSeries }\OtherTok{{-}\textgreater{}} \OperatorTok{\{\{}\NormalTok{m1}\OperatorTok{,} \DecValTok{0}\OperatorTok{,} \DecValTok{2}\OperatorTok{\},} \OperatorTok{\{}\NormalTok{m2}\OperatorTok{,} \DecValTok{0}\OperatorTok{,} \DecValTok{2}\OperatorTok{\}\},}\NormalTok{ PaXAnalytic }\OtherTok{{-}\textgreater{}} \ConstantTok{True}\OperatorTok{]}
\end{Highlighting}
\end{Shaded}

and reveal the origin of the poles

\begin{Shaded}
\begin{Highlighting}[]
\NormalTok{PaXEvaluateUVIRSplit}\OperatorTok{[}\NormalTok{B0}\OperatorTok{[}\NormalTok{SPD}\OperatorTok{[}\NormalTok{p1}\OperatorTok{,}\NormalTok{ p1}\OperatorTok{],}\NormalTok{ m1}\SpecialCharTok{\^{}}\DecValTok{2}\OperatorTok{,}\NormalTok{ m2}\SpecialCharTok{\^{}}\DecValTok{2}\OperatorTok{],}\NormalTok{ PaXSeries }\OtherTok{{-}\textgreater{}} \OperatorTok{\{\{}\NormalTok{m1}\OperatorTok{,} \DecValTok{0}\OperatorTok{,} \DecValTok{2}\OperatorTok{\}\},}\NormalTok{ PaXAnalytic }\OtherTok{{-}\textgreater{}} \ConstantTok{True}\OperatorTok{]}
\end{Highlighting}
\end{Shaded}

\begin{Shaded}
\begin{Highlighting}[]
\NormalTok{PaXEvaluateUVIRSplit}\OperatorTok{[}\NormalTok{B0}\OperatorTok{[}\NormalTok{SPD}\OperatorTok{[}\NormalTok{p1}\OperatorTok{,}\NormalTok{ p1}\OperatorTok{],}\NormalTok{ m1}\SpecialCharTok{\^{}}\DecValTok{2}\OperatorTok{,}\NormalTok{ m2}\SpecialCharTok{\^{}}\DecValTok{2}\OperatorTok{],}\NormalTok{ PaXSeries }\OtherTok{{-}\textgreater{}} \OperatorTok{\{\{}\NormalTok{m1}\OperatorTok{,} \DecValTok{0}\OperatorTok{,} \DecValTok{2}\OperatorTok{\},} \OperatorTok{\{}\NormalTok{m2}\OperatorTok{,} \DecValTok{0}\OperatorTok{,} \DecValTok{2}\OperatorTok{\}\},}\NormalTok{ PaXAnalytic }\OtherTok{{-}\textgreater{}} \ConstantTok{True}\OperatorTok{]}
\end{Highlighting}
\end{Shaded}

\end{document}
