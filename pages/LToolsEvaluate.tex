% !TeX program = pdflatex
% !TeX root = LToolsEvaluate.tex

\documentclass[../FeynHelpersManual.tex]{subfiles}
\begin{document}
\hypertarget{ltoolsevaluate}{
\section{LToolsEvaluate}\label{ltoolsevaluate}\index{LToolsEvaluate}}

\texttt{LToolsEvaluate[\allowbreak{}expr,\ \allowbreak{}q]} evaluates
Passarino-Veltman functions in \texttt{expr} numerically using LoopTools
by T. Hahn.

In contrast to the default behavior of LoopTools, the function returns
not just the finite part but also the singular pieces proportional to
\(1\varepsilon\) and \(1\varepsilon^2\). This behavior is controlled by
the option \texttt{LToolsFullResult}.

Notice that the normalization of Passarino-Veltman functions differs
between FeynCalc and LoopTools, cf.~Section 1.2 of the LoopTools manual.
In FeynCalc the overall prefactor is just \(1/(i \pi^2)\), while
LoopTools employs \(1/(i \pi^{D/2} r_{\Gamma})\) with
\(D = 4 -2 \varepsilon\) and
\(r_{\Gamma} = \Gamma^2 (1 - \varepsilon) \Gamma (1 + \varepsilon) / \Gamma(1-2 \varepsilon)\).

When the option \texttt{LToolsFullResult} is set to \texttt{True},
\texttt{LToolsEvaluate} will automatically account for this difference
by multiplying the LoopTools output with
\(1/\pi^{\varepsilon} r_\Gamma\).

However, for \texttt{LToolsFullResult -> False} no such conversion will
occur. This is because the proper conversion between different
\(\varepsilon\)-dependent normalizations requires the knowledge of the
poles: when terms proportional to \(\varepsilon\) multiply the poles,
they generate finite contributions. In this sense it is not recommended
to use \texttt{LToolsEvaluate} with \texttt{LToolsFullResult} set to
\texttt{False}, unless you precisely understand what you are doing.

\subsection{See also}

\hyperlink{toc}{Overview},
\hyperlink{ltoolsexpandinepsilon}{LToolsExpandInEpsilon},
\hyperlink{ltoolsfullresult}{LToolsFullResult},
\hyperlink{ltoolsimplicitprefactor}{LToolsImplicitPrefactor},
\hyperlink{ltoolssetmudim}{LToolsSetMudim},
\hyperlink{ltoolssetlambda}{LToolsSetLambda}.

\subsection{Examples}

Before using \texttt{LToolsEvaluate} we need to evaluate
\texttt{LToolsLoadLibray[\allowbreak{}]} to establish a connection with
the Mathlink executable. The value of the option \texttt{LToolsPath}
contains the full path to this file. You might need to adjust it
accordingly if LoopTools is installed in a different directory.

\begin{Shaded}
\begin{Highlighting}[]
\FunctionTok{OptionValue}\OperatorTok{[}\NormalTok{LToolsLoadLibrary}\OperatorTok{,}\NormalTok{ LToolsPath}\OperatorTok{]}
\end{Highlighting}
\end{Shaded}

\begin{dmath*}\breakingcomma
\text{/home/vs/.Mathematica/Applications/FeynCalc/AddOns/FeynHelpers/ExternalTools/LoopTools/LoopTools}
\end{dmath*}

Notice that \texttt{LToolsEvaluate} can also load the library by itself
on the first run, in case you forget to do so.

\begin{Shaded}
\begin{Highlighting}[]
\NormalTok{LToolsLoadLibrary}\OperatorTok{[]}
\end{Highlighting}
\end{Shaded}

\begin{dmath*}\breakingcomma
\text{LoopTools library loaded.}
\end{dmath*}

\begin{verbatim}
(* ====================================================
   FF 2.0, a package to evaluate one-loop integrals
 written by G. J. van Oldenborgh, NIKHEF-H, Amsterdam
 ====================================================
 for the algorithms used see preprint NIKHEF-H 89/17,
 'New Algorithms for One-loop Integrals', by G.J. van
 Oldenborgh and J.A.M. Vermaseren, published in 
 Zeitschrift fuer Physik C46(1990)425.
 ====================================================*)
\end{verbatim}

The value of the scale \(\mu\) can be set via the option
\texttt{LToolsSetMudim}. Evaluating the \texttt{PaVe}-function
\(A_0(m^2)\) at \(m^2 = 1/10\) with \(\mu^2=20\) yields

\begin{Shaded}
\begin{Highlighting}[]
\NormalTok{LToolsEvaluate}\OperatorTok{[}\NormalTok{A0}\OperatorTok{[}\FunctionTok{m}\SpecialCharTok{\^{}}\DecValTok{2}\OperatorTok{],}\NormalTok{ LToolsSetMudim }\OtherTok{{-}\textgreater{}} \DecValTok{20}\OperatorTok{,}\NormalTok{ InitialSubstitutions }\OtherTok{{-}\textgreater{}} \OperatorTok{\{}\FunctionTok{m}\SpecialCharTok{\^{}}\DecValTok{2} \OtherTok{{-}\textgreater{}} \DecValTok{1}\SpecialCharTok{/}\DecValTok{10}\OperatorTok{\}]}
\end{Highlighting}
\end{Shaded}

\begin{dmath*}\breakingcomma
0.457637\, +\frac{0.1}{\varepsilon }
\end{dmath*}

Cross-checking this result with Package-X yields the same value

\begin{Shaded}
\begin{Highlighting}[]
\NormalTok{(PaXEvaluate}\OperatorTok{[}\NormalTok{A0}\OperatorTok{[}\DecValTok{1}\SpecialCharTok{/}\DecValTok{10}\OperatorTok{]]} \OtherTok{/.}\NormalTok{ ScaleMu}\SpecialCharTok{\^{}}\DecValTok{2} \OtherTok{{-}\textgreater{}} \DecValTok{20}\NormalTok{) }\SpecialCharTok{//} \FunctionTok{N}
\end{Highlighting}
\end{Shaded}

\begin{dmath*}\breakingcomma
0.457637\, +\frac{0.1}{\varepsilon }
\end{dmath*}

More complicated input is also possible

\begin{Shaded}
\begin{Highlighting}[]
\FunctionTok{exp} \ExtensionTok{=} \FunctionTok{a}\NormalTok{ FAD}\OperatorTok{[\{}\FunctionTok{q}\OperatorTok{,} \FunctionTok{m}\OperatorTok{\},} \FunctionTok{q} \SpecialCharTok{{-}} \FunctionTok{p}\OperatorTok{]} \SpecialCharTok{+} \FunctionTok{b}\NormalTok{ FAD}\OperatorTok{[\{}\FunctionTok{q}\OperatorTok{,} \FunctionTok{M}\OperatorTok{,} \DecValTok{2}\OperatorTok{\}]}
\end{Highlighting}
\end{Shaded}

\begin{dmath*}\breakingcomma
\frac{a}{\left(q^2-m^2\right).(q-p)^2}+\frac{b}{\left(q^2-M^2\right)^2}
\end{dmath*}

\begin{Shaded}
\begin{Highlighting}[]
\NormalTok{res }\ExtensionTok{=}\NormalTok{ LToolsEvaluate}\OperatorTok{[}\FunctionTok{exp}\OperatorTok{,} \FunctionTok{q}\OperatorTok{,}\NormalTok{ InitialSubstitutions }\OtherTok{{-}\textgreater{}} \OperatorTok{\{}\FunctionTok{m} \OtherTok{{-}\textgreater{}} \FloatTok{0.12352}\OperatorTok{,} \FunctionTok{M} \OtherTok{{-}\textgreater{}} \FloatTok{5.14321}\OperatorTok{,}\NormalTok{ SPD}\OperatorTok{[}\FunctionTok{p}\OperatorTok{]} \OtherTok{{-}\textgreater{}} \FloatTok{0.8813}\OperatorTok{\},}\NormalTok{ LToolsImplicitPrefactor }\OtherTok{{-}\textgreater{}} \DecValTok{1}\SpecialCharTok{/}\NormalTok{(}\DecValTok{2} \FunctionTok{Pi}\NormalTok{)}\SpecialCharTok{\^{}}\NormalTok{(}\DecValTok{4} \SpecialCharTok{{-}} \DecValTok{2}\NormalTok{ Epsilon)}\OperatorTok{,}\NormalTok{ LToolsSetMudim }\OtherTok{{-}\textgreater{}} \DecValTok{23}\SpecialCharTok{\^{}}\DecValTok{2}\OperatorTok{]}
\end{Highlighting}
\end{Shaded}

\begin{dmath*}\breakingcomma
\frac{(0.\, +0.00633257 i) ((1.\, +0. i) a+(1.\, +0. i) b)}{\varepsilon }+(0.0661028\, +0.01955 i) ((0.\, +1. i) a+(0.128951\, +0.436013 i) b)
\end{dmath*}

Compare to Package-X

\begin{Shaded}
\begin{Highlighting}[]
\NormalTok{chk }\ExtensionTok{=}\NormalTok{ (PaXEvaluate}\OperatorTok{[}\FunctionTok{exp}\OperatorTok{,} \FunctionTok{q}\OperatorTok{,}\NormalTok{ PaXImplicitPrefactor }\OtherTok{{-}\textgreater{}} \DecValTok{1}\SpecialCharTok{/}\NormalTok{(}\DecValTok{2} \FunctionTok{Pi}\NormalTok{)}\SpecialCharTok{\^{}}\NormalTok{(}\DecValTok{4} \SpecialCharTok{{-}} \DecValTok{2}\NormalTok{ Epsilon)}\OperatorTok{]} \OtherTok{/.} \OperatorTok{\{}\NormalTok{ScaleMu}\SpecialCharTok{\^{}}\DecValTok{2} \OtherTok{{-}\textgreater{}}\DecValTok{23}\SpecialCharTok{\^{}}\DecValTok{2}\OperatorTok{,} \FunctionTok{m} \OtherTok{{-}\textgreater{}} \FloatTok{0.12352}\OperatorTok{,} \FunctionTok{M} \OtherTok{{-}\textgreater{}} \FloatTok{5.14321}\OperatorTok{,}\NormalTok{ FCI@SPD}\OperatorTok{[}\FunctionTok{p}\OperatorTok{]} \OtherTok{{-}\textgreater{}} \FloatTok{0.8813}\OperatorTok{\}}\NormalTok{) }\SpecialCharTok{//} \FunctionTok{N}
\end{Highlighting}
\end{Shaded}

\begin{dmath*}\breakingcomma
\frac{(0.\, +0.00633257 i) (a+b)}{\varepsilon }-(0.\, +0.00633257 i) (-14.4075 a-4.94944 b)+(-0.01955-0.0251337 i) a
\end{dmath*}

\begin{Shaded}
\begin{Highlighting}[]
\NormalTok{FCCompareNumbers}\OperatorTok{[}\NormalTok{res}\OperatorTok{,}\NormalTok{ chk}\OperatorTok{]}
\end{Highlighting}
\end{Shaded}

\begin{dmath*}\breakingcomma
\text{FCCompareNumbers: Minimal number of significant digits to agree in: }6
\end{dmath*}

\begin{dmath*}\breakingcomma
\text{FCCompareNumbers: Chop is set to }\text{1.$\grave{ }$*${}^{\wedge}$-10}
\end{dmath*}

\begin{dmath*}\breakingcomma
\text{FCCompareNumbers: No number is set to 0. by Chop at this stage. }
\end{dmath*}

\begin{dmath*}\breakingcomma
0
\end{dmath*}
\end{document}
