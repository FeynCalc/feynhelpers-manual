% !TeX program = pdflatex
% !TeX root = LToolsExpandInEpsilon.tex

\documentclass[../FeynHelpersManual.tex]{subfiles}
\begin{document}
\hypertarget{ltoolsexpandinepsilon}{
\section{LToolsExpandInEpsilon}\label{ltoolsexpandinepsilon}\index{LToolsExpandInEpsilon}}

\texttt{LToolsExpandInEpsilon} is an option for \texttt{LToolsEvaluate}.
When set to \texttt{True} (default), the result returned by LoopTools
and multiplied with proper conversion factors will be expanded around
\(\varepsilon = 0\) to \(\mathcal{O}(\varepsilon^0)\).

The \(\varepsilon\)-dependent conversion factors arise from the
differences in the normalization between Passarino-Veltman functions in
FeynCalc and LoopTools. In addition to that, the prefactor specified via
\texttt{LToolsImplicitPrefactor} may also depend on \(\varepsilon\).

Setting this option to \texttt{False} will leave the prefactors
unexpanded, which might sometimes be useful when examining the obtained
results.

\subsection{See also}

\hyperlink{toc}{Overview}, \hyperlink{ltoolsevaluate}{LToolsEvaluate},
\hyperlink{ltoolsimplicitprefactor}{LToolsImplicitPrefactor}.

\subsection{Examples}

\begin{Shaded}
\begin{Highlighting}[]
\NormalTok{LToolsLoadLibrary}\OperatorTok{[]}
\end{Highlighting}
\end{Shaded}

\begin{dmath*}\breakingcomma
\text{LoopTools library loaded.}
\end{dmath*}

\begin{Shaded}
\begin{Highlighting}[]
\CommentTok{(* ====================================================}
\CommentTok{   FF 2.0, a package to evaluate one{-}loop integrals}
\CommentTok{ written by G. J. van Oldenborgh, NIKHEF{-}H, Amsterdam}
\CommentTok{ ====================================================}
\CommentTok{ for the algorithms used see preprint NIKHEF{-}H 89/17,}
\CommentTok{ \textquotesingle{}New Algorithms for One{-}loop Integrals\textquotesingle{}, by G.J. van}
\CommentTok{ Oldenborgh and J.A.M. Vermaseren, published in }
\CommentTok{ Zeitschrift fuer Physik C46(1990)425.}
\CommentTok{ ====================================================*)}
\end{Highlighting}
\end{Shaded}

The default behavior of \texttt{LToolsEvaluate} is to do the
\(\varepsilon\)-expansion automatically

\begin{Shaded}
\begin{Highlighting}[]
\NormalTok{LToolsEvaluate}\OperatorTok{[}\NormalTok{FAD}\OperatorTok{[}\FunctionTok{q}\OperatorTok{,} \FunctionTok{q} \SpecialCharTok{{-}} \FunctionTok{p}\OperatorTok{],} \FunctionTok{q}\OperatorTok{,}\NormalTok{ InitialSubstitutions }\OtherTok{{-}\textgreater{}} \OperatorTok{\{}\NormalTok{SPD}\OperatorTok{[}\FunctionTok{p}\OperatorTok{]} \OtherTok{{-}\textgreater{}} \DecValTok{1}\OperatorTok{\}]}
\end{Highlighting}
\end{Shaded}

\begin{dmath*}\breakingcomma
\frac{0.\, +9.8696 i}{\varepsilon }-(31.0063\, -2.74429 i)
\end{dmath*}

This can be disabled by setting \texttt{LToolsExpandInEpsilon} to
\texttt{False}

\begin{Shaded}
\begin{Highlighting}[]
\NormalTok{LToolsEvaluate}\OperatorTok{[}\NormalTok{FAD}\OperatorTok{[}\FunctionTok{q}\OperatorTok{,} \FunctionTok{q} \SpecialCharTok{{-}} \FunctionTok{p}\OperatorTok{],} \FunctionTok{q}\OperatorTok{,}\NormalTok{ InitialSubstitutions }\OtherTok{{-}\textgreater{}} \OperatorTok{\{}\NormalTok{SPD}\OperatorTok{[}\FunctionTok{p}\OperatorTok{]} \OtherTok{{-}\textgreater{}} \DecValTok{1}\OperatorTok{\},}\NormalTok{ LToolsExpandInEpsilon }\OtherTok{{-}\textgreater{}} \ConstantTok{False}\OperatorTok{]}
\end{Highlighting}
\end{Shaded}

\begin{dmath*}\breakingcomma
\frac{(0.\, +1. i) \pi ^{2-\varepsilon } \Gamma (1-\varepsilon )^2 \Gamma (\varepsilon +1)}{\varepsilon  \Gamma (1-2 \varepsilon )}-\frac{(3.14159\, -2. i) \pi ^{2-\varepsilon } \Gamma (1-\varepsilon )^2 \Gamma (\varepsilon +1)}{\Gamma (1-2 \varepsilon )}
\end{dmath*}
\end{document}
