% !TeX program = pdflatex
% !TeX root = PaXContinuedDiLog.tex

\documentclass[../FeynHelpersManual.tex]{subfiles}
\begin{document}
\hypertarget{paxcontinueddilog}{
\section{PaXContinuedDiLog}\label{paxcontinueddilog}\index{PaXContinuedDiLog}}

\texttt{PaXContinuedDiLog} corresponds to \texttt{ContinuedDiLog} in
Package-X.

\subsection{See also}

\hyperlink{toc}{Overview}.

\subsection{Examples}

\begin{Shaded}
\begin{Highlighting}[]
\CommentTok{(*Just to load Package{-}X*)}
\NormalTok{  PaXEvaluate}\OperatorTok{[}\NormalTok{A0}\OperatorTok{[}\DecValTok{1}\OperatorTok{]]}\NormalTok{;}
\end{Highlighting}
\end{Shaded}

\texttt{PaXContinuedDiLog} uses \texttt{X\textasciigrave ContinuedDiLog}
for numerical evaluations

\begin{Shaded}
\begin{Highlighting}[]
\NormalTok{PaXContinuedDiLog}\OperatorTok{[\{}\FloatTok{3.2}\OperatorTok{,} \FloatTok{1.0}\OperatorTok{\},} \OperatorTok{\{}\FloatTok{1.1}\OperatorTok{,} \FloatTok{1.0}\OperatorTok{\}]}
\end{Highlighting}
\end{Shaded}

\begin{dmath*}\breakingcomma
-1.7089
\end{dmath*}

\begin{Shaded}
\begin{Highlighting}[]
\FunctionTok{X}\NormalTok{\textasciigrave{}ContinuedDiLog}\OperatorTok{[\{}\FloatTok{3.2}\OperatorTok{,} \FloatTok{1.0}\OperatorTok{\},} \OperatorTok{\{}\FloatTok{1.1}\OperatorTok{,} \FloatTok{1.0}\OperatorTok{\}]}
\end{Highlighting}
\end{Shaded}

\begin{dmath*}\breakingcomma
-1.7089
\end{dmath*}

The same goes for derivatives and series expansions

\begin{Shaded}
\begin{Highlighting}[]
\FunctionTok{D}\OperatorTok{[}\NormalTok{PaXContinuedDiLog}\OperatorTok{[\{}\FunctionTok{x}\OperatorTok{,}\NormalTok{ xInf}\OperatorTok{\},} \OperatorTok{\{}\FunctionTok{y}\OperatorTok{,}\NormalTok{ yInf}\OperatorTok{\}],} \FunctionTok{x}\OperatorTok{]}
\end{Highlighting}
\end{Shaded}

\begin{dmath*}\breakingcomma
-\frac{y (-\log (x+i \;\text{xInf}\epsilon )-\log (y+i \;\text{yInf}\epsilon ))}{1-x y}
\end{dmath*}

\begin{Shaded}
\begin{Highlighting}[]
\FunctionTok{Series}\OperatorTok{[}\NormalTok{PaXContinuedDiLog}\OperatorTok{[\{}\FunctionTok{x}\OperatorTok{,}\NormalTok{ xInf}\OperatorTok{\},} \OperatorTok{\{}\FunctionTok{y}\OperatorTok{,}\NormalTok{ yInf}\OperatorTok{\}],} \OperatorTok{\{}\FunctionTok{x}\OperatorTok{,} \DecValTok{1}\OperatorTok{,} \DecValTok{2}\OperatorTok{\}]}
\end{Highlighting}
\end{Shaded}

\begin{dmath*}\breakingcomma
\mathcal{L}_2(1+i \;\text{xInf}\epsilon ,y+i \;\text{yInf}\epsilon )-\frac{(x-1) y \log (y+i \;\text{yInf}\epsilon )}{y-1}+\frac{(x-1)^2 \left(y^2 \log (y+i \;\text{yInf}\epsilon )-y^2+y\right)}{2 (y-1)^2}+O\left((x-1)^3\right)
\end{dmath*}
\end{document}
