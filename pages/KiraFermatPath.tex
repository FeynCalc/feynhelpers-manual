% !TeX program = pdflatex
% !TeX root = KiraFermatPath.tex

\documentclass[../FeynHelpersManual.tex]{subfiles}
\begin{document}
\begin{Shaded}
\begin{Highlighting}[]
 
\end{Highlighting}
\end{Shaded}

\hypertarget{kirafermatpath}{
\section{KiraFermatPath}\label{kirafermatpath}\index{KiraFermatPath}}

\texttt{KiraFermatPath} is an option for \texttt{KiraRunReduction} and
other Kira-related functions.

It specifies the full path to the Fermat binary used by Kira to run the
reduction. This is done via the option \texttt{KiraFermatPath}. The
default value is \texttt{Automatic} meaning that suitable binaries are
expected to be located in
\texttt{FileNameJoin[\allowbreak{}\{\allowbreak{}\$FeynHelpersDirectory,\ \allowbreak{}"ExternalTools",\ \allowbreak{}"Fermat"\}]}

\subsection{See also}

\hyperlink{toc}{Overview},
\hyperlink{kiramassdimensions}{KiraMassDimensions},
\hyperlink{kirajobfilename}{KiraJobFileName},
\hyperlink{kiraintegrals}{KiraIntegrals}.

\subsection{Examples}
\end{document}
