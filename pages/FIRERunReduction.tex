% !TeX program = pdflatex
% !TeX root = FIRERunReduction.tex

\documentclass[../FeynHelpersManual.tex]{subfiles}
\begin{document}
\hypertarget{firerunreduction}{
\section{FIRERunReduction}\label{firerunreduction}\index{FIRERunReduction}}

\texttt{FIRERunReduction[\allowbreak{}path]} runs C++ FIRE on the FIRE
.config file specified by path. To that aim the FIRE binary is started
in the background via \texttt{RunProcess}. The function returns
\texttt{True} if the evaluation succeeds and \texttt{False} otherwise.

If \texttt{path} represents a full path to a file, then this file is
used as the .config file. If it is just a path to a directory, then
\texttt{path/topoName/topoName.config} is assumed to be the full path.

The default path to the FIRE binary is
\texttt{FileNameJoin[\allowbreak{}\{\allowbreak{}\$UserBaseDirectory,\ \allowbreak{}"Applications",\ \allowbreak{}"FIRE6",\ \allowbreak{}"bin",\ \allowbreak{}"FIRE6"\}]}.
It can be modified via the option \texttt{FIREBinaryPath}.

\subsection{See also}

\hyperlink{toc}{Overview},
\hyperlink{firecreateconfigfile}{FIRECreateConfigFile},
\hyperlink{firecreatestartfile}{FIRECreateStartFile}.

\subsection{Examples}

\begin{Shaded}
\begin{Highlighting}[]
\NormalTok{FIRERunReduction}\OperatorTok{[}\FunctionTok{FileNameJoin}\OperatorTok{[\{}\NormalTok{$FeynHelpersDirectory}\OperatorTok{,} \StringTok{"Documentation"}\OperatorTok{,} \StringTok{"Examples"}\OperatorTok{,} \StringTok{"asyR2prop2Ltopo13311X01201N1"}\OperatorTok{\}],}\NormalTok{ FCVerbose }\OtherTok{{-}\textgreater{}} \DecValTok{3}\OperatorTok{]}
\end{Highlighting}
\end{Shaded}

\begin{dmath*}\breakingcomma
\text{FIRERunReduction: Full path to the FIRE binary: }\;\text{/home/vs/.Mathematica/Applications/FIRE6/bin/FIRE6}
\end{dmath*}

\begin{dmath*}\breakingcomma
\text{FIRERunReduction: Working directory: }\;\text{/home/vs/.Mathematica/Applications/FeynCalc/AddOns/FeynHelpers/Documentation/Examples/asyR2prop2Ltopo13311X01201N1/}
\end{dmath*}

\begin{dmath*}\breakingcomma
\text{FIRERunReduction: Config file: }\;\text{asyR2prop2Ltopo13311X01201N1}
\end{dmath*}

\begin{dmath*}\breakingcomma
\text{True}
\end{dmath*}
\end{document}
