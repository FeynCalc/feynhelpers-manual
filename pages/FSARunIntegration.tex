% !TeX program = pdflatex
% !TeX root = FSARunIntegration.tex

\documentclass[../FeynHelpersManual.tex]{subfiles}
\begin{document}
\hypertarget{fsarunintegration}{
\section{FSARunIntegration}\label{fsarunintegration}\index{FSARunIntegration}}

\texttt{FSARunIntegration[\allowbreak{}path]} evaluates a FIESTA script
\texttt{FiestaScript.m} in \texttt{path}. To that aim a Mathematica
kernel is started in the background via \texttt{RunProcess}. The
function returns \texttt{True} if the evaluation succeeds and
\texttt{False} otherwise.

Alternatively, one can use
\texttt{FSARunIntegration[\allowbreak{}path,\ \allowbreak{}topo]} where
\texttt{topo} is an \texttt{FCTopology} symbol and the full path is
implied to be \texttt{path/topoName/FiestaScript.m}.

If you need to process a list of topologies, following syntaxes are
possible
\texttt{FiestaScript.m[\allowbreak{}\{\allowbreak{}path1,\ \allowbreak{}path2,\ \allowbreak{}...\}]},
\texttt{FiestaScript.m[\allowbreak{}path,\ \allowbreak{}\{\allowbreak{}topo1,\ \allowbreak{}topo2,\ \allowbreak{}...\}]}

The path to the Mathematica Kernel can be specified via
\texttt{FSAMathematicaKernelPath}. The default value is
\texttt{Automatic}.

\subsection{See also}

\hyperlink{toc}{Overview}, \hyperlink{fsashowoutput}{FSAShowOutput},
\hyperlink{fsamathematicakernelpath}{FSAMathematicaKernelPath}.

\subsection{Examples}

\begin{Shaded}
\begin{Highlighting}[]
\NormalTok{topo1 }\ExtensionTok{=}\NormalTok{ FCTopology}\OperatorTok{[}\NormalTok{prop1L}\OperatorTok{,} \OperatorTok{\{}\SpecialCharTok{{-}}\NormalTok{SFAD}\OperatorTok{[\{\{}\FunctionTok{I}\NormalTok{ p1}\OperatorTok{,} \DecValTok{0}\OperatorTok{\},} \OperatorTok{\{}\SpecialCharTok{{-}}\NormalTok{m1}\SpecialCharTok{\^{}}\DecValTok{2}\OperatorTok{,} \SpecialCharTok{{-}}\DecValTok{1}\OperatorTok{\},} \DecValTok{1}\OperatorTok{\}],} \SpecialCharTok{{-}}\NormalTok{SFAD}\OperatorTok{[\{\{}\FunctionTok{I}\NormalTok{ (p1 }\SpecialCharTok{+} \FunctionTok{q}\NormalTok{)}\OperatorTok{,} \DecValTok{0}\OperatorTok{\},} \OperatorTok{\{}\SpecialCharTok{{-}}\NormalTok{m2}\SpecialCharTok{\^{}}\DecValTok{2}\OperatorTok{,} \SpecialCharTok{{-}}\DecValTok{1}\OperatorTok{\},} \DecValTok{1}\OperatorTok{\}]\},} \OperatorTok{\{}\NormalTok{p1}\OperatorTok{\},} \OperatorTok{\{}\FunctionTok{q}\OperatorTok{\},} \OperatorTok{\{\},} \OperatorTok{\{\}]}
\NormalTok{int1 }\ExtensionTok{=}\NormalTok{ GLI}\OperatorTok{[}\NormalTok{prop1L}\OperatorTok{,} \OperatorTok{\{}\DecValTok{1}\OperatorTok{,} \DecValTok{1}\OperatorTok{\}]}
\end{Highlighting}
\end{Shaded}

\begin{dmath*}\breakingcomma
\text{FCTopology}\left(\text{prop1L},\left\{-\frac{1}{(-\text{p1}^2+\text{m1}^2-i \eta )},-\frac{1}{(-(\text{p1}+q)^2+\text{m2}^2-i \eta )}\right\},\{\text{p1}\},\{q\},\{\},\{\}\right)
\end{dmath*}

\begin{dmath*}\breakingcomma
G^{\text{prop1L}}(1,1)
\end{dmath*}

\begin{Shaded}
\begin{Highlighting}[]
\FunctionTok{fileNames} \ExtensionTok{=}\NormalTok{ FSAPrepareSDEvaluate}\OperatorTok{[}\NormalTok{int1}\OperatorTok{,}\NormalTok{ topo1}\OperatorTok{,} \FunctionTok{FileNameJoin}\OperatorTok{[\{}\NormalTok{$FeynCalcDirectory}\OperatorTok{,} \StringTok{"Database"}\OperatorTok{\}],} 
\NormalTok{    FinalSubstitutions }\OtherTok{{-}\textgreater{}} \OperatorTok{\{}\NormalTok{SPD}\OperatorTok{[}\FunctionTok{q}\OperatorTok{]} \OtherTok{{-}\textgreater{}}\NormalTok{ qq}\OperatorTok{,}\NormalTok{ qq }\OtherTok{{-}\textgreater{}} \FloatTok{20.} \OperatorTok{,}\NormalTok{ m1 }\OtherTok{{-}\textgreater{}} \FloatTok{2.} \OperatorTok{,}\NormalTok{ m2 }\OtherTok{{-}\textgreater{}} \FloatTok{2.}\OperatorTok{\}]}\NormalTok{;}
\end{Highlighting}
\end{Shaded}

\begin{Shaded}
\begin{Highlighting}[]
\NormalTok{FSARunIntegration}\OperatorTok{[}\FunctionTok{fileNames}\OperatorTok{[[}\DecValTok{1}\OperatorTok{]]]}
\end{Highlighting}
\end{Shaded}

\begin{dmath*}\breakingcomma
\text{FSARunIntegration}\left(G^{\text{prop1L}}(1,1)\right)
\end{dmath*}
\end{document}
