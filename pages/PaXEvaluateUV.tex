% !TeX program = pdflatex
% !TeX root = PaXEvaluateUV.tex

\documentclass[../FeynHelpersManual.tex]{subfiles}
\begin{document}
\hypertarget{paxevaluateuv}{
\section{PaXEvaluateUV}\label{paxevaluateuv}\index{PaXEvaluateUV}}

\texttt{PaXEvaluateUV[\allowbreak{}expr,\ \allowbreak{}q]} is like
\texttt{PaXEvaluate} but with the difference that it returns only the
UV-divergent part of the result.

The evaluation is using H. Patel's Package-X.

\subsection{See also}

\hyperlink{toc}{Overview}, \hyperlink{paxevaluateir}{PaXEvaluateIR},
\hyperlink{paxevaluate}{PaXEvaluate},
\hyperlink{paxevaluateuvirsplit}{PaXEvaluateUVIRSplit}.

\subsection{Examples}

\begin{Shaded}
\begin{Highlighting}[]
\NormalTok{int }\ExtensionTok{=} \SpecialCharTok{{-}}\NormalTok{FAD}\OperatorTok{[\{}\FunctionTok{k}\OperatorTok{,} \FunctionTok{m}\OperatorTok{\}]} \SpecialCharTok{+} \DecValTok{2}\SpecialCharTok{*}\NormalTok{FAD}\OperatorTok{[}\FunctionTok{k}\OperatorTok{,} \OperatorTok{\{}\FunctionTok{k} \SpecialCharTok{{-}} \FunctionTok{p}\OperatorTok{,} \FunctionTok{m}\OperatorTok{\}]}\SpecialCharTok{*}\NormalTok{(}\FunctionTok{m}\SpecialCharTok{\^{}}\DecValTok{2} \SpecialCharTok{+}\NormalTok{ SPD}\OperatorTok{[}\FunctionTok{p}\OperatorTok{,} \FunctionTok{p}\OperatorTok{]}\NormalTok{)}
\end{Highlighting}
\end{Shaded}

\begin{dmath*}\breakingcomma
\frac{2 \left(m^2+p^2\right)}{k^2.\left((k-p)^2-m^2\right)}-\frac{1}{k^2-m^2}
\end{dmath*}

\begin{Shaded}
\begin{Highlighting}[]
\NormalTok{PaXEvaluateUV}\OperatorTok{[}\SpecialCharTok{\%}\OperatorTok{,} \FunctionTok{k}\OperatorTok{,}\NormalTok{ PaXImplicitPrefactor }\OtherTok{{-}\textgreater{}} \DecValTok{1}\SpecialCharTok{/}\NormalTok{(}\DecValTok{2} \FunctionTok{Pi}\NormalTok{)}\SpecialCharTok{\^{}}\FunctionTok{D}\OperatorTok{,}\NormalTok{ FCE }\OtherTok{{-}\textgreater{}} \ConstantTok{True}\OperatorTok{]}
\end{Highlighting}
\end{Shaded}

\begin{dmath*}\breakingcomma
\frac{i m^2}{16 \pi ^2 \varepsilon _{\text{UV}}}+\frac{i p^2}{8 \pi ^2 \varepsilon _{\text{UV}}}
\end{dmath*}

Notice that with \texttt{PaVeUVPart} one can get the same result

\begin{Shaded}
\begin{Highlighting}[]
\NormalTok{res }\ExtensionTok{=}\NormalTok{ PaVeUVPart}\OperatorTok{[}\NormalTok{ToPaVe}\OperatorTok{[}\NormalTok{int}\OperatorTok{,} \FunctionTok{k}\OperatorTok{],}\NormalTok{ Prefactor }\OtherTok{{-}\textgreater{}} \DecValTok{1}\SpecialCharTok{/}\NormalTok{(}\DecValTok{2} \FunctionTok{Pi}\NormalTok{)}\SpecialCharTok{\^{}}\FunctionTok{D}\OperatorTok{]}
\end{Highlighting}
\end{Shaded}

\begin{dmath*}\breakingcomma
-\frac{i 2^{1-2 D} \pi ^{2-2 D} \left(2^{D+1} \pi ^D m^2-(2 \pi )^D m^2+2^{D+1} \pi ^D p^2\right)}{D-4}
\end{dmath*}

\begin{Shaded}
\begin{Highlighting}[]
\FunctionTok{Series}\OperatorTok{[}\NormalTok{FCReplaceD}\OperatorTok{[}\NormalTok{res}\OperatorTok{,} \FunctionTok{D} \OtherTok{{-}\textgreater{}} \DecValTok{4} \SpecialCharTok{{-}} \DecValTok{2}\NormalTok{ EpsilonUV}\OperatorTok{],} \OperatorTok{\{}\NormalTok{EpsilonUV}\OperatorTok{,} \DecValTok{0}\OperatorTok{,} \DecValTok{0}\OperatorTok{\}]} \SpecialCharTok{//} \FunctionTok{Normal} \SpecialCharTok{//}\NormalTok{ SelectNotFree2}\OperatorTok{[}\NormalTok{\#}\OperatorTok{,}\NormalTok{ EpsilonUV}\OperatorTok{]}\NormalTok{ \& }\SpecialCharTok{//} \FunctionTok{ExpandAll}
\end{Highlighting}
\end{Shaded}

\begin{dmath*}\breakingcomma
\frac{i m^2}{16 \pi ^2 \varepsilon _{\text{UV}}}+\frac{i p^2}{8 \pi ^2 \varepsilon _{\text{UV}}}
\end{dmath*}

\begin{Shaded}
\begin{Highlighting}[]
\NormalTok{int2 }\ExtensionTok{=}\NormalTok{ TID}\OperatorTok{[}\NormalTok{FVD}\OperatorTok{[}\DecValTok{2} \FunctionTok{k} \SpecialCharTok{{-}} \FunctionTok{p}\OperatorTok{,}\NormalTok{ mu}\OperatorTok{]}\NormalTok{ FVD}\OperatorTok{[}\DecValTok{2} \FunctionTok{k} \SpecialCharTok{{-}} \FunctionTok{p}\OperatorTok{,}\NormalTok{ nu}\OperatorTok{]}\NormalTok{ FAD}\OperatorTok{[\{}\FunctionTok{k}\OperatorTok{,} \FunctionTok{m}\OperatorTok{\},} \OperatorTok{\{}\FunctionTok{k} \SpecialCharTok{{-}} \FunctionTok{p}\OperatorTok{,} \FunctionTok{m}\OperatorTok{\}]} \SpecialCharTok{{-}} \DecValTok{2}\NormalTok{ MTD}\OperatorTok{[}\NormalTok{mu}\OperatorTok{,}\NormalTok{ nu}\OperatorTok{]}\NormalTok{ FAD}\OperatorTok{[\{}\FunctionTok{k}\OperatorTok{,} \FunctionTok{m}\OperatorTok{\}],} \FunctionTok{k}\OperatorTok{]}
\end{Highlighting}
\end{Shaded}

\begin{dmath*}\breakingcomma
\frac{-(1-D) p^2 p^{\text{mu}} p^{\text{nu}}-D p^2 p^{\text{mu}} p^{\text{nu}}-4 m^2 p^2 g^{\text{mu}\;\text{nu}}+p^4 g^{\text{mu}\;\text{nu}}+4 m^2 p^{\text{mu}} p^{\text{nu}}}{(1-D) p^2 \left(k^2-m^2\right).\left((k-p)^2-m^2\right)}+\frac{2 \left(-(1-D) p^2 g^{\text{mu}\;\text{nu}}-D p^{\text{mu}} p^{\text{nu}}-p^2 g^{\text{mu}\;\text{nu}}+2 p^{\text{mu}} p^{\text{nu}}\right)}{(1-D) p^2 \left(k^2-m^2\right)}
\end{dmath*}

\begin{Shaded}
\begin{Highlighting}[]
\NormalTok{PaXEvaluateUV}\OperatorTok{[}\NormalTok{int2}\OperatorTok{,} \FunctionTok{k}\OperatorTok{,}\NormalTok{ PaXImplicitPrefactor }\OtherTok{{-}\textgreater{}} \DecValTok{1}\SpecialCharTok{/}\NormalTok{(}\DecValTok{2} \FunctionTok{Pi}\NormalTok{)}\SpecialCharTok{\^{}}\FunctionTok{D}\OperatorTok{,}\NormalTok{ FCE }\OtherTok{{-}\textgreater{}} \ConstantTok{True}\OperatorTok{]}
\end{Highlighting}
\end{Shaded}

\begin{dmath*}\breakingcomma
\frac{i p^{\text{mu}} p^{\text{nu}}}{48 \pi ^2 \varepsilon _{\text{UV}}}-\frac{i p^2 g^{\text{mu}\;\text{nu}}}{48 \pi ^2 \varepsilon _{\text{UV}}}
\end{dmath*}
\end{document}
