% !TeX program = pdflatex
% !TeX root = QGCreateAmp.tex

\documentclass[../FeynHelpersManual.tex]{subfiles}
\begin{document}
\hypertarget{qgcreateamp}{
\section{QGCreateAmp}\label{qgcreateamp}\index{QGCreateAmp}}

\texttt{QGCreateAmp[\allowbreak{}nLoops,\ \allowbreak{}\{\allowbreak{}"InParticle1[\allowbreak{}p1]",\ \allowbreak{}"InParticle1[\allowbreak{}p2]",\ \allowbreak{}...\} -> \{\allowbreak{}"OutParticle1[\allowbreak{}k1]",\ \allowbreak{}"OutParticle1[\allowbreak{}k2]",\ \allowbreak{}...\}]}
calls \texttt{QGRAF} to generate Feynman amplitudes and (optionally) the
corresponding diagrams, using the specified model and style files.

The function returns a list with the paths to two files, where the first
file contains the amplitudes and the second file the diagrams (graphical
representations of the amplitudes).

\subsection{See also}

\hyperlink{toc}{Overview}, \hyperlink{qgconverttofc}{QGConvertToFC},
\hyperlink{qgloadinsertions}{QGLoadInsertions}.

\subsection{Examples}
\end{document}
