% !TeX program = pdflatex
% !TeX root = FSALoadNumericalResults.tex

\documentclass[../FeynHelpersManual.tex]{subfiles}
\begin{document}
\begin{Shaded}
\begin{Highlighting}[]
 
\end{Highlighting}
\end{Shaded}

\hypertarget{fsaloadnumericalresults}{
\section{FSALoadNumericalResults}\label{fsaloadnumericalresults}\index{FSALoadNumericalResults}}

\texttt{FSALoadNumericalResults[\allowbreak{}files]} is a simple
function that loads numerical results generated by the FIESTA script
created via \texttt{FSAPrepareSDEvaluate} into Mathematica. The argument
\texttt{files} is the output of \texttt{FSAPrepareSDEvaluate} that
contains the full paths to \texttt{FIESTAScript.m} and the output file.

The option\texttt{Chop} (set to \texttt{10^(-10)} by default) tells the
function to remove numerical artefacts.

\subsection{See also}

\hyperlink{toc}{Overview},
\hyperlink{psdcreatepythonscripts}{PSDCreatePythonScripts}.

\subsection{Examples}

\begin{Shaded}
\begin{Highlighting}[]
\NormalTok{files }\ExtensionTok{=} \OperatorTok{\{}
    \FunctionTok{FileNameJoin}\OperatorTok{[\{}\NormalTok{$FeynHelpersDirectory}\OperatorTok{,} \StringTok{"Documentation"}\OperatorTok{,} \StringTok{"Examples"}\OperatorTok{,} \StringTok{"FIESTA"}\OperatorTok{,} \StringTok{"prop1LX11"}\OperatorTok{,} \StringTok{"FIESTAScript.m"}\OperatorTok{\}],} 
    \FunctionTok{FileNameJoin}\OperatorTok{[\{}\StringTok{"numres\_1.\_1.\_1.\_fiesta.m"}\OperatorTok{\}]\}}\NormalTok{;}
\end{Highlighting}
\end{Shaded}

\begin{Shaded}
\begin{Highlighting}[]
\NormalTok{FSALoadNumericalResults}\OperatorTok{[}\NormalTok{files}\OperatorTok{]}
\end{Highlighting}
\end{Shaded}

\begin{dmath*}\breakingcomma
\frac{0.000021 \;\text{pm4}+0.999987}{\text{ep}}+0.000038 \;\text{pm5}+0.186207
\end{dmath*}
\end{document}
