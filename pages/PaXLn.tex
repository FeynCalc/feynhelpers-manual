% !TeX program = pdflatex
% !TeX root = PaXLn.tex

\documentclass[../FeynHelpersManual.tex]{subfiles}
\begin{document}
\hypertarget{paxln}{
\section{PaXLn}\label{paxln}\index{PaXLn}}

\texttt{PaXLn} corresponds to \texttt{Ln} in Package-X.

\subsection{See also}

\hyperlink{toc}{Overview}.

\subsection{Examples}

\begin{Shaded}
\begin{Highlighting}[]
\CommentTok{(*Just to load Package{-}X*)}
\NormalTok{  PaXEvaluate}\OperatorTok{[}\NormalTok{A0}\OperatorTok{[}\DecValTok{1}\OperatorTok{]]}\NormalTok{;}
\end{Highlighting}
\end{Shaded}

\texttt{PaXLn} uses \texttt{X\textasciigrave Ln} for numerical
evaluations

\begin{Shaded}
\begin{Highlighting}[]
\NormalTok{PaXLn}\OperatorTok{[}\SpecialCharTok{{-}}\FloatTok{4.5}\OperatorTok{,} \DecValTok{1}\OperatorTok{]}
\end{Highlighting}
\end{Shaded}

\begin{dmath*}\breakingcomma
1.50408\, +3.14159 i
\end{dmath*}

\begin{Shaded}
\begin{Highlighting}[]
\FunctionTok{X}\NormalTok{\textasciigrave{}Ln}\OperatorTok{[}\SpecialCharTok{{-}}\FloatTok{4.5}\OperatorTok{,} \DecValTok{1}\OperatorTok{]}
\end{Highlighting}
\end{Shaded}

\begin{dmath*}\breakingcomma
1.50408\, +3.14159 i
\end{dmath*}

The same goes for derivatives and series expansions

\begin{Shaded}
\begin{Highlighting}[]
\FunctionTok{D}\OperatorTok{[}\NormalTok{PaXLn}\OperatorTok{[}\FunctionTok{x}\OperatorTok{,} \SpecialCharTok{\textbackslash{}}\OperatorTok{[}\NormalTok{Alpha}\OperatorTok{]],} \FunctionTok{x}\OperatorTok{]}
\end{Highlighting}
\end{Shaded}

\begin{dmath*}\breakingcomma
\frac{1}{x}
\end{dmath*}

\begin{Shaded}
\begin{Highlighting}[]
\FunctionTok{Series}\OperatorTok{[}\NormalTok{PaXLn}\OperatorTok{[}\FunctionTok{x}\OperatorTok{,} \SpecialCharTok{\textbackslash{}}\OperatorTok{[}\NormalTok{Alpha}\OperatorTok{]],} \OperatorTok{\{}\FunctionTok{x}\OperatorTok{,} \DecValTok{1}\OperatorTok{,} \DecValTok{2}\OperatorTok{\}]}
\end{Highlighting}
\end{Shaded}

\begin{dmath*}\breakingcomma
\log (1+i \alpha \epsilon )+(x-1)-\frac{1}{2} (x-1)^2+O\left((x-1)^3\right)
\end{dmath*}
\end{document}
