% !TeX program = pdflatex
% !TeX root = PSDUsageExamples.tex

\documentclass[../FeynHelpersManual.tex]{subfiles}
\begin{document}
\hypertarget{pysecdec usage examples}{
\section{pySecDec usage examples}\label{pysecdec usage examples}\index{pySecDec usage examples}}

The main idea behind the FeynHelpers interface to pySecDec is to
facilitate the generation of pySecDec scripts for integrals written in
the FeynCalc notation (i.e.~as \texttt{GLI}s with the corresponding
lists of \texttt{FCTopology} symbols).

The main high-level function of this interface is called
\texttt{PSDCreatePythonScripts}. In the simplest case we need two
provide following arguments and options

\begin{itemize}
\tightlist
\item
  the 1st argument is some \texttt{GLI}
\item
  the 2nd argument is the \texttt{FCTopology} to which this \texttt{GLI}
  belongs
\item
  the 3rd argument is where to put the directory with pySecDec scripts.
  For quick tests one can simply use
  \texttt{NotebookDirectory[\allowbreak{}]}
\item
  the option \texttt{PSDRequestedOrder} specifies the order in
  \(\varepsilon\) to which the integral should be evaluated (default is
  \texttt{0})
\item
  the option \texttt{PSDRealParameterRules} is a list of rules for
  replacing kinematic invariants with numerical values which are real
  numbers. For complex numbers you need to use
  \texttt{PSDComplexParameterRules}
\item
  if the script directory already exists, the function will by default
  refuse to overwrite it. Setting the option \texttt{OverwriteTarget} to
  \texttt{True} you can tell the code that you do not care about that
\end{itemize}

Here is a simple 1-loop example that incorporates all of the above

\begin{verbatim}
int = GLI[prop1L, {1, 1}]
topo = FCTopology[prop1L, {FAD[{p1, m1}], FAD[{p1 + q, m2}]}, {p1}, {q}, {Hold[SPD][q] -> qq}, {}]
files = PSDCreatePythonScripts[int, topo, NotebookDirectory[], 
  PSDRealParameterRules -> {qq -> 1., m1 -> 2., m2 -> 3.}, OverwriteTarget -> True]
\end{verbatim}

The output is a list containing two elements which are full paths to the
two pySecDec script files \texttt{generate_int.py} and
\texttt{integrate_int.py}. You can now switch to the terminal, enter the
corresponding directory and perform the integral evaluation by first
running

\begin{verbatim}
python generate_int.py
\end{verbatim}

Here is a sample output of this script

\begin{verbatim}
running "sum_package" for loopint
running "make_package" for "loopint_integral"
computing Jacobian determinant for primary sector 0
total number sectors before symmetry finding: 2
total number sectors after symmetry finding (iterative): 2
total number sectors after symmetry finding (light Pak): 2
total number sectors after symmetry finding (full Pak): 2
writing FORM files for sector 1
writing FORM files for sector 2
expanding the prefactor exp(EulerGamma*eps)*gamma(eps) (regulators: [eps] , orders: [0] )
 + (1)*eps**-1 + (0)
"loopint_integral" done
\end{verbatim}

Now you need to compile the generated library files. This can be done
via

\begin{verbatim}
make -j8 -C loopint
\end{verbatim}

where 8 stands for the number threads to be run simultaneously. It
depends on how powerful the CPU in your machine is.

Finally, entering

\begin{verbatim}
python integrate_int.py
\end{verbatim}

will perform the actual numerical evaluation and save the obtained
results to \texttt{numres_*_psd.txt}, \texttt{numres_*_mma.m} and
\texttt{numres_*_maple.mpl}. Here \texttt{*} stands for the numerical
values of kinematic invariants present in the integral. You can modify
those values without the need to recompile the libraries by simply
editing the arrays \texttt{num_params_real} and
\texttt{num_params_complex} in \texttt{integrate_int.py}.

For Mathematica users the file \texttt{numres_*_mma.m} is probably the
most useful one. You can load the content of this file into your
Mathematica session using the function \texttt{PSDLoadNumericalResults}.
To that aim you just need to give it the output of
\texttt{PSDCreatePythonScripts} and set the options
\texttt{PSDRealParameterRules} and \texttt{PSDComplexParameterRules} to
the same values that were used when invoking
\texttt{PSDCreatePythonScripts}

\begin{verbatim}
PSDLoadNumericalResults[files, PSDRealParameterRules -> {qq -> 1., m1 -> 2., m2 -> 3.}]
\end{verbatim}
\end{document}
