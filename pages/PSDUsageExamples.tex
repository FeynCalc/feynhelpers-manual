% !TeX program = pdflatex
% !TeX root = PSDUsageExamples.tex

\documentclass[../FeynHelpersManual.tex]{subfiles}
\begin{document}
\hypertarget{pysecdec usage examples}{
\section{pySecDec usage examples}\label{pysecdec usage examples}\index{pySecDec usage examples}}

The main idea behind the FeynHelpers interface to pySecDec is to
facilitate the generation of pySecDec scripts for integrals written in
the FeynCalc notation (i.e.~as \texttt{GLI}s with the corresponding
lists of \texttt{FCTopology} symbols).

The main high-level function of this interface is called
\texttt{PSDCreatePythonScripts}. In the simplest case we need two
provide following arguments and options

\begin{itemize}
\tightlist
\item
  the 1st argument is some \texttt{GLI}
\item
  the 2nd argument is the \texttt{FCTopology} to which this \texttt{GLI}
  belongs
\item
  the 3rd argument is where to put the directory with pySecDec scripts.
  For quick tests one can simply use
  \texttt{NotebookDirectory[\allowbreak{}]}
\item
  the option \texttt{PSDRequestedOrder} specifies the order in
  \(\varepsilon\) to which the integral should be evaluated (default is
  \texttt{0})
\item
  the option \texttt{PSDRealParameterRules} is a list of rules for
  replacing kinematic invariants with numerical values which are real
  numbers. For complex numbers you need to use
  \texttt{PSDComplexParameterRules}
\item
  if the script directory already exists, the function will by default
  refuse to overwrite it. Setting the option \texttt{OverwriteTarget} to
  \texttt{True} you can tell the code that you do not care about that
\end{itemize}

Here is a simple 1-loop example that incorporates all of the above

\begin{Shaded}
\begin{Highlighting}[]
\NormalTok{int }\ExtensionTok{=}\NormalTok{ GLI}\OperatorTok{[}\NormalTok{prop1L}\OperatorTok{,} \OperatorTok{\{}\DecValTok{1}\OperatorTok{,} \DecValTok{1}\OperatorTok{\}]}
\NormalTok{topo }\ExtensionTok{=}\NormalTok{ FCTopology}\OperatorTok{[}\NormalTok{prop1L}\OperatorTok{,} \OperatorTok{\{}\NormalTok{FAD}\OperatorTok{[\{}\NormalTok{p1}\OperatorTok{,}\NormalTok{ m1}\OperatorTok{\}],}\NormalTok{ FAD}\OperatorTok{[\{}\NormalTok{p1 }\SpecialCharTok{+} \FunctionTok{q}\OperatorTok{,}\NormalTok{ m2}\OperatorTok{\}]\},} \OperatorTok{\{}\NormalTok{p1}\OperatorTok{\},} \OperatorTok{\{}\FunctionTok{q}\OperatorTok{\},} \OperatorTok{\{}\FunctionTok{Hold}\OperatorTok{[}\NormalTok{SPD}\OperatorTok{][}\FunctionTok{q}\OperatorTok{]} \OtherTok{{-}\textgreater{}}\NormalTok{ qq}\OperatorTok{\},} \OperatorTok{\{\}]}
\NormalTok{files }\ExtensionTok{=}\NormalTok{ PSDCreatePythonScripts}\OperatorTok{[}\NormalTok{int}\OperatorTok{,}\NormalTok{ topo}\OperatorTok{,} \FunctionTok{NotebookDirectory}\OperatorTok{[],} 
\NormalTok{  PSDRealParameterRules }\OtherTok{{-}\textgreater{}} \OperatorTok{\{}\NormalTok{qq }\OtherTok{{-}\textgreater{}} \FloatTok{1.}\OperatorTok{,}\NormalTok{ m1 }\OtherTok{{-}\textgreater{}} \FloatTok{2.}\OperatorTok{,}\NormalTok{ m2 }\OtherTok{{-}\textgreater{}} \FloatTok{3.}\OperatorTok{\},}\NormalTok{ OverwriteTarget }\OtherTok{{-}\textgreater{}} \ConstantTok{True}\OperatorTok{]}
\end{Highlighting}
\end{Shaded}

The output is a list containing two elements which are full paths to the
two pySecDec script files \texttt{generate_int.py} and
\texttt{integrate_int.py}. You can now switch to the terminal, enter the
corresponding directory and perform the integral evaluation by first
running

\begin{Shaded}
\begin{Highlighting}[]
\NormalTok{python }\AttributeTok{generate\_}\NormalTok{int.py}
\end{Highlighting}
\end{Shaded}

Here is a sample output of this script

\begin{Shaded}
\begin{Highlighting}[]
\NormalTok{running }\StringTok{"sum\_package"} \FunctionTok{for}\NormalTok{ loopint}
\NormalTok{running }\StringTok{"make\_package"} \FunctionTok{for} \StringTok{"loopint\_integral"}
\NormalTok{computing Jacobian determinant }\FunctionTok{for}\NormalTok{ primary sector }\DecValTok{0}
\FunctionTok{total} \FunctionTok{number}\NormalTok{ sectors before symmetry finding: }\DecValTok{2}
\FunctionTok{total} \FunctionTok{number}\NormalTok{ sectors after symmetry finding (iterative): }\DecValTok{2}
\FunctionTok{total} \FunctionTok{number}\NormalTok{ sectors after symmetry finding (light Pak): }\DecValTok{2}
\FunctionTok{total} \FunctionTok{number}\NormalTok{ sectors after symmetry finding (}\ConstantTok{full}\NormalTok{ Pak): }\DecValTok{2}
\NormalTok{writing FORM files }\FunctionTok{for}\NormalTok{ sector }\DecValTok{1}
\NormalTok{writing FORM files }\FunctionTok{for}\NormalTok{ sector }\DecValTok{2}
\NormalTok{expanding the prefactor }\FunctionTok{exp}\NormalTok{(}\FunctionTok{EulerGamma}\SpecialCharTok{*}\NormalTok{eps)}\SpecialCharTok{*}\FunctionTok{gamma}\NormalTok{(eps) (regulators: }\OperatorTok{[}\NormalTok{eps}\OperatorTok{]} \OperatorTok{,}\NormalTok{ orders: }\OperatorTok{[}\DecValTok{0}\OperatorTok{]}\NormalTok{ )}
 \SpecialCharTok{+}\NormalTok{ (}\DecValTok{1}\NormalTok{)}\SpecialCharTok{*}\NormalTok{eps}\SpecialCharTok{**{-}}\DecValTok{1} \SpecialCharTok{+}\NormalTok{ (}\DecValTok{0}\NormalTok{)}
\StringTok{"loopint\_integral"}\NormalTok{ done}
\end{Highlighting}
\end{Shaded}

Now you need to compile the generated library files. This can be done
via

\begin{Shaded}
\begin{Highlighting}[]
\NormalTok{make }\SpecialCharTok{{-}}\NormalTok{j8 }\SpecialCharTok{{-}}\FunctionTok{C}\NormalTok{ loopint}
\end{Highlighting}
\end{Shaded}

where 8 stands for the number threads to be run simultaneously. It
depends on how powerful the CPU in your machine is.

Finally, entering

\begin{Shaded}
\begin{Highlighting}[]
\NormalTok{python }\AttributeTok{integrate\_}\NormalTok{int.py}
\end{Highlighting}
\end{Shaded}

will perform the actual numerical evaluation and save the obtained
results to \texttt{numres_*_psd.txt}, \texttt{numres_*_mma.m} and
\texttt{numres_*_maple.mpl}. Here \texttt{*} stands for the numerical
values of kinematic invariants present in the integral. You can modify
those values without the need to recompile the libraries by simply
editing the arrays \texttt{num_params_real} and
\texttt{num_params_complex} in \texttt{integrate_int.py}.

For Mathematica users the file \texttt{numres_*_mma.m} is probably the
most useful one. You can load the content of this file into your
Mathematica session using the function \texttt{PSDLoadNumericalResults}.
To that aim you just need to give it the output of
\texttt{PSDCreatePythonScripts} and set the options
\texttt{PSDRealParameterRules} and \texttt{PSDComplexParameterRules} to
the same values that were used when invoking
\texttt{PSDCreatePythonScripts}

\begin{Shaded}
\begin{Highlighting}[]
\NormalTok{PSDLoadNumericalResults}\OperatorTok{[}\NormalTok{files}\OperatorTok{,}\NormalTok{ PSDRealParameterRules }\OtherTok{{-}\textgreater{}} \OperatorTok{\{}\NormalTok{qq }\OtherTok{{-}\textgreater{}} \FloatTok{1.}\OperatorTok{,}\NormalTok{ m1 }\OtherTok{{-}\textgreater{}} \FloatTok{2.}\OperatorTok{,}\NormalTok{ m2 }\OtherTok{{-}\textgreater{}} \FloatTok{3.}\OperatorTok{\}]}
\end{Highlighting}
\end{Shaded}

\end{document}
