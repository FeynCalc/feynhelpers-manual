% !TeX program = pdflatex
% !TeX root = PSDLoadNumericalResults.tex

\documentclass[../FeynHelpersManual.tex]{subfiles}
\begin{document}
\FloatBarrier
\begin{figure}[!ht]
\centering
\includegraphics[width=0.6\linewidth]{img/1ftg3ucp91fwc.pdf}
\end{figure}
\FloatBarrier

\begin{Shaded}
\begin{Highlighting}[]
 
\end{Highlighting}
\end{Shaded}

\hypertarget{psdloadnumericalresults}{
\section{PSDLoadNumericalResults}\label{psdloadnumericalresults}\index{PSDLoadNumericalResults}}

\texttt{PSDLoadNumericalResults[\allowbreak{}files]} is a simple
function that loads numerical results generated by the pySecDec script
\texttt{integrate__int.py} into Mathematica. The argument \texttt{files}
is the output of \texttt{PSDCreatePythonScripts} that contains the full
paths to \texttt{generate_int.py} and \texttt{integrate_int.py}.

Furthermore, the function requires the options
\texttt{PSDComplexParameterRules} and \texttt{PSDRealParameterRules}
that must be assigned exactly the same values that were used when
evaluating PSDCreatePythonScripts. From this information the function
will recover the full path to the \texttt{numres_*_mma.m} file and load
it.

The options \texttt{Normal} (set to \texttt{True} by default) and
\texttt{Chop} (set to \texttt{10^(-10)} by default) tell the function to
convert the expression from \texttt{SeriesData} to a polynomial and to
remove numerical artefacts.

The output for each integral is a list containing two entries. The first
entry is the numerical result, while the second one provides numerical
errors.

\subsection{See also}

\hyperlink{toc}{Overview},
\hyperlink{psdcreatepythonscripts}{PSDCreatePythonScripts}.

\subsection{Examples}

\begin{Shaded}
\begin{Highlighting}[]
\NormalTok{files }\ExtensionTok{=} \OperatorTok{\{}
    \FunctionTok{FileNameJoin}\OperatorTok{[\{}\NormalTok{$FeynHelpersDirectory}\OperatorTok{,} \StringTok{"Documentation"}\OperatorTok{,} \StringTok{"Examples"}\OperatorTok{,} \StringTok{"prop1LX11"}\OperatorTok{,} \StringTok{"integrate\_int.py"}\OperatorTok{\}],} 
    \FunctionTok{FileNameJoin}\OperatorTok{[\{}\NormalTok{$FeynHelpersDirectory}\OperatorTok{,} \StringTok{"Documentation"}\OperatorTok{,} \StringTok{"Examples"}\OperatorTok{,} \StringTok{"prop1LX11"}\OperatorTok{,} \StringTok{"generate\_int.py"}\OperatorTok{\}]\}}\NormalTok{;}
\end{Highlighting}
\end{Shaded}

\begin{Shaded}
\begin{Highlighting}[]
\NormalTok{FCCompareNumbers}\OperatorTok{[}\NormalTok{PSDLoadNumericalResults}\OperatorTok{[}\NormalTok{files}\OperatorTok{,}\NormalTok{ PSDRealParameterRules }\OtherTok{{-}\textgreater{}} \OperatorTok{\{}\NormalTok{qq }\OtherTok{{-}\textgreater{}} \FloatTok{1.} \OperatorTok{,}\NormalTok{ m1 }\OtherTok{{-}\textgreater{}} \FloatTok{2.} \OperatorTok{,}\NormalTok{ m2 }\OtherTok{{-}\textgreater{}} \FloatTok{3.}\OperatorTok{\}],} 
  \OperatorTok{\{}\SpecialCharTok{{-}}\FloatTok{1.819085009768877} \SpecialCharTok{+}\NormalTok{ eps}\SpecialCharTok{\^{}}\NormalTok{(}\SpecialCharTok{{-}}\DecValTok{1}\NormalTok{)}\OperatorTok{,} \DecValTok{0}\OperatorTok{\},}\NormalTok{ FCVerbose }\OtherTok{{-}\textgreater{}} \SpecialCharTok{{-}}\DecValTok{1}\OperatorTok{]}
\end{Highlighting}
\end{Shaded}

\begin{dmath*}\breakingcomma
\{0,0\}
\end{dmath*}
\end{document}
