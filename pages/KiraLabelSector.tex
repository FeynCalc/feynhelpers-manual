% !TeX program = pdflatex
% !TeX root = KiraLabelSector.tex

\documentclass[../FeynHelpersManual.tex]{subfiles}
\begin{document}
\begin{Shaded}
\begin{Highlighting}[]
 
\end{Highlighting}
\end{Shaded}

\hypertarget{kiralabelsector}{
\section{KiraLabelSector}\label{kiralabelsector}\index{KiraLabelSector}}

\texttt{KiraLabelSector[\allowbreak{}sec]} returns the standard Kira
labelling \(S\) for the given sector \texttt{sec}
(e.g.~\texttt{\{\allowbreak{}1,\ \allowbreak{}1,\ \allowbreak{}0,\ \allowbreak{}1,\ \allowbreak{}1\}}).

\subsection{See also}

\hyperlink{toc}{Overview},
\hyperlink{kiracreatejobfile}{KiraCreateJobFile}.

\subsection{Examples}

\begin{Shaded}
\begin{Highlighting}[]
\NormalTok{KiraLabelSector}\OperatorTok{[\{}\DecValTok{1}\OperatorTok{,} \DecValTok{1}\OperatorTok{,} \DecValTok{0}\OperatorTok{,} \DecValTok{1}\OperatorTok{,} \DecValTok{1}\OperatorTok{\}]}
\end{Highlighting}
\end{Shaded}

\begin{dmath*}\breakingcomma
27
\end{dmath*}

\begin{Shaded}
\begin{Highlighting}[]
\NormalTok{KiraLabelSector}\OperatorTok{[\{}\DecValTok{1}\OperatorTok{,} \DecValTok{1}\OperatorTok{,} \DecValTok{0}\OperatorTok{,} \DecValTok{0}\OperatorTok{,} \DecValTok{0}\OperatorTok{\}]}
\end{Highlighting}
\end{Shaded}

\begin{dmath*}\breakingcomma
3
\end{dmath*}

\begin{Shaded}
\begin{Highlighting}[]
\NormalTok{KiraLabelSector}\OperatorTok{[\{\{}\DecValTok{1}\OperatorTok{,} \DecValTok{1}\OperatorTok{,} \DecValTok{0}\OperatorTok{,} \DecValTok{1}\OperatorTok{,} \DecValTok{1}\OperatorTok{\},} \OperatorTok{\{}\DecValTok{1}\OperatorTok{,} \DecValTok{1}\OperatorTok{,} \DecValTok{1}\OperatorTok{,} \DecValTok{0}\OperatorTok{,} \DecValTok{0}\OperatorTok{\}\}]}
\end{Highlighting}
\end{Shaded}

\begin{dmath*}\breakingcomma
\{27,7\}
\end{dmath*}

\begin{Shaded}
\begin{Highlighting}[]
\NormalTok{KiraLabelSector}\OperatorTok{[\{}\DecValTok{1}\OperatorTok{,} \DecValTok{1}\OperatorTok{,} \DecValTok{0}\OperatorTok{,} \DecValTok{0}\OperatorTok{,} \DecValTok{0}\OperatorTok{\}]}
\end{Highlighting}
\end{Shaded}

\begin{dmath*}\breakingcomma
3
\end{dmath*}
\end{document}
