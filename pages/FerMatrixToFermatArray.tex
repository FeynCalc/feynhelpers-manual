% !TeX program = pdflatex
% !TeX root = FerMatrixToFermatArray.tex

\documentclass[../FeynHelpersManual.tex]{subfiles}
\begin{document}
\hypertarget{fermatrixtofermatarray}{
\section{FerMatrixToFermatArray}\label{fermatrixtofermatarray}\index{FerMatrixToFermatArray}}

\texttt{FerMatrixToFermatArray[\allowbreak{}mat,\ \allowbreak{}varName]}
is an auxiliary function that converts the matrix \texttt{mat} to a
Fermat array named \texttt{varName}, where the latter must be a string.

The function returns a string that represents the matrix, a list of
auxiliary variables (introduced to be compatible with the restrictions
of Fermat) and a replacement rule for converting auxiliary variables
back into the original variables.

\subsection{See also}

\hyperlink{toc}{Overview}.

\subsection{Examples}

\begin{Shaded}
\begin{Highlighting}[]
\NormalTok{FerMatrixToFermatArray}\OperatorTok{[\{\{}\FunctionTok{a}\OperatorTok{,} \FunctionTok{b}\OperatorTok{\},} \OperatorTok{\{}\FunctionTok{c}\OperatorTok{,} \FunctionTok{d}\OperatorTok{\}\},} \StringTok{"mat"}\OperatorTok{]}
\end{Highlighting}
\end{Shaded}

\begin{dmath*}\breakingcomma
\{\text{Array mat[2,2];$\backslash $n[mat]:=[[fv1, fv3, fv2, fv4]];},\{\text{fv1},\text{fv2},\text{fv3},\text{fv4}\},\{\text{fv1}\to a,\text{fv2}\to b,\text{fv3}\to c,\text{fv4}\to d\}\}
\end{dmath*}
\end{document}
