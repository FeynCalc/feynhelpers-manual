% !TeX program = pdflatex
% !TeX root = PaXEvaluate.tex

\documentclass[../FeynHelpersManual.tex]{subfiles}
\begin{document}
\hypertarget{paxevaluate}{
\section{PaXEvaluate}\label{paxevaluate}\index{PaXEvaluate}}

\texttt{PaXEvaluate[\allowbreak{}expr,\ \allowbreak{}q]} evaluates
scalar 1-loop integrals (up to 4-point functions) in \texttt{expr} that
depend on the loop momentum \texttt{q} in \texttt{D} dimensions.

The evaluation is using H. Patel's Package-X.

\subsection{See also}

\hyperlink{toc}{Overview}, \hyperlink{paxevaluateir}{PaXEvaluateIR},
\hyperlink{paxevaluateuv}{PaXEvaluateUV},
\hyperlink{paxevaluateuvirsplit}{PaXEvaluateUVIRSplit}.

\subsection{Examples}

\begin{Shaded}
\begin{Highlighting}[]
\NormalTok{FAD}\OperatorTok{[\{}\FunctionTok{q}\OperatorTok{,} \FunctionTok{m}\OperatorTok{\}]}
\NormalTok{PaXEvaluate}\OperatorTok{[}\SpecialCharTok{\%}\OperatorTok{,} \FunctionTok{q}\OperatorTok{,}\NormalTok{ PaXImplicitPrefactor }\OtherTok{{-}\textgreater{}} \DecValTok{1}\SpecialCharTok{/}\NormalTok{(}\DecValTok{2} \FunctionTok{Pi}\NormalTok{)}\SpecialCharTok{\^{}}\NormalTok{(}\DecValTok{4} \SpecialCharTok{{-}} \DecValTok{2}\NormalTok{ Epsilon)}\OperatorTok{]}
\end{Highlighting}
\end{Shaded}

\begin{dmath*}\breakingcomma
\frac{1}{q^2-m^2}
\end{dmath*}

\begin{dmath*}\breakingcomma
\frac{i m^2}{16 \pi ^2 \varepsilon }-\frac{i m^2 \left(-\log \left(\frac{\mu ^2}{m^2}\right)+\gamma -1-\log (4 \pi )\right)}{16 \pi ^2}
\end{dmath*}

\begin{Shaded}
\begin{Highlighting}[]
\NormalTok{FAD}\OperatorTok{[\{}\FunctionTok{l}\OperatorTok{,} \DecValTok{0}\OperatorTok{\},} \OperatorTok{\{}\FunctionTok{q} \SpecialCharTok{+} \FunctionTok{l}\OperatorTok{,} \DecValTok{0}\OperatorTok{\}]}
\NormalTok{PaXEvaluate}\OperatorTok{[}\SpecialCharTok{\%}\OperatorTok{,} \FunctionTok{l}\OperatorTok{,}\NormalTok{ PaXImplicitPrefactor }\OtherTok{{-}\textgreater{}} \DecValTok{1}\SpecialCharTok{/}\NormalTok{(}\DecValTok{2} \FunctionTok{Pi}\NormalTok{)}\SpecialCharTok{\^{}}\NormalTok{(}\DecValTok{4} \SpecialCharTok{{-}} \DecValTok{2}\NormalTok{ Epsilon)}\OperatorTok{]}
\end{Highlighting}
\end{Shaded}

\begin{dmath*}\breakingcomma
\frac{1}{l^2.(l+q)^2}
\end{dmath*}

\begin{dmath*}\breakingcomma
\frac{i}{16 \pi ^2 \varepsilon }+\frac{i \log \left(-\frac{4 \pi  \mu ^2}{q^2}\right)}{16 \pi ^2}-\frac{i (\gamma -2)}{16 \pi ^2}
\end{dmath*}

\texttt{PaVe} functions do not require the second argument specifying
the loop momentum

\begin{Shaded}
\begin{Highlighting}[]
\NormalTok{PaVe}\OperatorTok{[}\DecValTok{0}\OperatorTok{,} \OperatorTok{\{}\DecValTok{0}\OperatorTok{,}\NormalTok{ Pair}\OperatorTok{[}\NormalTok{Momentum}\OperatorTok{[}\FunctionTok{p}\OperatorTok{,} \FunctionTok{D}\OperatorTok{],}\NormalTok{ Momentum}\OperatorTok{[}\FunctionTok{p}\OperatorTok{,} \FunctionTok{D}\OperatorTok{]],}\NormalTok{ Pair}\OperatorTok{[}\NormalTok{Momentum}\OperatorTok{[}\FunctionTok{p}\OperatorTok{,} \FunctionTok{D}\OperatorTok{],}\NormalTok{ Momentum}\OperatorTok{[}\FunctionTok{p}\OperatorTok{,} \FunctionTok{D}\OperatorTok{]]\},} \OperatorTok{\{}\DecValTok{0}\OperatorTok{,} \DecValTok{0}\OperatorTok{,} \FunctionTok{M}\OperatorTok{\}]}
\NormalTok{PaXEvaluate}\OperatorTok{[}\SpecialCharTok{\%}\OperatorTok{]} 
  
 
\end{Highlighting}
\end{Shaded}

\begin{dmath*}\breakingcomma
\text{C}_0\left(0,p^2,p^2,0,0,M\right)
\end{dmath*}

\begin{dmath*}\breakingcomma
\frac{1}{\varepsilon  M-\varepsilon  p^2}-\frac{\gamma -\log \left(\frac{\mu ^2}{\pi  M}\right)}{M-p^2}+\frac{\log \left(\frac{M}{M-p^2}\right)}{M-p^2}+\frac{M \log \left(\frac{M}{M-p^2}\right)}{p^2 \left(M-p^2\right)}
\end{dmath*}
\end{document}
