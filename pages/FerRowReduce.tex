% !TeX program = pdflatex
% !TeX root = FerRowReduce.tex

\documentclass[../FeynHelpersManual.tex]{subfiles}
\begin{document}
\hypertarget{ferrowreduce}{
\section{FerRowReduce}\label{ferrowreduce}\index{FerRowReduce}}

\texttt{FerRowReduce[\allowbreak{}mat]}uses Fermat to obtain the
row-reduced echelon form of matrix \texttt{mat}. An important difference
to Mathematica's \texttt{RowReduce} is that Fermat does not assume all
symbolic variables to be nonzero by default.

The location of script, input and output files is controlled by the
options \texttt{FerScriptFile}, \texttt{FerInputFile},
\texttt{FerOutputFile}. When set to \texttt{Automatic} (default), these
files will be automatically created via
\texttt{CreateTemporary[\allowbreak{}]}. If the option \texttt{Delete}
is set to \texttt{True} (default), the files will be deleted after a
successful Fermat run.

\subsection{See also}

\hyperlink{toc}{Overview}, \hyperlink{ferrowreduce}{FerRowReduce}.

\subsection{Examples}

The syntax of FerSolve is very similar to that of \texttt{Solve}

\begin{Shaded}
\begin{Highlighting}[]
\NormalTok{res1 }\ExtensionTok{=} \FunctionTok{RowReduce}\OperatorTok{[\{\{}\DecValTok{3}\OperatorTok{,} \DecValTok{1}\OperatorTok{,} \FunctionTok{a}\OperatorTok{\},} \OperatorTok{\{}\DecValTok{2}\OperatorTok{,} \DecValTok{1}\OperatorTok{,} \FunctionTok{b}\OperatorTok{\}\}]}
\end{Highlighting}
\end{Shaded}

\begin{dmath*}\breakingcomma
\left(
\begin{array}{ccc}
 1 & 0 & a-b \\
 0 & 1 & 3 b-2 a \\
\end{array}
\right)
\end{dmath*}

\begin{Shaded}
\begin{Highlighting}[]
\NormalTok{res2 }\ExtensionTok{=}\NormalTok{ FerRowReduce}\OperatorTok{[\{\{}\DecValTok{3}\OperatorTok{,} \DecValTok{1}\OperatorTok{,} \FunctionTok{a}\OperatorTok{\},} \OperatorTok{\{}\DecValTok{2}\OperatorTok{,} \DecValTok{1}\OperatorTok{,} \FunctionTok{b}\OperatorTok{\}\}]} \SpecialCharTok{//} \FunctionTok{Normal}

\CommentTok{(*FerRunScript: Running Fermat.}
\CommentTok{FerRunScript: Done running Fermat, timing: 0.5115*)}
\end{Highlighting}
\end{Shaded}

\begin{dmath*}\breakingcomma
\left(
\begin{array}{ccc}
 1 & 0 & a-b \\
 0 & 1 & 3 b-2 a \\
\end{array}
\right)
\end{dmath*}

\begin{Shaded}
\begin{Highlighting}[]
\NormalTok{res1 }\ExtensionTok{===}\NormalTok{ res2}
\end{Highlighting}
\end{Shaded}

\begin{dmath*}\breakingcomma
\text{True}
\end{dmath*}

This is an example for
\href{https://mathematica.stackexchange.com/questions/228098/how-to-write-a-more-concise-rowreduce-function-that-can-deal-with-symbolic-mat?noredirect=1\&lq=1}{Mathematica
SE}, where RowReduce assumes \(a \neq 0\)

\begin{Shaded}
\begin{Highlighting}[]
\FunctionTok{RowReduce}\OperatorTok{[\{\{}\DecValTok{1}\OperatorTok{,} \FunctionTok{a}\OperatorTok{,} \DecValTok{2}\OperatorTok{\},} \OperatorTok{\{}\DecValTok{0}\OperatorTok{,} \DecValTok{1}\OperatorTok{,} \DecValTok{1}\OperatorTok{\},} \OperatorTok{\{}\SpecialCharTok{{-}}\DecValTok{1}\OperatorTok{,} \DecValTok{1}\OperatorTok{,} \DecValTok{1}\OperatorTok{\}\}]}
\end{Highlighting}
\end{Shaded}

\begin{dmath*}\breakingcomma
\left(
\begin{array}{ccc}
 1 & 0 & 0 \\
 0 & 1 & 0 \\
 0 & 0 & 1 \\
\end{array}
\right)
\end{dmath*}

\begin{Shaded}
\begin{Highlighting}[]
\NormalTok{FerRowReduce}\OperatorTok{[\{\{}\DecValTok{1}\OperatorTok{,} \FunctionTok{a}\OperatorTok{,} \DecValTok{2}\OperatorTok{\},} \OperatorTok{\{}\DecValTok{0}\OperatorTok{,} \DecValTok{1}\OperatorTok{,} \DecValTok{1}\OperatorTok{\},} \OperatorTok{\{}\SpecialCharTok{{-}}\DecValTok{1}\OperatorTok{,} \DecValTok{1}\OperatorTok{,} \DecValTok{1}\OperatorTok{\}\}]} \SpecialCharTok{//} \FunctionTok{Normal} 
  
 


\CommentTok{(*FerRunScript: Running Fermat.}
\CommentTok{FerRunScript: Done running Fermat, timing: 0.1393*)}
\end{Highlighting}
\end{Shaded}

\begin{dmath*}\breakingcomma
\left(
\begin{array}{ccc}
 1 & 0 & 2-a \\
 0 & 1 & 1 \\
 0 & 0 & 2-a \\
\end{array}
\right)
\end{dmath*}
\end{document}
