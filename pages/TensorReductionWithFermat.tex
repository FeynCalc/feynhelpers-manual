% !TeX program = pdflatex
% !TeX root = TensorReductionWithFermat.tex

\documentclass[../FeynHelpersManual.tex]{subfiles}
\begin{document}
\hypertarget{tensor reduction with fermat}{
\section{Tensor reduction with Fermat}\label{tensor reduction with fermat}\index{Tensor reduction with Fermat}}

One of the most useful functions exposed by the Fermat interface is
\texttt{FerSolve} that is vastly superior to Mathematica's
\texttt{Solve} when dealing with very large symbolic systems of
equations.

A typical situation where one needs to solve such equations is the
derivation of tensor decomposition formulas. To this aim FeynCalc's
\texttt{Tdec} can directly use \texttt{FerSolve}, once FeynHelpers is
loaded. One just needs to set the option \texttt{Solve} to
\texttt{FerSolve}.

The following example calculates tensor reduction formula for a rank 6
integral with 2 loop momenta and two external momenta. The Fermat part
requires only 40 seconds on a modern laptop to solve the corresponding
\(52 \times 52\) symbolic system.

\begin{verbatim}
Tdec[{{p1, mu1}, {p1, mu2}, {p1, mu3}, {p1, mu4}, {p2, mu5}, {p2, mu6}},
{Q1, Q2}, Solve -> FerSolve, UseTIDL -> False, FCVerbose -> 1]
\end{verbatim}
\end{document}
