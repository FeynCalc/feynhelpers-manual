% !TeX program = pdflatex
% !TeX root = FerPath.tex

\documentclass[../FeynHelpersManual.tex]{subfiles}
\begin{document}
\hypertarget{ferpath}{
\section{FerPath}\label{ferpath}\index{FerPath}}

\texttt{FerPath} is an option for \texttt{FerRunScript} and other
functions of the Fermat interface and multiple other Fer* functions. It
specifies the full path to the Fermat binary.

If set to Automatic, Fermat binaries are expected to be located in
\texttt{FileNameJoin[\allowbreak{}\{\allowbreak{}\$FeynHelpersDirectory,\ \allowbreak{}"ExternalTools",\ \allowbreak{}"Fermat",\ \allowbreak{}"ferl6",\ \allowbreak{}"fer64"\}]}
and
\texttt{FileNameJoin[\allowbreak{}\{\allowbreak{}\$FeynHelpersDirectory,\ \allowbreak{}"ExternalTools",\ \allowbreak{}"Fermat",\ \allowbreak{}"ferm6",\ \allowbreak{}"fer64"\}]}
for Linux and macOS respectively.

\subsection{See also}

\hyperlink{toc}{Overview}, \hyperlink{fersolve}{FerSolve},
\hyperlink{ferrowreduce}{FerRowReduce},
\hyperlink{ferrunscript}{FerRunScript}.

\subsection{Examples}
\end{document}
