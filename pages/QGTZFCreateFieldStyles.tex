% !TeX program = pdflatex
% !TeX root = QGTZFCreateFieldStyles.tex

\documentclass[../FeynHelpersManual.tex]{subfiles}
\begin{document}
\hypertarget{qgtzfcreatefieldstyles}{
\section{QGTZFCreateFieldStyles}\label{qgtzfcreatefieldstyles}\index{QGTZFCreateFieldStyles}}

\texttt{QGTZFCreateTeXFiles[\allowbreak{}model_,\ \allowbreak{}output_]}
generates TikZ-Feynman stylings for the fields present in the QGRAF
model file \texttt{model}. The resulting file containing the stylings
set via \texttt{tikzset} and \texttt{tikzfeynmanset}is saved to
\texttt{output}.

It is also possible to invoke the function via
\texttt{QGTZFCreateTeXFiles[\allowbreak{}model,\ \allowbreak{}qgOutput]}
where qgOutput is the output \texttt{QGCreateAmp}.

The stylings can be generated in a semi-automatic fashion but for higher
quality results it is recommended to provide the necessary information
for each field via the option \texttt{QGFieldStyles}. It is a list of
lists, where each sublist contains the field name (e.g.~\texttt{El}),
its type (e.g.~\texttt{photon}, \texttt{boson}, \texttt{fermion},
\texttt{anti fermion} etc.) and its T EX label (e.g \texttt{\\gamma}).

\subsection{See also}

\hyperlink{toc}{Overview}, \hyperlink{qgconverttofc}{QGConvertToFC},
\hyperlink{qgcreateamp}{QGCreateAmp},
\hyperlink{qgtzfcreatetexfiles}{QGTZFCreateTeXFiles}.

\subsection{Examples}
\end{document}
