% !TeX program = pdflatex
% !TeX root = PaXD0Expand.tex

\documentclass[../FeynHelpersManual.tex]{subfiles}
\begin{document}
\hypertarget{paxd0expand}{
\section{PaXD0Expand}\label{paxd0expand}\index{PaXD0Expand}}

\texttt{PaXD0Expand} is an option for \texttt{PaXEvaluate}. If set to
\texttt{True}, Package-X function \texttt{D0Expand} will be applied to
the output of Package-X.

\subsection{See also}

\hyperlink{toc}{Overview}, \hyperlink{paxevaluate}{PaXEvaluate}.

\subsection{Examples}

\begin{Shaded}
\begin{Highlighting}[]
\NormalTok{D0}\OperatorTok{[}\DecValTok{0}\OperatorTok{,} \DecValTok{0}\OperatorTok{,} \DecValTok{0}\OperatorTok{,} \DecValTok{0}\OperatorTok{,} \FunctionTok{s}\OperatorTok{,} \FunctionTok{t}\OperatorTok{,} \FunctionTok{m}\SpecialCharTok{\^{}}\DecValTok{2}\OperatorTok{,} \FunctionTok{m}\SpecialCharTok{\^{}}\DecValTok{2}\OperatorTok{,} \FunctionTok{m}\SpecialCharTok{\^{}}\DecValTok{2}\OperatorTok{,} \FunctionTok{m}\SpecialCharTok{\^{}}\DecValTok{2}\OperatorTok{]}
\NormalTok{PaXEvaluate}\OperatorTok{[}\SpecialCharTok{\%}\OperatorTok{]}
\end{Highlighting}
\end{Shaded}

\begin{dmath*}\breakingcomma
\text{D}_0\left(0,0,0,0,s,t,m^2,m^2,m^2,m^2\right)
\end{dmath*}

\FloatBarrier
\begin{figure}[!ht]
\centering
\includegraphics[width=0.6\linewidth]{img/0nwzbpg5gzgtm.pdf}
\end{figure}
\FloatBarrier

\begin{dmath*}\breakingcomma
\text{D}_0\left(0,0,0,0,s,t,m^2,m^2,m^2,m^2\right)
\end{dmath*}

The full result is a \texttt{ConditionalExpression}

\begin{Shaded}
\begin{Highlighting}[]
\NormalTok{D0}\OperatorTok{[}\DecValTok{0}\OperatorTok{,} \DecValTok{0}\OperatorTok{,} \DecValTok{0}\OperatorTok{,} \DecValTok{0}\OperatorTok{,} \FunctionTok{s}\OperatorTok{,} \FunctionTok{t}\OperatorTok{,} \FunctionTok{m}\SpecialCharTok{\^{}}\DecValTok{2}\OperatorTok{,} \FunctionTok{m}\SpecialCharTok{\^{}}\DecValTok{2}\OperatorTok{,} \FunctionTok{m}\SpecialCharTok{\^{}}\DecValTok{2}\OperatorTok{,} \FunctionTok{m}\SpecialCharTok{\^{}}\DecValTok{2}\OperatorTok{]}
\NormalTok{res }\ExtensionTok{=}\NormalTok{ PaXEvaluate}\OperatorTok{[}\SpecialCharTok{\%}\OperatorTok{,}\NormalTok{ PaXC0Expand }\OtherTok{{-}\textgreater{}} \ConstantTok{True}\OperatorTok{]}\NormalTok{;}
\end{Highlighting}
\end{Shaded}

\begin{dmath*}\breakingcomma
\text{D}_0\left(0,0,0,0,s,t,m^2,m^2,m^2,m^2\right)
\end{dmath*}

\FloatBarrier
\begin{figure}[!ht]
\centering
\includegraphics[width=0.6\linewidth]{img/18tyhteyfoyw5.pdf}
\end{figure}
\FloatBarrier

\begin{Shaded}
\begin{Highlighting}[]
\NormalTok{res }\SpecialCharTok{//} \FunctionTok{Short}
\NormalTok{res }\SpecialCharTok{//} \FunctionTok{Last}
\end{Highlighting}
\end{Shaded}

\begin{dmath*}\breakingcomma
\text{D}_0\left(0,0,0,0,s,t,m^2,m^2,m^2,m^2\right)
\end{dmath*}

\begin{dmath*}\breakingcomma
m^2
\end{dmath*}

Use \texttt{Normal} to get the actual expression

\begin{Shaded}
\begin{Highlighting}[]
\NormalTok{(res }\SpecialCharTok{//} \FunctionTok{Normal}\NormalTok{)}
\end{Highlighting}
\end{Shaded}

\begin{dmath*}\breakingcomma
\text{D}_0\left(0,0,0,0,s,t,m^2,m^2,m^2,m^2\right)
\end{dmath*}
\end{document}
