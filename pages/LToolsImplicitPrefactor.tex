% !TeX program = pdflatex
% !TeX root = LToolsImplicitPrefactor.tex

\documentclass[../FeynHelpersManual.tex]{subfiles}
\begin{document}
\hypertarget{ltoolsimplicitprefactor}{
\section{LToolsImplicitPrefactor}\label{ltoolsimplicitprefactor}\index{LToolsImplicitPrefactor}}

\texttt{LToolsImplicitPrefactor} is an option for
\texttt{LToolsEvaluate}. It specifies a prefactor that does not show up
explicitly in the input expression, but is understood to appear in front
of every Passarino-Veltman function. The default value is \texttt{1}.

You may want to use \texttt{LToolsImplicitPrefactor->1/(2Pi)^D} when
working with 1-loop amplitudes, if no explicit prefactor has been
introduced from the very beginning.

\subsection{See also}

\hyperlink{toc}{Overview}, \hyperlink{ltoolsevaluate}{LToolsEvaluate}.

\subsection{Examples}

\begin{Shaded}
\begin{Highlighting}[]
\NormalTok{LToolsLoadLibrary}\OperatorTok{[]}
\end{Highlighting}
\end{Shaded}

\begin{dmath*}\breakingcomma
\text{LoopTools library loaded.}
\end{dmath*}

\begin{verbatim}
(* ====================================================
   FF 2.0, a package to evaluate one-loop integrals
 written by G. J. van Oldenborgh, NIKHEF-H, Amsterdam
 ====================================================
 for the algorithms used see preprint NIKHEF-H 89/17,
 'New Algorithms for One-loop Integrals', by G.J. van
 Oldenborgh and J.A.M. Vermaseren, published in 
 Zeitschrift fuer Physik C46(1990)425.
 ====================================================*)
\end{verbatim}

Here the prefactor \(i \pi^2\) arises from the conversion of
\(\int d^D q\, 1/(q^2-m^2)\) to \(A_0(m^2)\)

\begin{Shaded}
\begin{Highlighting}[]
\NormalTok{LToolsEvaluate}\OperatorTok{[}\NormalTok{FAD}\OperatorTok{[\{}\FunctionTok{q}\OperatorTok{,} \FunctionTok{m}\OperatorTok{\}],} \FunctionTok{q}\OperatorTok{,}\NormalTok{ InitialSubstitutions }\OtherTok{{-}\textgreater{}} \OperatorTok{\{}\FunctionTok{m} \OtherTok{{-}\textgreater{}} \DecValTok{5}\OperatorTok{\}]}
\end{Highlighting}
\end{Shaded}

\begin{dmath*}\breakingcomma
\frac{0.\, +246.74 i}{\varepsilon }-(0.\, +972.359 i)
\end{dmath*}

\begin{Shaded}
\begin{Highlighting}[]
\NormalTok{LToolsEvaluate}\OperatorTok{[}\NormalTok{FAD}\OperatorTok{[\{}\FunctionTok{q}\OperatorTok{,} \FunctionTok{m}\OperatorTok{\}],} \FunctionTok{q}\OperatorTok{,}\NormalTok{ InitialSubstitutions }\OtherTok{{-}\textgreater{}} \OperatorTok{\{}\FunctionTok{m} \OtherTok{{-}\textgreater{}} \DecValTok{5}\OperatorTok{\},} \FunctionTok{Head} \OtherTok{{-}\textgreater{}}\NormalTok{ keep}\OperatorTok{]}
\end{Highlighting}
\end{Shaded}

\begin{dmath*}\breakingcomma
\frac{i \pi ^2 \;\text{keep}(25.)}{\varepsilon }+i \pi ^2 (\text{keep}(-55.4719)-\gamma  \;\text{keep}(25.)-\text{keep}(25.) \log (\pi ))
\end{dmath*}

This recovers the textbook prefactor

\begin{Shaded}
\begin{Highlighting}[]
\NormalTok{LToolsEvaluate}\OperatorTok{[}\NormalTok{FAD}\OperatorTok{[\{}\FunctionTok{q}\OperatorTok{,} \FunctionTok{m}\OperatorTok{\}],} \FunctionTok{q}\OperatorTok{,}\NormalTok{ InitialSubstitutions }\OtherTok{{-}\textgreater{}} \OperatorTok{\{}\FunctionTok{m} \OtherTok{{-}\textgreater{}} \DecValTok{5}\OperatorTok{\},}\NormalTok{ LToolsImplicitPrefactor }\OtherTok{{-}\textgreater{}} \DecValTok{1}\SpecialCharTok{/}\NormalTok{(}\DecValTok{2} \FunctionTok{Pi}\NormalTok{)}\SpecialCharTok{\^{}}\NormalTok{(}\DecValTok{4} \SpecialCharTok{{-}} \DecValTok{2}\NormalTok{ Epsilon)}\OperatorTok{]}
\end{Highlighting}
\end{Shaded}

\begin{dmath*}\breakingcomma
\frac{0.\, +0.158314 i}{\varepsilon }-(0.\, +0.0419639 i)
\end{dmath*}

\begin{Shaded}
\begin{Highlighting}[]
\NormalTok{(PaXEvaluate}\OperatorTok{[}\NormalTok{FAD}\OperatorTok{[\{}\FunctionTok{q}\OperatorTok{,} \FunctionTok{m}\OperatorTok{\}],} \FunctionTok{q}\OperatorTok{,}\NormalTok{ PaXImplicitPrefactor }\OtherTok{{-}\textgreater{}} \DecValTok{1}\SpecialCharTok{/}\NormalTok{(}\DecValTok{2} \FunctionTok{Pi}\NormalTok{)}\SpecialCharTok{\^{}}\NormalTok{(}\DecValTok{4} \SpecialCharTok{{-}} \DecValTok{2}\NormalTok{ Epsilon)}\OperatorTok{]} \OtherTok{/.} \OperatorTok{\{}\FunctionTok{m} \OtherTok{{-}\textgreater{}} \DecValTok{5}\OperatorTok{,}\NormalTok{ ScaleMu}\SpecialCharTok{\^{}}\DecValTok{2} \OtherTok{{-}\textgreater{}} \DecValTok{1}\OperatorTok{\}}\NormalTok{) }\SpecialCharTok{//} \FunctionTok{N}
\end{Highlighting}
\end{Shaded}

\begin{dmath*}\breakingcomma
\frac{0.\, +0.158314 i}{\varepsilon }-(0.\, +0.0419639 i)
\end{dmath*}

If the input expression contains both loop and non-loop terms, only the
terms containing a \texttt{PaVe}-function will be multiplied by the
implicit prefactor

\begin{Shaded}
\begin{Highlighting}[]
\NormalTok{LToolsEvaluate}\OperatorTok{[}\NormalTok{extra }\SpecialCharTok{+}\NormalTok{ FAD}\OperatorTok{[\{}\FunctionTok{q}\OperatorTok{,} \FunctionTok{m}\OperatorTok{\}],} \FunctionTok{q}\OperatorTok{,}\NormalTok{ InitialSubstitutions }\OtherTok{{-}\textgreater{}} \OperatorTok{\{}\FunctionTok{m} \OtherTok{{-}\textgreater{}} \DecValTok{2}\OperatorTok{\},}\NormalTok{ LToolsExpandInEpsilon }\OtherTok{{-}\textgreater{}} \ConstantTok{False}\OperatorTok{]}
\end{Highlighting}
\end{Shaded}

\begin{dmath*}\breakingcomma
\frac{(0.\, +4. i) \pi ^{2-\varepsilon } \Gamma (\varepsilon +1) \Gamma (1-\varepsilon )^2}{\varepsilon  \Gamma (1-2 \varepsilon )}-\frac{(0.\, +1.54518 i) \pi ^{2-\varepsilon } \Gamma (\varepsilon +1) \Gamma (1-\varepsilon )^2}{\Gamma (1-2 \varepsilon )}+\text{extra}
\end{dmath*}
\end{document}
