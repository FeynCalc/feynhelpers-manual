% !TeX program = pdflatex
% !TeX root = KiraCreateJobFile.tex

\documentclass[../FeynHelpersManual.tex]{subfiles}
\begin{document}
\hypertarget{kiracreatejobfile}{
\section{KiraCreateJobFile}\label{kiracreatejobfile}\index{KiraCreateJobFile}}

KiraCreateJobFile{[}topo, sectors, \{r,s\}, path{]} can be used to
generate Kira job files from an \texttt{FCTopology} or a list thereof.
Here \texttt{sectors} is a list of sectors need to be reduced,
e.g.~\texttt{\{\allowbreak{}\{\allowbreak{}1,\ \allowbreak{}0,\ \allowbreak{}0,\ \allowbreak{}0\},\ \allowbreak{}\{\allowbreak{}1,\ \allowbreak{}1,\ \allowbreak{}0,\ \allowbreak{}0\},\ \allowbreak{}\{\allowbreak{}1,\ \allowbreak{}1,\ \allowbreak{}1,\ \allowbreak{}0\}\}}
etc. The values of \texttt{r} and \texttt{s} correspond to the maximal
positive and negative powers that may appear in the loop integrals to be
reduced.

The functions creates the corresponding yaml files and saves them in
\texttt{path/topoName}. Using
\texttt{KiraCreateJobFile[\allowbreak{}\{\allowbreak{}topo1,\ \allowbreak{}topo2,\ \allowbreak{}...\},\ \allowbreak{}\{\allowbreak{}sectors1,\ \allowbreak{}sectors2,\ \allowbreak{}...\},\ \allowbreak{}\{\allowbreak{}\{\allowbreak{}r1,\ \allowbreak{}s1\},\ \allowbreak{}\{\allowbreak{}r2,\ \allowbreak{}s2\},\ \allowbreak{}...\},\ \allowbreak{} path]}
will save the scripts to \texttt{path/topoName1},
\texttt{path/topoName2} etc. The syntax using
\texttt{KiraCreateJobFile[\allowbreak{}\{\allowbreak{}topo1,\ \allowbreak{}topo2,\ \allowbreak{}...\},\ \allowbreak{}\{\allowbreak{}sectors1,\ \allowbreak{}sectors2,\ \allowbreak{}...\},\ \allowbreak{}\{\allowbreak{}\{\allowbreak{}r1,\ \allowbreak{}s1\},\ \allowbreak{}\{\allowbreak{}r2,\ \allowbreak{}s2\},\ \allowbreak{}...\},\ \allowbreak{} \{\allowbreak{}path1,\ \allowbreak{}path2,\ \allowbreak{}...\}]}
is also possible.

It is also possible to supply a list of \texttt{GLI}s instead of
sectors. In that case \texttt{FCLoopFindSectors} and \texttt{KiraGetRS}
will be used to determine the top sector for each topology.

The syntax
\texttt{KiraCreateJobFile[\allowbreak{}\{\allowbreak{}topo1,\ \allowbreak{}topo2,\ \allowbreak{}...\},\ \allowbreak{}\{\allowbreak{}sectors1,\ \allowbreak{}sectors2,\ \allowbreak{}...\},\ \allowbreak{}\{\allowbreak{}\{\allowbreak{}r1,\ \allowbreak{}s1\},\ \allowbreak{}\{\allowbreak{}r2,\ \allowbreak{}s2\},\ \allowbreak{}...\},\ \allowbreak{}path]}
or
\texttt{KiraCreateJobFile[\allowbreak{}\{\allowbreak{}topo1,\ \allowbreak{}topo2,\ \allowbreak{}...\},\ \allowbreak{}\{\allowbreak{}glis1,\ \allowbreak{}glis2,\ \allowbreak{}...\},\ \allowbreak{} path]}
is also allowed. This implies that all config files will go into the
corresponding subdirectories of path,
e.g.~\texttt{path/topoName1/config}, \texttt{path/topoName2/config} etc.

The default name for job files is \texttt{job.yaml} and can be changed
via the option \texttt{KiraJobFileName}.

\subsection{See also}

\hyperlink{toc}{Overview},
\hyperlink{kiracreateconfigfiles}{KiraCreateConfigFiles},
\hyperlink{kirajobfilename}{KiraJobFileName},
\hyperlink{kiraintegrals}{KiraIntegrals}

\subsection{Examples}

\begin{Shaded}
\begin{Highlighting}[]
\NormalTok{topo }\ExtensionTok{=}\NormalTok{ FCTopology}\OperatorTok{[}\NormalTok{prop3lX1}\OperatorTok{,} \OperatorTok{\{}\NormalTok{SFAD}\OperatorTok{[\{}\NormalTok{p1}\OperatorTok{,} \FunctionTok{m}\SpecialCharTok{\^{}}\DecValTok{2}\OperatorTok{\}],}\NormalTok{ SFAD}\OperatorTok{[}\NormalTok{p2}\OperatorTok{],}\NormalTok{ SFAD}\OperatorTok{[\{}\NormalTok{p3}\OperatorTok{,} \FunctionTok{m}\SpecialCharTok{\^{}}\DecValTok{2}\OperatorTok{\}],}\NormalTok{ SFAD}\OperatorTok{[}\FunctionTok{Q} \SpecialCharTok{{-}}\NormalTok{ p1 }\SpecialCharTok{{-}}\NormalTok{ p2 }\SpecialCharTok{{-}}\NormalTok{ p3}\OperatorTok{],} 
\NormalTok{    SFAD}\OperatorTok{[}\FunctionTok{Q} \SpecialCharTok{{-}}\NormalTok{ p1 }\SpecialCharTok{{-}}\NormalTok{ p2}\OperatorTok{],}\NormalTok{ SFAD}\OperatorTok{[}\FunctionTok{Q} \SpecialCharTok{{-}}\NormalTok{ p1}\OperatorTok{],}\NormalTok{ SFAD}\OperatorTok{[}\FunctionTok{Q} \SpecialCharTok{{-}}\NormalTok{ p2}\OperatorTok{],}\NormalTok{ SFAD}\OperatorTok{[}\NormalTok{p1 }\SpecialCharTok{+}\NormalTok{ p3}\OperatorTok{],}\NormalTok{ SFAD}\OperatorTok{[}\NormalTok{p2 }\SpecialCharTok{+}\NormalTok{ p3}\OperatorTok{]\},} \OperatorTok{\{}\NormalTok{p1}\OperatorTok{,}\NormalTok{ p2}\OperatorTok{,}\NormalTok{ p3}\OperatorTok{\},} \OperatorTok{\{}\FunctionTok{Q}\OperatorTok{\},} \OperatorTok{\{\},} \OperatorTok{\{\}]}
\end{Highlighting}
\end{Shaded}

\begin{dmath*}\breakingcomma
\text{FCTopology}\left(\text{prop3lX1},\left\{\frac{1}{(\text{p1}^2-m^2+i \eta )},\frac{1}{(\text{p2}^2+i \eta )},\frac{1}{(\text{p3}^2-m^2+i \eta )},\frac{1}{((-\text{p1}-\text{p2}-\text{p3}+Q)^2+i \eta )},\frac{1}{((-\text{p1}-\text{p2}+Q)^2+i \eta )},\frac{1}{((Q-\text{p1})^2+i \eta )},\frac{1}{((Q-\text{p2})^2+i \eta )},\frac{1}{((\text{p1}+\text{p3})^2+i \eta )},\frac{1}{((\text{p2}+\text{p3})^2+i \eta )}\right\},\{\text{p1},\text{p2},\text{p3}\},\{Q\},\{\},\{\}\right)
\end{dmath*}

\begin{Shaded}
\begin{Highlighting}[]
\NormalTok{KiraCreateJobFile}\OperatorTok{[}\NormalTok{topo}\OperatorTok{,} \OperatorTok{\{\{}\DecValTok{1}\OperatorTok{,} \DecValTok{1}\OperatorTok{,} \DecValTok{1}\OperatorTok{,} \DecValTok{1}\OperatorTok{,} \DecValTok{1}\OperatorTok{,} \DecValTok{1}\OperatorTok{,} \DecValTok{1}\OperatorTok{,} \DecValTok{1}\OperatorTok{,} \DecValTok{1}\OperatorTok{\}\},} \OperatorTok{\{}\DecValTok{4}\OperatorTok{,} \DecValTok{4}\OperatorTok{\},} \FunctionTok{FileNameJoin}\OperatorTok{[\{}\NormalTok{$FeynCalcDirectory}\OperatorTok{,} \StringTok{"Database"}\OperatorTok{\}]]}
\end{Highlighting}
\end{Shaded}

\begin{dmath*}\breakingcomma
\text{/home/vs/.Mathematica/Applications/FeynCalc/Database/prop3lX1/job.yaml}
\end{dmath*}

\begin{Shaded}
\begin{Highlighting}[]
\NormalTok{KiraCreateJobFile}\OperatorTok{[}\NormalTok{topo}\OperatorTok{,} \OperatorTok{\{}
\NormalTok{   GLI}\OperatorTok{[}\NormalTok{prop3lX1}\OperatorTok{,} \OperatorTok{\{}\DecValTok{1}\OperatorTok{,} \DecValTok{1}\OperatorTok{,} \DecValTok{1}\OperatorTok{,} \DecValTok{1}\OperatorTok{,} \DecValTok{1}\OperatorTok{,} \DecValTok{1}\OperatorTok{,} \DecValTok{1}\OperatorTok{,} \DecValTok{0}\OperatorTok{,} \DecValTok{0}\OperatorTok{\}],} 
\NormalTok{   GLI}\OperatorTok{[}\NormalTok{prop3lX1}\OperatorTok{,} \OperatorTok{\{}\DecValTok{1}\OperatorTok{,} \DecValTok{1}\OperatorTok{,} \DecValTok{1}\OperatorTok{,} \DecValTok{1}\OperatorTok{,} \DecValTok{1}\OperatorTok{,} \DecValTok{1}\OperatorTok{,} \DecValTok{1}\OperatorTok{,} \DecValTok{0}\OperatorTok{,} \DecValTok{1}\OperatorTok{\}]\},} 
  \FunctionTok{FileNameJoin}\OperatorTok{[\{}\NormalTok{$FeynCalcDirectory}\OperatorTok{,} \StringTok{"Database"}\OperatorTok{\}],}\NormalTok{ KiraJobFileName }\OtherTok{{-}\textgreater{}} \StringTok{"job2.yaml"}\OperatorTok{]}
\end{Highlighting}
\end{Shaded}

\begin{dmath*}\breakingcomma
\text{/home/vs/.Mathematica/Applications/FeynCalc/Database/prop3lX1/job2.yaml}
\end{dmath*}
\end{document}
