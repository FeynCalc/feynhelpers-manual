% !TeX program = pdflatex
% !TeX root = PaXImplicitPrefactor.tex

\documentclass[../FeynHelpersManual.tex]{subfiles}
\begin{document}
\FloatBarrier
\begin{figure}[!ht]
\centering
\includegraphics[width=0.6\linewidth]{img/1ne6vpix78l6i.pdf}
\end{figure}
\FloatBarrier

\hypertarget{paximplicitprefactor}{
\section{PaXImplicitPrefactor}\label{paximplicitprefactor}\index{PaXImplicitPrefactor}}

\texttt{PaXImplicitPrefactor} is an option for \texttt{PaXEvaluate}. It
specifies the prefactor that does not show up explicitly in the input
expression, but is understood to appear in front of every 1-loop
integral. For technical reasons, \texttt{PaXImplicitPrefactor} should
not depend on the number of dimensions \texttt{D}. Instead you should
explicitly specify what \texttt{D} is (e.g.~\texttt{4-2 Epsilon}). The
default value is \texttt{1}.

If the standard prefactor \(1/(2 \pi)^D\) is implicit in your
calculations, use \texttt{ImplicitPrefactor -> 1/(2 Pi)^(4 - 2 Epsilon)}
.

\subsection{See also}

\hyperlink{toc}{Overview}, \hyperlink{paxevaluate}{PaXEvaluate}.

\subsection{Examples}
\end{document}
