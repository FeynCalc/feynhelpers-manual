% !TeX program = pdflatex
% !TeX root = LToolsSetLambda.tex

\documentclass[../FeynHelpersManual.tex]{subfiles}
\begin{document}
\hypertarget{ltoolssetlambda}{
\section{LToolsSetLambda}\label{ltoolssetlambda}\index{LToolsSetLambda}}

\texttt{LToolsSetLambda} corresponds to the \texttt{SetLambda} function
in LoopTools.

See \texttt{?LoopTools\textasciigrave SetLambda} for further information
regarding this LoopTools symbol.

\texttt{LToolsSetLambda} is also an option for \texttt{LToolsEvaluate}
that sets the numerical value for the IR regularization parameter
\(\lambda^2\).

Setting \(\lambda^2\) to \texttt{-2} or \texttt{-1} will make LoopTools
return the coefficients of the \(1/\varepsilon\) and
\(1/\varepsilon\)-poles respectively. The value \texttt{0} yields the
finite part of the integral where IR divergences are regularized
dimensionally.

When \(\lambda^2\) is set to some positive value (say \texttt{2.}),
\texttt{LoopTools} will return the finite part of the integral with IR
divergences being regularized using a fictitious mass. The result will
naturally depend on the value of \(\lambda^2\).

It is important to keep in mind that for \(\lambda^2 = -1\) LoopTools
also returns the UV-pole, although this not so clearly stated in the
official manual.

Notice that the option \texttt{LToolsSetLambda} is ignored, as long as
\texttt{LToolsFullResult} is set to \texttt{True}.

\subsection{See also}

\hyperlink{toc}{Overview}, \hyperlink{ltoolsevaluate}{LToolsEvaluate}

\subsection{Examples}
\end{document}
