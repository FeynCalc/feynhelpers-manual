% !TeX program = pdflatex
% !TeX root = LToolsFullResult.tex

\documentclass[../FeynHelpersManual.tex]{subfiles}
\begin{document}
\hypertarget{ltoolsfullresult}{
\section{LToolsFullResult}\label{ltoolsfullresult}\index{LToolsFullResult}}

\texttt{LToolsFullResult} is an option for \texttt{LToolsEvaluate}. When
set to \texttt{True} (default), \texttt{LToolsEvaluate} will return the
full result including singularities and accompanying terms. Otherwise,
only the finite part (standard output of LoopTools) will be provided.

The full result is assembled from pieces returned by LoopTools for the
\(\lambda^2\)-parameter set to \(-2\), \(-1\) and \(0\) respectively.
The correct prefactor that accounts for the normalization differences
between Passarino-Veltman function in FeynCalc and LoopTools is added as
well.

As long as \texttt{LToolsFullResult} is set to \texttt{True}, the value
of the \texttt{LToolsSetLambda} option is ignored.

Disabling \texttt{LToolsFullResult} will most likely lead to incorrect
normalization of the results (especially if you are only interested in
the finite part). The reason for this are missing contributions to the
finite part generated from poles being multiplied by terms proportional
to \(\varepsilon\) or \(\varepsilon^2\).

\subsection{See also}

\hyperlink{toc}{Overview}, \hyperlink{ltoolsevaluate}{LToolsEvaluate}.

\subsection{Examples}

\begin{Shaded}
\begin{Highlighting}[]
\NormalTok{LToolsLoadLibrary}\OperatorTok{[]}\NormalTok{;}
\end{Highlighting}
\end{Shaded}

\begin{dmath*}\breakingcomma
\text{LoopTools library loaded.}
\end{dmath*}

\begin{Shaded}
\begin{Highlighting}[]
\CommentTok{(* ====================================================}
\CommentTok{   FF 2.0, a package to evaluate one{-}loop integrals}
\CommentTok{ written by G. J. van Oldenborgh, NIKHEF{-}H, Amsterdam}
\CommentTok{ ====================================================}
\CommentTok{ for the algorithms used see preprint NIKHEF{-}H 89/17,}
\CommentTok{ \textquotesingle{}New Algorithms for One{-}loop Integrals\textquotesingle{}, by G.J. van}
\CommentTok{ Oldenborgh and J.A.M. Vermaseren, published in }
\CommentTok{ Zeitschrift fuer Physik C46(1990)425.}
\CommentTok{ ====================================================*)}
\end{Highlighting}
\end{Shaded}

\begin{Shaded}
\begin{Highlighting}[]
\NormalTok{LToolsEvaluate}\OperatorTok{[}\NormalTok{A0}\OperatorTok{[}\FunctionTok{m}\SpecialCharTok{\^{}}\DecValTok{2}\OperatorTok{],}\NormalTok{ InitialSubstitutions }\OtherTok{{-}\textgreater{}} \OperatorTok{\{}\FunctionTok{m}\SpecialCharTok{\^{}}\DecValTok{2} \OtherTok{{-}\textgreater{}} \DecValTok{1}\OperatorTok{\}]}
\end{Highlighting}
\end{Shaded}

\begin{dmath*}\breakingcomma
\frac{1.}{\varepsilon }-0.721946
\end{dmath*}

Setting \texttt{LToolsFullResult} to \texttt{False} will make
\texttt{LToolsEvaluate} return only the finite part since the default
value for \texttt{LToolsSetLambda} is \texttt{0}. However, the
normalization does not agree with the FeynCalc convention

\begin{Shaded}
\begin{Highlighting}[]
\NormalTok{LToolsEvaluate}\OperatorTok{[}\NormalTok{A0}\OperatorTok{[}\FunctionTok{m}\SpecialCharTok{\^{}}\DecValTok{2}\OperatorTok{],}\NormalTok{ InitialSubstitutions }\OtherTok{{-}\textgreater{}} \OperatorTok{\{}\FunctionTok{m}\SpecialCharTok{\^{}}\DecValTok{2} \OtherTok{{-}\textgreater{}} \DecValTok{1}\OperatorTok{\},}\NormalTok{ LToolsFullResult }\OtherTok{{-}\textgreater{}} \ConstantTok{False}\OperatorTok{]}
\end{Highlighting}
\end{Shaded}

\begin{dmath*}\breakingcomma
1.
\end{dmath*}

Even though \texttt{LToolsEvaluate} includes the correct prefactor to
convert to the FeynCalc normalization, the finite contribution generated
by the \(1/\varepsilon\)-pole is missing here.

\begin{Shaded}
\begin{Highlighting}[]
\NormalTok{finRes }\ExtensionTok{=}\NormalTok{ LToolsEvaluate}\OperatorTok{[}\NormalTok{A0}\OperatorTok{[}\FunctionTok{m}\SpecialCharTok{\^{}}\DecValTok{2}\OperatorTok{],}\NormalTok{ InitialSubstitutions }\OtherTok{{-}\textgreater{}} \OperatorTok{\{}\FunctionTok{m}\SpecialCharTok{\^{}}\DecValTok{2} \OtherTok{{-}\textgreater{}} \DecValTok{1}\OperatorTok{\},}\NormalTok{ LToolsFullResult }\OtherTok{{-}\textgreater{}} \ConstantTok{False}\OperatorTok{,}\NormalTok{ LToolsExpandInEpsilon }\OtherTok{{-}\textgreater{}} \ConstantTok{False}\OperatorTok{]}
\end{Highlighting}
\end{Shaded}

\begin{dmath*}\breakingcomma
\frac{1. \pi ^{-\varepsilon } \Gamma (1-\varepsilon )^2 \Gamma (\varepsilon +1)}{\Gamma (1-2 \varepsilon )}
\end{dmath*}

By setting \texttt{LToolsSetLambda->-1} we can get the coefficient of
the pole. Here it is obvious that the function is IR-finite so that we
do not need to check for the \(1/\varepsilon^2\)-pole

\begin{Shaded}
\begin{Highlighting}[]
\NormalTok{poleRes }\ExtensionTok{=}\NormalTok{ LToolsEvaluate}\OperatorTok{[}\NormalTok{A0}\OperatorTok{[}\FunctionTok{m}\SpecialCharTok{\^{}}\DecValTok{2}\OperatorTok{],}\NormalTok{ InitialSubstitutions }\OtherTok{{-}\textgreater{}} \OperatorTok{\{}\FunctionTok{m}\SpecialCharTok{\^{}}\DecValTok{2} \OtherTok{{-}\textgreater{}} \DecValTok{1}\OperatorTok{\},}\NormalTok{ LToolsFullResult }\OtherTok{{-}\textgreater{}} \ConstantTok{False}\OperatorTok{,}\NormalTok{ LToolsExpandInEpsilon }\OtherTok{{-}\textgreater{}} \ConstantTok{False}\OperatorTok{,}\NormalTok{ LToolsSetLambda }\OtherTok{{-}\textgreater{}} \SpecialCharTok{{-}}\DecValTok{1}\OperatorTok{]}
\end{Highlighting}
\end{Shaded}

\begin{dmath*}\breakingcomma
\frac{1. \pi ^{-\varepsilon } \Gamma (1-\varepsilon )^2 \Gamma (\varepsilon +1)}{\Gamma (1-2 \varepsilon )}
\end{dmath*}

Combining both pieces and expanding in \(\varepsilon\) up to zeroth
order we recover the same result as when using the option
\texttt{LToolsFullResult}

\begin{Shaded}
\begin{Highlighting}[]
\FunctionTok{Series}\OperatorTok{[}\DecValTok{1}\SpecialCharTok{/}\NormalTok{Epsilon poleRes }\SpecialCharTok{+}\NormalTok{ finRes}\OperatorTok{,} \OperatorTok{\{}\NormalTok{Epsilon}\OperatorTok{,} \DecValTok{0}\OperatorTok{,} \DecValTok{0}\OperatorTok{\}]} \SpecialCharTok{//} \FunctionTok{Normal}
\end{Highlighting}
\end{Shaded}

\begin{dmath*}\breakingcomma
\frac{1.}{\varepsilon }-0.721946
\end{dmath*}
\end{document}
