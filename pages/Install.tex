% !TeX program = pdflatex
% !TeX root = Install.tex

\documentclass[../FeynHelpersManual.tex]{subfiles}
\begin{document}
\hypertarget{installation}{
\section{Installation}\label{installation}\index{Installation}}

The installation of FeynHelpers from scratch is a two step process.
Apart from installing the actual FeynCalc addon (step 1) it is also
necessary to set up the tools to which FeynHelpers provides FeynCalc
interfaces (step 2).

Due to the fact that these tools are written in different programming
languages and usually must be compiled on user's machine, it is not
possible to fully automatize this second step. Instead, we expect the
user to install the relevant tools manually.

In the following we provide some relevant instructions but without any
warranty that this will work on your machine. \emph{If you are
experiencing issues setting up a particular tools please contact the
corresponding developer team.}

\hypertarget{feynhelpers}{%
\subsection{FeynHelpers}\label{feynhelpers}}

\hypertarget{automatic-installation}{%
\subsubsection{Automatic installation}\label{automatic-installation}}

Run the following instruction in a Kernel or Notebook session of
Mathematica to install the stable version

\begin{verbatim}
Import["https://raw.githubusercontent.com/FeynCalc/feynhelpers/master/install.m"]
InstallFeynHelpers[]
\end{verbatim}

If you like the bleeding edge and you are already using the development
version of FeynCalc, you can also install the development version of
FeynHelpers

\begin{verbatim}
Import["https://raw.githubusercontent.com/FeynCalc/feynhelpers/master/install.m"]
InstallFeynHelpers[InstallFeynHelpersDevelopmentVersion->True]
\end{verbatim}

\hypertarget{manual-installation}{%
\subsubsection{Manual installation}\label{manual-installation}}

Create a directory \emph{FeynHelpers} inside

\begin{verbatim}
FileNameJoin[{$UserBaseDirectory, "Applications", "FeynCalc", "AddOns"}]
\end{verbatim}

and put the source code there.

\hypertarget{fermat}{%
\subsection{Fermat}\label{fermat}}

You can download FERMAT binaries for Linux or macOS from the
\href{https://home.bway.net/lewis/zip.html}{developer's website}.

\hypertarget{linux}{%
\subsubsection{Linux}\label{linux}}

Copy the directory \texttt{ferl64} to
\texttt{FileNameJoin[\allowbreak{}\{\allowbreak{}\$UserBaseDirectory,\ \allowbreak{}"Applications",\ \allowbreak{}"FeynCalc",\ \allowbreak{}"AddOns",\ \allowbreak{}"FeynHelpers",\ \allowbreak{}"ExternalTools",\ \allowbreak{}"Fermat"\}]}.

\hypertarget{macos}{%
\subsubsection{macOS}\label{macos}}

Copy the directory \texttt{ferm64} to
\texttt{FileNameJoin[\allowbreak{}\{\allowbreak{}\$UserBaseDirectory,\ \allowbreak{}"Applications",\ \allowbreak{}"FeynCalc",\ \allowbreak{}"AddOns",\ \allowbreak{}"FeynHelpers",\ \allowbreak{}"ExternalTools",\ \allowbreak{}"Fermat"\}]}.

\hypertarget{windows}{%
\subsubsection{Windows}\label{windows}}

Currently there is no native Windows version of FERMAT. The Linux
version appears to be usable via WSL, but currently there is no support
for that in FeynHelpers.

\hypertarget{fire}{%
\subsection{FIRE}\label{fire}}

You can download the source code of FIRE from the
\href{https://bitbucket.org/feynmanIntegrals/fire}{developer's website}.
The content should be extracted to
\texttt{FileNameJoin[\allowbreak{}\{\allowbreak{}\$UserBaseDirectory,\ \allowbreak{}"Applications",\ \allowbreak{}"FIRE"\}]}.

\hypertarget{linux-1}{%
\subsubsection{Linux}\label{linux-1}}

The instructions for compiling FIRE from source on Linux are provided
\href{https://bitbucket.org/feynmanIntegrals/fire/src/master/README.md}{here}.

\hypertarget{macos-1}{%
\subsubsection{macOS}\label{macos-1}}

The instructions for compiling FIRE from source on macOS can be found
\href{https://bitbucket.org/feynmanIntegrals/fire/issues/10/issue-of-the-installation-on-macos}{here}.

\hypertarget{windows-1}{%
\subsubsection{Windows}\label{windows-1}}

There is no native Windows port of FIRE. It should be possible to
compile FIRE on WSL with an Ubuntu installation, but currently there is
no support for that in FeynHelpers.

\hypertarget{looptools}{%
\subsection{LoopTools}\label{looptools}}

On the \href{http://www.feynarts.de/looptools/}{developer's website} you
can download not only the source code but also precompiled binaries

\hypertarget{linux-or-macos}{%
\subsubsection{Linux or macOS}\label{linux-or-macos}}

Copy the self-compiled or precompiled MathLink executable
\texttt{LoopTools} to
\texttt{FileNameJoin[\allowbreak{}\{\allowbreak{}\$UserBaseDirectory,\ \allowbreak{}"Applications",\ \allowbreak{}"FeynCalc",\ \allowbreak{}"AddOns",\ \allowbreak{}"FeynHelpers",\ \allowbreak{}"ExternalTools",\ \allowbreak{}"LoopTools"\}]}.

\hypertarget{windows-2}{%
\subsubsection{Windows}\label{windows-2}}

Rename the self-compiled or precompiled MathLink executable
\texttt{LoopTools.exe} to \texttt{LoopTools} and copy it to
\texttt{FileNameJoin[\allowbreak{}\{\allowbreak{}\$UserBaseDirectory,\ \allowbreak{}"Applications",\ \allowbreak{}"FeynCalc",\ \allowbreak{}"AddOns",\ \allowbreak{}"FeynHelpers",\ \allowbreak{}"ExternalTools",\ \allowbreak{}"LoopTools"\}]}.

\hypertarget{pysecdec}{%
\subsection{pySecDec}\label{pysecdec}}

The installation instructions for pySecDec can be found
\href{https://secdec.readthedocs.io/en/stable/installation.html\#download-the-program-and-install}{here}.
FeynHelpers does not require pySecDec to \emph{generate} the
corresponding Python scripts. However, in order to actually compute the
given loop integrals one has to run those scripts, which is possible
only when pySecDec is installed.

\hypertarget{linux-or-macos-1}{%
\subsubsection{Linux or macOS}\label{linux-or-macos-1}}

It should be possible to install pySecDec via pip automatically.

\hypertarget{windows-3}{%
\subsubsection{Windows}\label{windows-3}}

It could be possible to set up pySecDec on WSL, but currently it is
unclear whether this can work.

\hypertarget{package-x}{%
\subsection{Package-X}\label{package-x}}

Currently the installation of Package-X is handled by the automatic
installer. Since Package-X is a Mathematica package that only needs to
be copied to the correct location, the installation is easy to
automatize.

Alternatively, you can download Package-X from the
\href{https://packagex.hepforge.org/}{developer's website} and copy the
folder \texttt{X} to
\texttt{FileNameJoin[\allowbreak{}\{\allowbreak{}\$UserBaseDirectory,\ \allowbreak{}"Applications"\}]}.
If you can load Package-X by evaluating \texttt{X\textasciigrave } in a
notebook, then the installation was successful.

\hypertarget{qgraf}{%
\subsection{QGRAF}\label{qgraf}}

You can download the source code of QFRAF from the
\href{http://cfif.ist.utl.pt/~paulo/qgraf.html}{developer's website}.

\hypertarget{linux-and-macos}{%
\subsubsection{Linux and macOS}\label{linux-and-macos}}

The compilation instructions can be found in the section ``Compiling''
of the manual \texttt{qgraf-3.6.0.pdf} inside the tarball. Having
compiled the program you need to rename the binary to \texttt{qgraf} and
put it to
\texttt{FileNameJoin[\allowbreak{}\{\allowbreak{}\$UserBaseDirectory,\ \allowbreak{}"Applications",\ \allowbreak{}"FeynCalc",\ \allowbreak{}"AddOns",\ \allowbreak{}"FeynHelpers",\ \allowbreak{}"ExternalTools",\ \allowbreak{}"QGRAF",\ \allowbreak{}"Binary"\}]}.

\hypertarget{windows-4}{%
\subsubsection{Windows}\label{windows-4}}

To extract the source code tarball of QGRAF you need a tool that can
deal with tar.gz archives, e.g. \href{https://www.7-zip.org/}{7-zip}.

To build QGRAF from source you need a FORTRAN compiler. You can use the
\href{https://sourceforge.net/projects/mingw-w64/files/mingw-w64}{MinGW
compiler} via the \emph{MinGW-W64 Online Installer}
(\texttt{MinGW-W64-install.exe}). When the \texttt{Settings} page
appears in the installation wizard, change \texttt{Architecture} from
\texttt{i686} to \texttt{x86_64}.

Unfortunately, as of June 2022 the installer is broken and fails with
the error message ``file was downloaded incorrectly''. A possible
workaround is described
\href{https://sourceforge.net/p/mingw-w64/support-requests/125/}{here}.
When you reach the \texttt{Installation folder} page in the installation
wizard, open

\begin{verbatim}
C:\Users\YOUR_USER_NAME\AppData\Local\Temp\gentee
\end{verbatim}

and drop there the file
\texttt{x86_64-8.1.0-release-win32-seh-rt_v6-rev0.7z} that you can
donwload from
\href{https://sourceforge.net/projects/mingw-w64/files/Toolchains\%20targetting\%20Win64/Personal\%20Builds/mingw-builds/8.1.0/threads-posix/seh/x86_64-8.1.0-release-posix-seh-rt_v6-rev0.7z}{SourceForge}.
This skips the broken process of downloading the file by the installer
and should get you through the installation.

Finally, open \texttt{MiniGW-W64 project} -\textgreater{}
\texttt{Open Terminal} via the start menu. Go to the folder where you
extracted the source code of QGRAF and compile it with
\texttt{gfortran.exe -static qgraf-3.x.y.f08 -o qgraf.exe}, where
\texttt{x} and \texttt{y} denote the current version numbers.

Run \texttt{qgraf.exe} to make sure that it works properly.

Put \texttt{qgraf.exe} to
\texttt{FileNameJoin[\allowbreak{}\{\allowbreak{}\$UserBaseDirectory,\ \allowbreak{}"Applications",\ \allowbreak{}"FeynCalc",\ \allowbreak{}"AddOns",\ \allowbreak{}"FeynHelpers",\ \allowbreak{}"ExternalTools",\ \allowbreak{}"QGRAF",\ \allowbreak{}"Binary"\}]}.
\end{document}
