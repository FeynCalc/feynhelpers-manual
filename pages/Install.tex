% !TeX program = pdflatex
% !TeX root = Install.tex

\documentclass[../FeynHelpersManual.tex]{subfiles}
\begin{document}
\hypertarget{installation}{
\section{Installation}\label{installation}\index{Installation}}

The installation of FeynHelpers from scratch is a two step process.
Apart from installing the actual FeynCalc addon (step 1) it is also
necessary to set up the tools to which FeynHelpers provides FeynCalc
interfaces (step 2).

Due to the fact that these tools are written in different programming
languages and usually must be compiled on user's machine, it is not
possible to fully automatize this second step. Instead, we expect the
user to install the relevant tools manually.

In the following we provide some relevant instructions but without any
warranty that this will work on your machine. \emph{If you are
experiencing issues setting up a particular tools please contact the
corresponding developer team.}

\hypertarget{feynhelpers}{%
\subsection{FeynHelpers}\label{feynhelpers}}

\hypertarget{automatic-installation}{%
\subsubsection{Automatic installation}\label{automatic-installation}}

\begin{itemize}
\item
  Run the following instruction in a Kernel or Notebook session of
  Mathematica to install the stable version

\begin{verbatim}
Import["https://raw.githubusercontent.com/FeynCalc/feynhelpers/master/install.m"]
InstallFeynHelpers[]
\end{verbatim}
\item
  If you like the bleeding edge and you are already using the
  development version of FeynCalc, you can also install the development
  version of FeynHelpers
\end{itemize}

\begin{verbatim}
Import["https://raw.githubusercontent.com/FeynCalc/feynhelpers/master/install.m"]
InstallFeynHelpers[InstallFeynHelpersDevelopmentVersion->True]
\end{verbatim}

\hypertarget{manual-installation}{%
\subsubsection{Manual installation}\label{manual-installation}}

\begin{itemize}
\item
  Create a directory \emph{FeynHelpers} inside

\begin{verbatim}
FileNameJoin[{$UserBaseDirectory, "Applications", "FeynCalc", "AddOns"}]
\end{verbatim}

  and put the source code there.
\end{itemize}

\hypertarget{fermat}{%
\subsection{Fermat}\label{fermat}}

You can download FERMAT binaries for Linux or macOS from the
\href{https://home.bway.net/lewis/zip.html}{developer's website}.

\hypertarget{linux}{%
\subsubsection{Linux}\label{linux}}

\begin{itemize}
\tightlist
\item
  Copy the directory \texttt{ferl64} to
  \texttt{FileNameJoin[\allowbreak{}\{\allowbreak{}\$UserBaseDirectory,\ \allowbreak{}"Applications",\ \allowbreak{}"FeynCalc",\ \allowbreak{}"AddOns",\ \allowbreak{}"FeynHelpers",\ \allowbreak{}"ExternalTools",\ \allowbreak{}"Fermat"\}]}.
\end{itemize}

\hypertarget{macos}{%
\subsubsection{macOS}\label{macos}}

\begin{itemize}
\tightlist
\item
  Copy the directory \texttt{ferm64} to
  \texttt{FileNameJoin[\allowbreak{}\{\allowbreak{}\$UserBaseDirectory,\ \allowbreak{}"Applications",\ \allowbreak{}"FeynCalc",\ \allowbreak{}"AddOns",\ \allowbreak{}"FeynHelpers",\ \allowbreak{}"ExternalTools",\ \allowbreak{}"Fermat"\}]}.
\end{itemize}

\hypertarget{windows}{%
\subsubsection{Windows}\label{windows}}

\begin{itemize}
\tightlist
\item
  Currently there is no native Windows version of FERMAT. The Linux
  version appears to be usable via WSL, but currently there is no
  support for that in FeynHelpers.
\end{itemize}

\hypertarget{fire}{%
\subsection{FIRE}\label{fire}}

You can download the source code of FIRE from the
\href{https://bitbucket.org/feynmanIntegrals/fire}{developer's website}.
The content should be extracted to
\texttt{FileNameJoin[\allowbreak{}\{\allowbreak{}\$UserBaseDirectory,\ \allowbreak{}"Applications",\ \allowbreak{}"FIRE"\}]}.

\hypertarget{linux-1}{%
\subsubsection{Linux}\label{linux-1}}

\begin{itemize}
\tightlist
\item
  The instructions for compiling FIRE from source on Linux are provided
  \href{https://bitbucket.org/feynmanIntegrals/fire/src/master/README.md}{here}.
\end{itemize}

\hypertarget{macos-1}{%
\subsubsection{macOS}\label{macos-1}}

\begin{itemize}
\tightlist
\item
  The instructions for compiling FIRE from source on macOS can be found
  \href{https://bitbucket.org/feynmanIntegrals/fire/issues/10/issue-of-the-installation-on-macos}{here}.
\end{itemize}

\hypertarget{windows-1}{%
\subsubsection{Windows}\label{windows-1}}

\begin{itemize}
\tightlist
\item
  There is no native Windows port of FIRE. It should be possible to
  compile FIRE on WSL with an Ubuntu installation, but currently there
  is no support for that in FeynHelpers.
\end{itemize}

\hypertarget{looptools}{%
\subsection{LoopTools}\label{looptools}}

On the \href{http://www.feynarts.de/looptools/}{developer's website} you
can download not only the source code but also precompiled binaries

\hypertarget{linux-or-macos}{%
\subsubsection{Linux or macOS}\label{linux-or-macos}}

\begin{itemize}
\tightlist
\item
  Copy the self-compiled or precompiled MathLink executable
  \texttt{LoopTools} to
  \texttt{FileNameJoin[\allowbreak{}\{\allowbreak{}\$UserBaseDirectory,\ \allowbreak{}"Applications",\ \allowbreak{}"FeynCalc",\ \allowbreak{}"AddOns",\ \allowbreak{}"FeynHelpers",\ \allowbreak{}"ExternalTools",\ \allowbreak{}"LoopTools"\}]}.
\end{itemize}

\hypertarget{windows-2}{%
\subsubsection{Windows}\label{windows-2}}

\begin{itemize}
\tightlist
\item
  Rename the self-compiled or precompiled MathLink executable
  \texttt{LoopTools.exe} to \texttt{LoopTools} and copy it to
  \texttt{FileNameJoin[\allowbreak{}\{\allowbreak{}\$UserBaseDirectory,\ \allowbreak{}"Applications",\ \allowbreak{}"FeynCalc",\ \allowbreak{}"AddOns",\ \allowbreak{}"FeynHelpers",\ \allowbreak{}"ExternalTools",\ \allowbreak{}"LoopTools"\}]}.
\end{itemize}

\hypertarget{pysecdec}{%
\subsection{pySecDec}\label{pysecdec}}

The installation instructions for pySecDec can be found
\href{https://secdec.readthedocs.io/en/stable/installation.html\#download-the-program-and-install}{here}.
FeynHelpers does not require pySecDec to \emph{generate} the
corresponding Python scripts. However, in order to actually compute the
given loop integrals one has to run those scripts, which is possible
only when pySecDec is installed.

\hypertarget{linux-or-macos-1}{%
\subsubsection{Linux or macOS}\label{linux-or-macos-1}}

It should be possible to install pySecDec via pip automatically.

\hypertarget{windows-3}{%
\subsubsection{Windows}\label{windows-3}}

It could be possible to set up pySecDec on WSL, but currently it is
unclear whether this can work.

\hypertarget{package-x}{%
\subsection{Package-X}\label{package-x}}

Currently the installation of Package-X is handled by the automatic
installer. Since Package-X is a Mathematica package that only needs to
be copied to the correct location, the installation is easy to
automatize.

Alternatively, you can download Package-X from the
\href{https://packagex.hepforge.org/}{developer's website} and copy the
folder \texttt{X} to
\texttt{FileNameJoin[\allowbreak{}\{\allowbreak{}\$UserBaseDirectory,\ \allowbreak{}"Applications"\}]}.
If you can load Package-X by evaluating \texttt{X\textasciigrave } in a
notebook, then the installation was successful.

\hypertarget{qgraf}{%
\subsection{QGRAF}\label{qgraf}}

You can download the source code of QFRAF from the
\href{http://cfif.ist.utl.pt/~paulo/qgraf.html}{developer's website}.

\hypertarget{linux-and-macos}{%
\subsubsection{Linux and macOS}\label{linux-and-macos}}

The compilation instructions can be found in the section ``Compiling''
of the manual \texttt{qgraf-3.6.0.pdf} inside the tarball. Having
compiled the program you need to rename the binary to \texttt{qgraf} and
put it to
\texttt{FileNameJoin[\allowbreak{}\{\allowbreak{}\$UserBaseDirectory,\ \allowbreak{}"Applications",\ \allowbreak{}"FeynCalc",\ \allowbreak{}"AddOns",\ \allowbreak{}"FeynHelpers",\ \allowbreak{}"ExternalTools",\ \allowbreak{}"QGRAF",\ \allowbreak{}"Binary"\}]}.

\hypertarget{windows-4}{%
\subsubsection{Windows}\label{windows-4}}

You can download and install the
\href{https://sourceforge.net/projects/mingw-w64/files/mingw-w64}{MinGW
compiler}. Do not forget to change the architecture from i686 to x86\_64
in the install wizard. Then it should be possible to compile QGRAF with
statically linked libraries via
\texttt{gfortran.exe -static qgraf-3.x.y.f08}. Finally rename the
resulting binary to \texttt{qgraf.exe} and put it to
\texttt{FileNameJoin[\allowbreak{}\{\allowbreak{}\$UserBaseDirectory,\ \allowbreak{}"Applications",\ \allowbreak{}"FeynCalc",\ \allowbreak{}"AddOns",\ \allowbreak{}"FeynHelpers",\ \allowbreak{}"ExternalTools",\ \allowbreak{}"QGRAF",\ \allowbreak{}"Binary"\}]}.
\end{document}
