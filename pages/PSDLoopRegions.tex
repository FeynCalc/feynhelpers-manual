% !TeX program = pdflatex
% !TeX root = PSDLoopRegions.tex

\documentclass[../FeynHelpersManual.tex]{subfiles}
\begin{document}
\hypertarget{psdloopregions}{
\section{PSDLoopRegions}\label{psdloopregions}\index{PSDLoopRegions}}

\texttt{PSDLoopRegions[\allowbreak{}name,\ \allowbreak{}loopIntegral,\ \allowbreak{}order,\ \allowbreak{}smallnessParameter]}
is an auxiliary function that creates input for pySecDec's
\texttt{loop_regions} routine. The results is returned as a string.

\texttt{PSDLoopPackage} is used by \texttt{PSDCreatePythonScripts} when
assembling the generation script for an asymptotic expansion.

\subsection{See also}

\hyperlink{toc}{Overview},
\hyperlink{psdcreatepythonscripts}{PSDCreatePythonScripts},
\hyperlink{psdintegrate}{PSDIntegrate},
\hyperlink{psdloopintegralfrompropagators}{PSDLoopIntegralFromPropagators},
\hyperlink{psdlooppackage}{PSDLoopPackage}.

\subsection{Examples}

\begin{Shaded}
\begin{Highlighting}[]
\NormalTok{PSDLoopRegions}\OperatorTok{[}\StringTok{"loopint"}\OperatorTok{,} \StringTok{"li"}\OperatorTok{,} \DecValTok{2}\OperatorTok{,} \FunctionTok{z}\OperatorTok{]}
\end{Highlighting}
\end{Shaded}

\begin{dmath*}\breakingcomma
\text{loop$\_$regions($\backslash $nname = 'loopint',$\backslash $nloop$\_$integral = li,$\backslash $nsmallness$\_$parameter = 'z',$\backslash $nexpansion$\_$by$\_$regions$\_$order = 2,$\backslash $ndecomposition$\_$method = 'geometric'$\backslash $n)}
\end{dmath*}
\end{document}
