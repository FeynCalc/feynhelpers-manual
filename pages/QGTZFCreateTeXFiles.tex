% !TeX program = pdflatex
% !TeX root = QGTZFCreateTeXFiles.tex

\documentclass[../FeynHelpersManual.tex]{subfiles}
\begin{document}
\hypertarget{qgtzfcreatetexfiles}{
\section{QGTZFCreateTeXFiles}\label{qgtzfcreatetexfiles}\index{QGTZFCreateTeXFiles}}

\texttt{QGTZFCreateTeXFiles[\allowbreak{}input_]} processes the T EX
representation of Feynman diagrams generated by QGRAF in the
TikZ-Feynman format. The input file is the path to the temporary
diagrams file generated by QGCreateAmp.

The function can be also invoked via
\texttt{QGTZFCreateTeXFiles[\allowbreak{}qgOutput]} where
\texttt{qgOutput} is the output of \texttt{QGCreateAmp}.

Notice that to complete the creation of T EX files it is also necessary
to provide a file that contains stylings for the involved fields defined
up via \texttt{tikzset} and \texttt{tikzfeynmanset}. By default the
function assumes that this file is called \texttt{tikz-styles.tex} and
is located in the same directory as the input file. The full path to the
styling file can be provided via the option \texttt{QGDiagramStyle}.

The styling file contains stylings for the involved fields defined up
via \texttt{tikzset} and \texttt{tikzfeynmanset}. This file can be
generated in advance using \texttt{QGTZFCreateFieldStyles} in a
semi-automatic fashion.

The resulting T EX code is saved to the same directory as the input
file. When the option \texttt{Split} is set to \texttt{False} (default),
all diagrams are put into a single tex file called
\texttt{diagrams.tex}. Compiling this file with \texttt{lualatex} can
take some time, which is why this approach is recommended only for a
small \(\mathcal{O}(10)\) number of diagrams. The name of the output
file can be changed using the option \texttt{QGOutputDiagrams}. By
default the alignment is to put 6 diagrams in one row. To change this
number use the option \texttt{Alignment}.

Setting the option \texttt{Split} to \texttt{True} will put each diagram
into a single file. The function will also copy two shell scripts
(specified via the option \texttt{CopyFile}) into the same directory
which will automatize the process of compiling the source files and
gluing them together. This requires the programs \texttt{GNU parallel}
and \texttt{pdfunite}.

\subsection{See also}

\hyperlink{toc}{Overview}, \hyperlink{qgconverttofc}{QGConvertToFC},
\hyperlink{qgcreateamp}{QGCreateAmp},
\hyperlink{qgtzfcreatefieldstyles}{QGTZFCreateFieldStyles}.

\subsection{Examples}
\end{document}
