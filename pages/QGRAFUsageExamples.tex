% !TeX program = pdflatex
% !TeX root = QGRAFUsageExamples.tex

\documentclass[../FeynHelpersManual.tex]{subfiles}
\begin{document}
\hypertarget{qgraf usage examples}{
\section{QGRAF usage examples}\label{qgraf usage examples}\index{QGRAF usage examples}}

\hypertarget{generic-approach}{%
\subsection{Generic approach}\label{generic-approach}}

The main idea behind the FeynHelpers interface to QGRAF is to facilitate
the generation of Feynman diagrams using QGRAF and the subsequent
conversion of the obtained amplitudes into the FeynCalc notation.

The main high-level function of this interface is called
\texttt{QGCreateAmp}. In the simplest case we need to provide following
arguments and options

\begin{itemize}
\tightlist
\item
  the 1st argument is the number of loops, e.g.~\texttt{0}, \texttt{1}
  or \texttt{2}
\item
  the 2nd argument is the process we are considering,
  e.g.~\texttt{\{\allowbreak{}"El[\allowbreak{}p1]",\ \allowbreak{}"Ael[\allowbreak{}p2]"\}->\{\allowbreak{}"El[\allowbreak{}p3]",\ \allowbreak{}"Ael[\allowbreak{}p4]"\}}
  for \(e^- (p_1) e^+ (p_2) \to e^- (p_3) e^+ (p_4)\)
\item
  the option \texttt{QGModel} specifies the QGRAF model used to generate
  the diagrams. FeynHelpers has several simple built-in models such as
  one flavor QED (\texttt{"QEDOneFlavor"}), one flavor QCD
  (\texttt{"QCDOneFlavor"}) etc. To use a custom model this option
  should be given the full path to the corresponding file.
\item
  the option \texttt{QGLoopMomentum} provides the naming scheme for the
  loop momenta, e.g.~\texttt{l} or \texttt{q}
\item
  the option \texttt{QGOptions} is a list of string that will be passed
  to the \texttt{options=} statement in the \texttt{qgraf.dat} file. The
  most useful ones are \texttt{"notadpole"} and \texttt{"onshell"}
\item
  the option \texttt{QGOutputDirectory} specifies the path to the
  directory containing the QGRAF output
\end{itemize}

Here is a simple 1-loop example that incorporates all of the above

\begin{Shaded}
\begin{Highlighting}[]
\NormalTok{qgOutput}\ExtensionTok{=}\NormalTok{QGCreateAmp}\OperatorTok{[}\DecValTok{1}\OperatorTok{,\{}\StringTok{"El[p1]"}\OperatorTok{,}\StringTok{"Ael[p2]"}\OperatorTok{\}}\OtherTok{{-}\textgreater{}}\OperatorTok{\{}\StringTok{"El[p3]"}\OperatorTok{,}\StringTok{"Ael[p4]"}\OperatorTok{\},}\NormalTok{QGModel}\OtherTok{{-}\textgreater{}}\StringTok{"QEDOneFlavor"}\OperatorTok{,}
\NormalTok{QGLoopMomentum}\OtherTok{{-}\textgreater{}}\FunctionTok{l}\OperatorTok{,}\NormalTok{QGOptions}\OtherTok{{-}\textgreater{}}\OperatorTok{\{}\StringTok{"notadpole"}\OperatorTok{,}\StringTok{"onshell"}\OperatorTok{\},}
\NormalTok{QGOutputDirectory}\OtherTok{{-}\textgreater{}}\FunctionTok{FileNameJoin}\OperatorTok{[\{}\NormalTok{$FeynCalcDirectory}\OperatorTok{,}\StringTok{"Database"}\OperatorTok{,}\StringTok{"ElAelToElAelAt1L"}\OperatorTok{\}]]}\NormalTok{;}
\end{Highlighting}
\end{Shaded}

The output is a list containing two elements which are full paths to the
two files \texttt{amplitudes.m} and \texttt{diagrams-raw.tex}. Since
QGRAF has no built-in capabilities for visualizing the generated Feynman
diagrams, we need to use extra tools for this task. The most convenient
way to do this is to employ \texttt{lualatex} together with the
TikZ-Feyman package. By evaluating

\begin{Shaded}
\begin{Highlighting}[]
\NormalTok{tikzStyles}\ExtensionTok{=}\NormalTok{QGTZFCreateFieldStyles}\OperatorTok{[}\StringTok{"QEDOneFlavor"}\OperatorTok{,}\NormalTok{ qgOutput}\OperatorTok{,}
\NormalTok{QGFieldStyles}\OtherTok{{-}\textgreater{}}\OperatorTok{\{\{}\StringTok{"Ga"}\OperatorTok{,}\StringTok{"photon"}\OperatorTok{,}\StringTok{"}\SpecialCharTok{\textbackslash{}\textbackslash{}}\StringTok{gamma"}\OperatorTok{\},}
\OperatorTok{\{}\StringTok{"El"}\OperatorTok{,}\StringTok{"fermion"}\OperatorTok{,}\StringTok{"e\^{}{-}"}\OperatorTok{\},}
\OperatorTok{\{}\StringTok{"Ael"}\OperatorTok{,}\StringTok{"anti fermion"}\OperatorTok{,}\StringTok{"e\^{}+"}\OperatorTok{\}\}]}\NormalTok{;}
\end{Highlighting}
\end{Shaded}

we can create a file containing the styling for the fields present in
our model, so that the diagrams will look nice. Then,

\begin{Shaded}
\begin{Highlighting}[]
\NormalTok{QGTZFCreateTeXFiles}\OperatorTok{[}\NormalTok{qgOutput}\OperatorTok{,}\FunctionTok{Split}\OtherTok{{-}\textgreater{}}\ConstantTok{True}\OperatorTok{]}\NormalTok{;}
\end{Highlighting}
\end{Shaded}

will generate a TeX file for each of the diagrams located in
\texttt{FileNameJoin[\allowbreak{}\{\allowbreak{}\$FeynCalcDirectory,\ \allowbreak{}"Database",\ \allowbreak{}"ElAelToElAelAt1L",\ \allowbreak{}"TeX"\}]]}.
Provided that we have \texttt{GNU parallel} and \texttt{pdfunite}
installed, we can now switch to the terminal, enter the corresponding
directory and generate the diagrams via

\begin{Shaded}
\begin{Highlighting}[]
\NormalTok{.}\SpecialCharTok{/}\NormalTok{makeDiagrams.sh}
\NormalTok{.}\SpecialCharTok{/}\NormalTok{glueDiagrams.sh}
\end{Highlighting}
\end{Shaded}

If everything goes as expected, this will give us a file
\texttt{allDiagrams.pdf} containing all the generated diagrams.

If one wants to visualize the momentum flow through the diagrams, one
can use a special style when calling \texttt{QGCreateAmp}. This is done
by setting the option \texttt{QGDiagramStyle} to
\texttt{tikz-feynman-momentumflow.sty}.

Coming back to the analytic part of the calculation, we need to load the
list of Feynman rules for the vertices and propagators present in the
generated amplitudes. Again, FeynHelpers contains a built-in collection
of Feynman rules that can be loaded using

\begin{Shaded}
\begin{Highlighting}[]
\NormalTok{QGLoadInsertions}\OperatorTok{[}\StringTok{"QGCommonInsertions.m"}\OperatorTok{]}\NormalTok{;}
\end{Highlighting}
\end{Shaded}

If we need to use some new rules for a custom model, then
\texttt{QGLoadInsertions} should be given the full path to the
corresponding insertions file. Finally, with

\begin{Shaded}
\begin{Highlighting}[]
\NormalTok{amps}\ExtensionTok{=}\NormalTok{QGConvertToFC}\OperatorTok{[}\NormalTok{qgOutput}\OperatorTok{,}\NormalTok{DiracChainJoin}\OtherTok{{-}\textgreater{}}\ConstantTok{True}\OperatorTok{]}\NormalTok{;}
\end{Highlighting}
\end{Shaded}

we obtain the list of our amplitudes ready for a subsequent evaluation
within FeynCalc.

\hypertarget{custom-models}{%
\subsection{Custom models}\label{custom-models}}

The following example shows how to generate diagrams for a custom
\(\phi^4\)-model, where we write our own model file and implement the
corresponding Feynman rules.

The process of writing new models is explained in the QGRAF manual. The
only special feature required for a FeynCalc is a custom function in the
propagators called \texttt{mass} that encodes the mass of the particles.
A model for the real scalar field with quartic self-interactions can be
implemented as follows

\begin{Shaded}
\begin{Highlighting}[]
\OperatorTok{[}\NormalTok{ model }\ExtensionTok{=}\NormalTok{ \textquotesingle{}phi}\SpecialCharTok{\^{}}\DecValTok{4}\NormalTok{\textquotesingle{} }\OperatorTok{]}

\SpecialCharTok{\%}\NormalTok{ Propagators:}
\OperatorTok{[}\NormalTok{Phi}\OperatorTok{,}\NormalTok{ Phi}\OperatorTok{,} \SpecialCharTok{+}\NormalTok{; mass}\ExtensionTok{=}\NormalTok{\textquotesingle{}mphi\textquotesingle{}}\OperatorTok{]}

\SpecialCharTok{\%}\NormalTok{ Vertices:}
\OperatorTok{[}\NormalTok{Phi}\OperatorTok{,}\NormalTok{  Phi}\OperatorTok{,}\NormalTok{  Phi}\OperatorTok{,}\NormalTok{ Phi}\OperatorTok{]}
\end{Highlighting}
\end{Shaded}

We need to introduce Feynman rules for the external states, propagators
and vertices. Notice that in the case of vertices all momenta should be
ingoing. The corresponding model file and the collection of insertions
are located in
\texttt{FileNameJoin[\allowbreak{}\{\allowbreak{}\$FeynHelpersDirectory,\ \allowbreak{}"Documentation",\ \allowbreak{}"Examples",\ \allowbreak{}"Phi4\}];}
When using QGRAF via the FeynHelpers interface we need to specify the
full path to those files. For example,

\begin{Shaded}
\begin{Highlighting}[]
\NormalTok{qgModel}\ExtensionTok{=}\FunctionTok{FileNameJoin}\OperatorTok{[\{}\NormalTok{$FeynHelpersDirectory}\OperatorTok{,}\StringTok{"Documentation"}\OperatorTok{,}
\StringTok{"Examples"}\OperatorTok{,}\StringTok{"Phi4"}\OperatorTok{,}\StringTok{"Phi4"}\OperatorTok{\}]}\NormalTok{;}

\NormalTok{qgInsertions}\ExtensionTok{=}\FunctionTok{FileNameJoin}\OperatorTok{[\{}\NormalTok{$FeynHelpersDirectory}\OperatorTok{,}\StringTok{"Documentation"}\OperatorTok{,}
\StringTok{"Examples"}\OperatorTok{,}\StringTok{"Phi4"}\OperatorTok{,}\StringTok{"Insertions{-}Phi4.m"}\OperatorTok{\}]}\NormalTok{;}

\NormalTok{qgOutput}\ExtensionTok{=}\NormalTok{QGCreateAmp}\OperatorTok{[}\DecValTok{1}\OperatorTok{,\{}\StringTok{"Phi[p1]"}\OperatorTok{,}\StringTok{"Phi[p2]"}\OperatorTok{\}}\OtherTok{{-}\textgreater{}}\OperatorTok{\{}\StringTok{"Phi[p3]"}\OperatorTok{,}\StringTok{"Phi[p4]"}\OperatorTok{\},}
\NormalTok{QGModel}\OtherTok{{-}\textgreater{}}\NormalTok{qgModel}\OperatorTok{,}\NormalTok{ QGLoopMomentum}\OtherTok{{-}\textgreater{}}\FunctionTok{l}\OperatorTok{,}\NormalTok{QGOptions}\OtherTok{{-}\textgreater{}}\OperatorTok{\{}\StringTok{"notadpole"}\OperatorTok{,}\StringTok{"onshell"}\OperatorTok{\},}
\NormalTok{QGOutputDirectory}\OtherTok{{-}\textgreater{}}\FunctionTok{FileNameJoin}\OperatorTok{[\{}\NormalTok{$FeynCalcDirectory}\OperatorTok{,}\StringTok{"Database"}\OperatorTok{,}\StringTok{"PhiPhiToPhiPhiAt1L"}\OperatorTok{\}]]}\NormalTok{;}

\NormalTok{tikzStyles}\ExtensionTok{=}\NormalTok{QGTZFCreateFieldStyles}\OperatorTok{[}\NormalTok{qgModel}\OperatorTok{,}\NormalTok{qgOutput}\OperatorTok{,}
\NormalTok{QGFieldStyles}\OtherTok{{-}\textgreater{}}\OperatorTok{\{\{}\StringTok{"Phi"}\OperatorTok{,}\StringTok{"scalar"}\OperatorTok{,}\StringTok{"}\SpecialCharTok{\textbackslash{}\textbackslash{}}\StringTok{phi"}\OperatorTok{\}\}]}\NormalTok{;}

\NormalTok{QGTZFCreateTeXFiles}\OperatorTok{[}\NormalTok{qgOutput}\OperatorTok{,}\FunctionTok{Split}\OtherTok{{-}\textgreater{}}\ConstantTok{True}\OperatorTok{]}\NormalTok{;}

\NormalTok{QGLoadInsertions}\OperatorTok{[}\NormalTok{qgInsertions}\OperatorTok{]}

\NormalTok{amps}\ExtensionTok{=}\NormalTok{QGConvertToFC}\OperatorTok{[}\NormalTok{qgOutput}\OperatorTok{,}\NormalTok{DiracChainJoin}\OtherTok{{-}\textgreater{}}\ConstantTok{True}\OperatorTok{,}\NormalTok{QGInsertionRule}\OtherTok{{-}\textgreater{}}\OperatorTok{\{}\FunctionTok{FileBaseName}\OperatorTok{[}\NormalTok{qgInsertions}\OperatorTok{]\}]}\SpecialCharTok{//}\NormalTok{SMPToSymbol;}

\NormalTok{amps}
\end{Highlighting}
\end{Shaded}

\end{document}
