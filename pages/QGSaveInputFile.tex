% !TeX program = pdflatex
% !TeX root = QGSaveInputFile.tex

\documentclass[../FeynHelpersManual.tex]{subfiles}
\begin{document}
\hypertarget{qgsaveinputfile}{
\section{QGSaveInputFile}\label{qgsaveinputfile}\index{QGSaveInputFile}}

\texttt{QGSaveInputFile} is an option for \texttt{QGCreateAmp}, which
specifies where to save the QGRAF input file \texttt{"qgraf.dat"}. This
file is automatically created from the input parameters of
\texttt{QGCreateAmp} but it must be located in the same directory as the
QGRAF binary when QGRAF is invoked. The default value is False, which
means that \texttt{"qgraf.dat"} will be deleted after the successful
QGRAF run. When set to \texttt{True}, \texttt{"qgraf.dat"} will be
copied to the current directory.

Specifying an explicit path will make \texttt{QGCreateAmp} put
\texttt{"qgraf.dat"} there. Notice that only the file for generating the
amplitudes is saved. The file for generating the diagrams (if exists) is
identical except for the difference in the style line.

\subsection{See also}

\hyperlink{toc}{Overview}, \hyperlink{qgcreateamp}{QGCreateAmp}.

\subsection{Examples}
\end{document}
