% !TeX program = pdflatex
% !TeX root = FerCommand.tex

\documentclass[../FeynHelpersManual.tex]{subfiles}
\begin{document}
\hypertarget{fercommand}{
\section{FerCommand}\label{fercommand}\index{FerCommand}}

\texttt{FerCommand[\allowbreak{}str,\ \allowbreak{}args]} is an
auxiliary function that returns a Fermat command corresponding to
\texttt{str} (possibly using arguments \texttt{args}) as a list of
strings.

At the moment only a very limited set of Fermat instructions is
implemented.

\subsection{See also}

\hyperlink{toc}{Overview}.

\subsection{Examples}

\begin{Shaded}
\begin{Highlighting}[]
\NormalTok{FerCommand}\OperatorTok{[}\StringTok{"Quit"}\OperatorTok{]}
\end{Highlighting}
\end{Shaded}

\begin{dmath*}\breakingcomma
\text{$\&$q;}
\end{dmath*}

\begin{Shaded}
\begin{Highlighting}[]
\NormalTok{FerCommand}\OperatorTok{[}\StringTok{"StopReadingFromTheInputFile"}\OperatorTok{]}
\end{Highlighting}
\end{Shaded}

\begin{dmath*}\breakingcomma
\text{$\&$x;}
\end{dmath*}

\begin{Shaded}
\begin{Highlighting}[]
\NormalTok{FerCommand}\OperatorTok{[}\StringTok{"EnableUglyDisplay"}\OperatorTok{]}
\end{Highlighting}
\end{Shaded}

\begin{dmath*}\breakingcomma
\text{$\&$(U=1);}
\end{dmath*}

\begin{Shaded}
\begin{Highlighting}[]
\NormalTok{FerCommand}\OperatorTok{[}\StringTok{"ReadFromAnInputFile"}\OperatorTok{,} \StringTok{"myFile.txt"}\OperatorTok{]}
\end{Highlighting}
\end{Shaded}

\begin{dmath*}\breakingcomma
\text{$\&$(R='myFile.txt');}
\end{dmath*}

\begin{Shaded}
\begin{Highlighting}[]
\NormalTok{FerCommand}\OperatorTok{[}\StringTok{"SaveToAnOutputFile"}\OperatorTok{,} \StringTok{"myFile.txt"}\OperatorTok{]}
\end{Highlighting}
\end{Shaded}

\begin{dmath*}\breakingcomma
\text{$\&$(S='myFile.txt');}
\end{dmath*}

\begin{Shaded}
\begin{Highlighting}[]
\NormalTok{FerCommand}\OperatorTok{[}\StringTok{"SaveSpecifiedVariablesToAnOutputFile"}\OperatorTok{,} \OperatorTok{\{}\FunctionTok{x}\OperatorTok{,} \FunctionTok{y}\OperatorTok{,} \FunctionTok{z}\OperatorTok{\}]}
\end{Highlighting}
\end{Shaded}

\begin{dmath*}\breakingcomma
\text{!($\&$o, x, y, z);}
\end{dmath*}

\begin{Shaded}
\begin{Highlighting}[]
\NormalTok{FerCommand}\OperatorTok{[}\StringTok{"ReducedRowEchelonForm"}\OperatorTok{,} \StringTok{"mat"}\OperatorTok{]}
\end{Highlighting}
\end{Shaded}

\begin{dmath*}\breakingcomma
\text{Redrowech(mat);}
\end{dmath*}

\begin{Shaded}
\begin{Highlighting}[]
\NormalTok{FerCommand}\OperatorTok{[}\StringTok{"AdjoinPolynomialVariable"}\OperatorTok{,} \FunctionTok{x}\OperatorTok{]}
\end{Highlighting}
\end{Shaded}

\begin{dmath*}\breakingcomma
\text{$\&$(J=x);}
\end{dmath*}
\end{document}
