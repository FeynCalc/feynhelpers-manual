% !TeX program = pdflatex
% !TeX root = KiraGetRS.tex

\documentclass[../FeynHelpersManual.tex]{subfiles}
\begin{document}
\begin{Shaded}
\begin{Highlighting}[]
 
\end{Highlighting}
\end{Shaded}

\hypertarget{kiragetrs}{
\section{KiraGetRS}\label{kiragetrs}\index{KiraGetRS}}

\texttt{KiraGetRS[\allowbreak{}\{\allowbreak{}sec1,\ \allowbreak{}\{\allowbreak{}gli1,\ \allowbreak{}...\}\}]}
returns the number of positive and negative propagator powers (\(r\) and
\(s\) in the Kira/Reduze syntax) for the \texttt{GLI} integrals
belonging to sector \texttt{sec1}. The input can be also a list of such
entries.

If the option \texttt{Max} is set to \texttt{True} (default), only the
largest \texttt{\{\allowbreak{}r,\ \allowbreak{}s\}} combination made of
the largest \texttt{r} and the largest \texttt{s} values in this sector
will be returned. Otherwise, the function will return the
\texttt{\{\allowbreak{}r,\ \allowbreak{}s\}}-pair for each \texttt{GLI}.

\texttt{KiraGetRS} can directly handle the output of
\texttt{FCLoopFindSectors}. Furthermore, the function can also deal with
a simple list of \texttt{GLI}s as in
\texttt{KiraGetRS[\allowbreak{}\{\allowbreak{}gli1,\ \allowbreak{}gli2,\ \allowbreak{}...\}]}.

If the option \texttt{Top} is set to \texttt{True}, and the input is a
list of the form
\texttt{\{\allowbreak{}\{\allowbreak{}sec1,\ \allowbreak{}\{\allowbreak{}gli11,\ \allowbreak{}...\}\},\ \allowbreak{}\{\allowbreak{}sec2,\ \allowbreak{}\{\allowbreak{}gli21,\ \allowbreak{}...\}\},\ \allowbreak{}...\}}
then the output will consists of a list of top sectors and the largest
possible \texttt{\{\allowbreak{}r,\ \allowbreak{}s\}} combination.

\subsection{See also}

\hyperlink{toc}{Overview}, \hyperlink{kiralabelsector}{KiraLabelSector},
\hyperlink{kiracreatejobfile}{KiraCreateJobFile}.

\begin{Shaded}
\begin{Highlighting}[]
\NormalTok{KiraGetRS}\OperatorTok{[\{}\NormalTok{GLI}\OperatorTok{[}\NormalTok{topo1}\OperatorTok{,} \OperatorTok{\{}\DecValTok{1}\OperatorTok{,} \DecValTok{1}\OperatorTok{,} \DecValTok{1}\OperatorTok{,} \DecValTok{1}\OperatorTok{\}],}\NormalTok{ GLI}\OperatorTok{[}\NormalTok{topo1}\OperatorTok{,} \OperatorTok{\{}\DecValTok{2}\OperatorTok{,} \DecValTok{1}\OperatorTok{,} \DecValTok{2}\OperatorTok{,} \DecValTok{1}\OperatorTok{\}],} 
\NormalTok{   GLI}\OperatorTok{[}\NormalTok{topo2}\OperatorTok{,} \OperatorTok{\{}\DecValTok{1}\OperatorTok{,} \DecValTok{0}\OperatorTok{,} \DecValTok{1}\OperatorTok{,} \DecValTok{1}\OperatorTok{\}],}\NormalTok{ GLI}\OperatorTok{[}\NormalTok{topo3}\OperatorTok{,} \OperatorTok{\{}\DecValTok{1}\OperatorTok{,} \DecValTok{0}\OperatorTok{,} \DecValTok{1}\OperatorTok{,} \SpecialCharTok{{-}}\DecValTok{1}\OperatorTok{\}]\}]}
\end{Highlighting}
\end{Shaded}

\begin{dmath*}\breakingcomma
\{6,1\}
\end{dmath*}

\begin{Shaded}
\begin{Highlighting}[]
\NormalTok{KiraGetRS}\OperatorTok{[\{}\NormalTok{GLI}\OperatorTok{[}\NormalTok{topo1}\OperatorTok{,} \OperatorTok{\{}\DecValTok{1}\OperatorTok{,} \DecValTok{1}\OperatorTok{,} \DecValTok{1}\OperatorTok{,} \DecValTok{1}\OperatorTok{\}],}\NormalTok{ GLI}\OperatorTok{[}\NormalTok{topo1}\OperatorTok{,} \OperatorTok{\{}\DecValTok{2}\OperatorTok{,} \DecValTok{1}\OperatorTok{,} \DecValTok{2}\OperatorTok{,} \DecValTok{1}\OperatorTok{\}],} 
\NormalTok{   GLI}\OperatorTok{[}\NormalTok{topo2}\OperatorTok{,} \OperatorTok{\{}\DecValTok{1}\OperatorTok{,} \DecValTok{0}\OperatorTok{,} \DecValTok{1}\OperatorTok{,} \DecValTok{1}\OperatorTok{\}],}\NormalTok{ GLI}\OperatorTok{[}\NormalTok{topo3}\OperatorTok{,} \OperatorTok{\{}\DecValTok{1}\OperatorTok{,} \DecValTok{0}\OperatorTok{,} \DecValTok{1}\OperatorTok{,} \SpecialCharTok{{-}}\DecValTok{1}\OperatorTok{\}]\},} \FunctionTok{Max} \OtherTok{{-}\textgreater{}} \ConstantTok{False}\OperatorTok{]}
\end{Highlighting}
\end{Shaded}

\begin{dmath*}\breakingcomma
\left(
\begin{array}{cc}
 2 & 1 \\
 3 & 0 \\
 4 & 0 \\
 6 & 0 \\
\end{array}
\right)
\end{dmath*}

\begin{Shaded}
\begin{Highlighting}[]
\FunctionTok{out} \ExtensionTok{=}\NormalTok{ FCLoopFindSectors}\OperatorTok{[\{}\NormalTok{GLI}\OperatorTok{[}\NormalTok{topo1}\OperatorTok{,} \OperatorTok{\{}\DecValTok{1}\OperatorTok{,} \DecValTok{1}\OperatorTok{,} \DecValTok{1}\OperatorTok{,} \DecValTok{1}\OperatorTok{\}],}\NormalTok{ GLI}\OperatorTok{[}\NormalTok{topo1}\OperatorTok{,} \OperatorTok{\{}\DecValTok{2}\OperatorTok{,} \DecValTok{1}\OperatorTok{,} \DecValTok{2}\OperatorTok{,} \DecValTok{1}\OperatorTok{\}],} 
\NormalTok{    GLI}\OperatorTok{[}\NormalTok{topo2}\OperatorTok{,} \OperatorTok{\{}\DecValTok{1}\OperatorTok{,} \DecValTok{0}\OperatorTok{,} \DecValTok{1}\OperatorTok{,} \DecValTok{1}\OperatorTok{\}],}\NormalTok{ GLI}\OperatorTok{[}\NormalTok{topo3}\OperatorTok{,} \OperatorTok{\{}\DecValTok{1}\OperatorTok{,} \DecValTok{0}\OperatorTok{,} \DecValTok{1}\OperatorTok{,} \SpecialCharTok{{-}}\DecValTok{1}\OperatorTok{\}]\}]}
\end{Highlighting}
\end{Shaded}

\begin{dmath*}\breakingcomma
\left(
\begin{array}{ccc}
 \left\{\{1,0,1,0\},\left\{G^{\text{topo3}}(1,0,1,-1)\right\}\right\} & \left\{\{1,0,1,1\},\left\{G^{\text{topo2}}(1,0,1,1)\right\}\right\} & \left\{\{1,1,1,1\},\left\{G^{\text{topo1}}(1,1,1,1),G^{\text{topo1}}(2,1,2,1)\right\}\right\} \\
 \{1,0,1,0\} & \{1,0,1,1\} & \{1,1,1,1\} \\
\end{array}
\right)
\end{dmath*}

\begin{Shaded}
\begin{Highlighting}[]
\NormalTok{KiraGetRS}\OperatorTok{[}\FunctionTok{out}\OperatorTok{]}
\end{Highlighting}
\end{Shaded}

\begin{dmath*}\breakingcomma
\left(
\begin{array}{cc}
 \{1,0,1,0\} & \{2,1\} \\
 \{1,0,1,1\} & \{3,0\} \\
 \{1,1,1,1\} & \{6,0\} \\
\end{array}
\right)
\end{dmath*}

\begin{Shaded}
\begin{Highlighting}[]
\NormalTok{KiraGetRS}\OperatorTok{[}\FunctionTok{out}\OperatorTok{,} \FunctionTok{Top} \OtherTok{{-}\textgreater{}} \ConstantTok{True}\OperatorTok{]}
\end{Highlighting}
\end{Shaded}

\begin{dmath*}\breakingcomma
\left\{\left(
\begin{array}{cccc}
 1 & 1 & 1 & 1 \\
\end{array}
\right),\{6,1\}\right\}
\end{dmath*}
\end{document}
