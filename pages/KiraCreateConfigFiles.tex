% !TeX program = pdflatex
% !TeX root = KiraCreateConfigFiles.tex

\documentclass[../FeynHelpersManual.tex]{subfiles}
\begin{document}
\hypertarget{kiracreateconfigfiles}{
\section{KiraCreateConfigFiles}\label{kiracreateconfigfiles}\index{KiraCreateConfigFiles}}

\texttt{KiraCreateConfigFiles[\allowbreak{}topo,\ \allowbreak{}sectors,\ \allowbreak{}path]}
can be used to generate Kira configuration files
(\texttt{integralfamilies.yaml} and \texttt{kinematics.yaml}) from an
FCTopology or list thereof. Here \texttt{sectors} is a list of sectors
need to be reduced,
e.g.~\texttt{\{\allowbreak{}\{\allowbreak{}1,\ \allowbreak{}0,\ \allowbreak{}0,\ \allowbreak{}0\},\ \allowbreak{}\{\allowbreak{}1,\ \allowbreak{}1,\ \allowbreak{}0,\ \allowbreak{}0\},\ \allowbreak{}\{\allowbreak{}1,\ \allowbreak{}1,\ \allowbreak{}1,\ \allowbreak{}0\}\}}
etc.

The functions creates the corresponding yaml files and saves them in
path/topoName/config. Using
\texttt{KiraCreateConfigFiles[\allowbreak{}\{\allowbreak{}topo1,\ \allowbreak{}topo2,\ \allowbreak{}...\},\ \allowbreak{}\{\allowbreak{}sectors1,\ \allowbreak{}sectors2,\ \allowbreak{}...\},\ \allowbreak{} path]}
will save the scripts to \texttt{path/topoName1},
\texttt{path/topoName2} etc. The syntax
\texttt{KiraCreateConfigFiles[\allowbreak{}\{\allowbreak{}topo1,\ \allowbreak{}topo2,\ \allowbreak{}...\},\ \allowbreak{}\{\allowbreak{}sectors1,\ \allowbreak{}sectors2,\ \allowbreak{}...\},\ \allowbreak{} \{\allowbreak{}path1,\ \allowbreak{}path2,\ \allowbreak{}...\}]}
is also possible.

If the directory specified in \texttt{path/topoName} does not exist, it
will be created automatically. If it already exists, its content will be
automatically overwritten, unless the option \texttt{OverwriteTarget} is
set to \texttt{False}.

It is also possible to supply a list of \texttt{GLI}s instead of
sectors. In that case \texttt{FCLoopFindSectors} will be used to
determine the top sector for each topology.

The syntax
\texttt{KiraCreateConfigFiles[\allowbreak{}\{\allowbreak{}topo1,\ \allowbreak{}topo2,\ \allowbreak{}...\},\ \allowbreak{}\{\allowbreak{}sectors1,\ \allowbreak{}sectors2,\ \allowbreak{}...\},\ \allowbreak{}path]}
or
\texttt{KiraCreateConfigFiles[\allowbreak{}\{\allowbreak{}topo1,\ \allowbreak{}topo2,\ \allowbreak{}...\},\ \allowbreak{}\{\allowbreak{}glis1,\ \allowbreak{}glis2,\ \allowbreak{}...\},\ \allowbreak{} path]}
is also allowed. This implies that all config files will go into the
corresponding subdirectories of path,
e.g.~\texttt{path/topoName1/config}, \texttt{path/topoName2/config} etc.

The mass dimension of all kinematic variables should be specified via
the option \texttt{KiraMassDimensions} e.g.~as in
\texttt{\{\allowbreak{}\{\allowbreak{}m1->1\},\ \allowbreak{}\{\allowbreak{}M->1\},\ \allowbreak{}\{\allowbreak{}msq->2\}\}}.

If the amplitude originally contained any external momenta that were
eliminated using momentum conservation, .e.g. by replacing \(k_2\) by
\(P-k_1\) for \(P=k_1+k_2\), those momenta must be nevertheless
specified via the option \texttt{KiraImplicitIncomingMomenta}.

\subsection{See also}

\hyperlink{toc}{Overview},
\hyperlink{kiramassdimensions}{KiraMassDimensions},
\hyperlink{kiraimplicitincomingmomenta}{KiraImplicitIncomingMomenta}.

\subsection{Examples}

\begin{Shaded}
\begin{Highlighting}[]
\NormalTok{topo }\ExtensionTok{=}\NormalTok{ FCTopology}\OperatorTok{[}\StringTok{"asyR1prop2Ltopo01310X11111N1"}\OperatorTok{,} \OperatorTok{\{}\NormalTok{SFAD}\OperatorTok{[\{}\NormalTok{p1}\OperatorTok{,} \DecValTok{0}\OperatorTok{\}],} 
\NormalTok{    SFAD}\OperatorTok{[\{}\NormalTok{p3}\OperatorTok{,}\NormalTok{ mg}\SpecialCharTok{\^{}}\DecValTok{2}\OperatorTok{\}],}\NormalTok{ SFAD}\OperatorTok{[\{\{}\DecValTok{0}\OperatorTok{,} \DecValTok{2}\SpecialCharTok{*}\NormalTok{p3 . }\FunctionTok{q}\OperatorTok{\}\}],} 
\NormalTok{    SFAD}\OperatorTok{[\{}\NormalTok{(p1 }\SpecialCharTok{+} \FunctionTok{q}\NormalTok{)}\OperatorTok{,}\NormalTok{ mb}\SpecialCharTok{\^{}}\DecValTok{2}\OperatorTok{\}],}\NormalTok{ SFAD}\OperatorTok{[\{\{}\DecValTok{0}\OperatorTok{,}\NormalTok{ p1 . p3}\OperatorTok{\}\}]\},} 
   \OperatorTok{\{}\NormalTok{p1}\OperatorTok{,}\NormalTok{ p3}\OperatorTok{\},} \OperatorTok{\{}\FunctionTok{q}\OperatorTok{\},} \OperatorTok{\{}\FunctionTok{Hold}\OperatorTok{[}\NormalTok{SPD}\OperatorTok{][}\FunctionTok{q}\OperatorTok{,} \FunctionTok{q}\OperatorTok{]} \OtherTok{{-}\textgreater{}}\NormalTok{ mb}\SpecialCharTok{\^{}}\DecValTok{2}\OperatorTok{\},} \OperatorTok{\{\}]}
\end{Highlighting}
\end{Shaded}

\begin{dmath*}\breakingcomma
\text{FCTopology}\left(\text{asyR1prop2Ltopo01310X11111N1},\left\{\frac{1}{(\text{p1}^2+i \eta )},\frac{1}{(\text{p3}^2-\text{mg}^2+i \eta )},\frac{1}{(2 (\text{p3}\cdot q)+i \eta )},\frac{1}{((\text{p1}+q)^2-\text{mb}^2+i \eta )},\frac{1}{(\text{p1}\cdot \;\text{p3}+i \eta )}\right\},\{\text{p1},\text{p3}\},\{q\},\left\{\text{Hold}[\text{SPD}][q,q]\to \;\text{mb}^2\right\},\{\}\right)
\end{dmath*}

\begin{Shaded}
\begin{Highlighting}[]
\NormalTok{fileName }\ExtensionTok{=}\NormalTok{ KiraCreateConfigFiles}\OperatorTok{[}\NormalTok{topo}\OperatorTok{,} \OperatorTok{\{\{}\DecValTok{1}\OperatorTok{,} \DecValTok{1}\OperatorTok{,} \DecValTok{1}\OperatorTok{,} \DecValTok{1}\OperatorTok{,} \DecValTok{1}\OperatorTok{\}\},} 
    \FunctionTok{FileNameJoin}\OperatorTok{[\{}\NormalTok{$FeynCalcDirectory}\OperatorTok{,} \StringTok{"Database"}\OperatorTok{\}],}\NormalTok{ KiraMassDimensions }\OtherTok{{-}\textgreater{}} \OperatorTok{\{}\NormalTok{mb }\OtherTok{{-}\textgreater{}} \DecValTok{1}\OperatorTok{,}\NormalTok{ mg }\OtherTok{{-}\textgreater{}} \DecValTok{1}\OperatorTok{\}]}\NormalTok{;}
\end{Highlighting}
\end{Shaded}

\begin{Shaded}
\begin{Highlighting}[]
\NormalTok{fileName}\OperatorTok{[[}\DecValTok{1}\OperatorTok{]]} \SpecialCharTok{//} \FunctionTok{FilePrint}

\CommentTok{(*integralfamilies:}
\CommentTok{  {-} name: asyR1prop2Ltopo01310X11111N1}
\CommentTok{    loop\_momenta: [p1, p3]}
\CommentTok{    top\_level\_sectors: [b11111]}
\CommentTok{    propagators:}
\CommentTok{      {-} [ "p1", 0]}
\CommentTok{      {-} [ "p3", "mg\^{}2"]}
\CommentTok{      {-} \{ bilinear: [ [ "2*p3", "q"], 0] \}}
\CommentTok{      {-} [ "p1 + q", "mb\^{}2"]}
\CommentTok{      {-} \{ bilinear: [ [ "p1", "p3"], 0] \}*)}
\end{Highlighting}
\end{Shaded}

\begin{Shaded}
\begin{Highlighting}[]
\NormalTok{fileName}\OperatorTok{[[}\DecValTok{2}\OperatorTok{]]} \SpecialCharTok{//} \FunctionTok{FilePrint}

\CommentTok{(*kinematics:}
\CommentTok{  incoming\_momenta: [q]}
\CommentTok{  outgoing\_momenta: [q]}
\CommentTok{  momentum\_conservation: []}
\CommentTok{  kinematic\_invariants:}
\CommentTok{    {-} [mb, 1]}
\CommentTok{    {-} [mg, 1]}
\CommentTok{  scalarproduct\_rules:}
\CommentTok{    {-} [ [ q, q],mb\^{}2]*)}
\end{Highlighting}
\end{Shaded}

\begin{Shaded}
\begin{Highlighting}[]
\NormalTok{topos }\ExtensionTok{=} \OperatorTok{\{}
\NormalTok{   FCTopology}\OperatorTok{[}\StringTok{"asyR3prop2Ltopo01310X11111N1"}\OperatorTok{,} \OperatorTok{\{}\NormalTok{SFAD}\OperatorTok{[\{\{}\SpecialCharTok{{-}}\NormalTok{p1}\OperatorTok{,} \DecValTok{0}\OperatorTok{\},} \OperatorTok{\{}\DecValTok{0}\OperatorTok{,} \DecValTok{1}\OperatorTok{\},} \DecValTok{1}\OperatorTok{\}],}\NormalTok{ SFAD}\OperatorTok{[\{\{}\SpecialCharTok{{-}}\NormalTok{p3}\OperatorTok{,} \DecValTok{0}\OperatorTok{\},} \OperatorTok{\{}\NormalTok{mg}\SpecialCharTok{\^{}}\DecValTok{2}\OperatorTok{,} \DecValTok{1}\OperatorTok{\},} \DecValTok{1}\OperatorTok{\}],}\NormalTok{ SFAD}\OperatorTok{[\{\{}\DecValTok{0}\OperatorTok{,} \DecValTok{2}\SpecialCharTok{*}\NormalTok{p3 . }\FunctionTok{q}\OperatorTok{\},} \OperatorTok{\{}\DecValTok{0}\OperatorTok{,} \DecValTok{1}\OperatorTok{\},} \DecValTok{1}\OperatorTok{\}],} 
\NormalTok{     SFAD}\OperatorTok{[\{\{}\DecValTok{0}\OperatorTok{,} \DecValTok{2}\SpecialCharTok{*}\NormalTok{p1 . }\FunctionTok{q}\OperatorTok{\},} \OperatorTok{\{}\DecValTok{0}\OperatorTok{,} \DecValTok{1}\OperatorTok{\},} \DecValTok{1}\OperatorTok{\}],}\NormalTok{ SFAD}\OperatorTok{[\{\{}\SpecialCharTok{{-}}\NormalTok{p1 }\SpecialCharTok{+}\NormalTok{ p3}\OperatorTok{,} \DecValTok{0}\OperatorTok{\},} \OperatorTok{\{}\DecValTok{0}\OperatorTok{,} \DecValTok{1}\OperatorTok{\},} \DecValTok{1}\OperatorTok{\}]\},} \OperatorTok{\{}\NormalTok{p1}\OperatorTok{,}\NormalTok{ p3}\OperatorTok{\},} \OperatorTok{\{}\FunctionTok{q}\OperatorTok{\},} \OperatorTok{\{}\FunctionTok{Hold}\OperatorTok{[}\NormalTok{SPD}\OperatorTok{][}\FunctionTok{q}\OperatorTok{,} \FunctionTok{q}\OperatorTok{]} \OtherTok{{-}\textgreater{}}\NormalTok{ mb}\SpecialCharTok{\^{}}\DecValTok{2}\OperatorTok{\},} \OperatorTok{\{\}],} 
\NormalTok{   FCTopology}\OperatorTok{[}\StringTok{"asyR1prop2Ltopo01310X11111N1"}\OperatorTok{,} \OperatorTok{\{}\NormalTok{SFAD}\OperatorTok{[\{\{}\SpecialCharTok{{-}}\NormalTok{p1}\OperatorTok{,} \DecValTok{0}\OperatorTok{\},} \OperatorTok{\{}\DecValTok{0}\OperatorTok{,} \DecValTok{1}\OperatorTok{\},} \DecValTok{1}\OperatorTok{\}],}\NormalTok{ SFAD}\OperatorTok{[\{\{}\SpecialCharTok{{-}}\NormalTok{p3}\OperatorTok{,} \DecValTok{0}\OperatorTok{\},} \OperatorTok{\{}\NormalTok{mg}\SpecialCharTok{\^{}}\DecValTok{2}\OperatorTok{,} \DecValTok{1}\OperatorTok{\},} \DecValTok{1}\OperatorTok{\}],}\NormalTok{ SFAD}\OperatorTok{[\{\{}\DecValTok{0}\OperatorTok{,} \DecValTok{2}\SpecialCharTok{*}\NormalTok{p3 . }\FunctionTok{q}\OperatorTok{\},} \OperatorTok{\{}\DecValTok{0}\OperatorTok{,} \DecValTok{1}\OperatorTok{\},} \DecValTok{1}\OperatorTok{\}],} 
\NormalTok{      SFAD}\OperatorTok{[\{\{}\SpecialCharTok{{-}}\NormalTok{p1 }\SpecialCharTok{{-}} \FunctionTok{q}\OperatorTok{,} \DecValTok{0}\OperatorTok{\},} \OperatorTok{\{}\NormalTok{mb}\SpecialCharTok{\^{}}\DecValTok{2}\OperatorTok{,} \DecValTok{1}\OperatorTok{\},} \DecValTok{1}\OperatorTok{\}],}\NormalTok{ SFAD}\OperatorTok{[\{\{}\DecValTok{0}\OperatorTok{,} \SpecialCharTok{{-}}\NormalTok{p1 . p3}\OperatorTok{\},} \OperatorTok{\{}\DecValTok{0}\OperatorTok{,} \DecValTok{1}\OperatorTok{\},} \DecValTok{1}\OperatorTok{\}]\},} \OperatorTok{\{}\NormalTok{p1}\OperatorTok{,}\NormalTok{ p3}\OperatorTok{\},} \OperatorTok{\{}\FunctionTok{q}\OperatorTok{\},} \OperatorTok{\{}\FunctionTok{Hold}\OperatorTok{[}\NormalTok{SPD}\OperatorTok{][}\FunctionTok{q}\OperatorTok{,} \FunctionTok{q}\OperatorTok{]} \OtherTok{{-}\textgreater{}}\NormalTok{ mb}\SpecialCharTok{\^{}}\DecValTok{2}\OperatorTok{\},} \OperatorTok{\{\}]} 
   \OperatorTok{\}}
\end{Highlighting}
\end{Shaded}

\begin{dmath*}\breakingcomma
\left\{\text{FCTopology}\left(\text{asyR3prop2Ltopo01310X11111N1},\left\{\frac{1}{(\text{p1}^2+i \eta )},\frac{1}{(\text{p3}^2-\text{mg}^2+i \eta )},\frac{1}{(2 (\text{p3}\cdot q)+i \eta )},\frac{1}{(2 (\text{p1}\cdot q)+i \eta )},\frac{1}{((\text{p3}-\text{p1})^2+i \eta )}\right\},\{\text{p1},\text{p3}\},\{q\},\left\{\text{Hold}[\text{SPD}][q,q]\to \;\text{mb}^2\right\},\{\}\right),\text{FCTopology}\left(\text{asyR1prop2Ltopo01310X11111N1},\left\{\frac{1}{(\text{p1}^2+i \eta )},\frac{1}{(\text{p3}^2-\text{mg}^2+i \eta )},\frac{1}{(2 (\text{p3}\cdot q)+i \eta )},\frac{1}{((-\text{p1}-q)^2-\text{mb}^2+i \eta )},\frac{1}{(-\text{p1}\cdot \;\text{p3}+i \eta )}\right\},\{\text{p1},\text{p3}\},\{q\},\left\{\text{Hold}[\text{SPD}][q,q]\to \;\text{mb}^2\right\},\{\}\right)\right\}
\end{dmath*}

\begin{Shaded}
\begin{Highlighting}[]
\OperatorTok{\{}\NormalTok{GLI}\OperatorTok{[}\StringTok{"asyR3prop2Ltopo01310X11111N1"}\OperatorTok{,} \OperatorTok{\{}\DecValTok{1}\OperatorTok{,} \DecValTok{1}\OperatorTok{,} \DecValTok{1}\OperatorTok{,} \DecValTok{1}\OperatorTok{,} \DecValTok{2}\OperatorTok{\}],}\NormalTok{ GLI}\OperatorTok{[}\StringTok{"asyR1prop2Ltopo01310X11111N1"}\OperatorTok{,} \OperatorTok{\{}\DecValTok{1}\OperatorTok{,} \DecValTok{1}\OperatorTok{,} \DecValTok{1}\OperatorTok{,} \DecValTok{1}\OperatorTok{,} \DecValTok{2}\OperatorTok{\}]\}}
\end{Highlighting}
\end{Shaded}

\begin{dmath*}\breakingcomma
\left\{G^{\text{asyR3prop2Ltopo01310X11111N1}}(1,1,1,1,2),G^{\text{asyR1prop2Ltopo01310X11111N1}}(1,1,1,1,2)\right\}
\end{dmath*}

\begin{Shaded}
\begin{Highlighting}[]
\FunctionTok{fileNames} \ExtensionTok{=}\NormalTok{ KiraCreateConfigFiles}\OperatorTok{[}\NormalTok{topos}\OperatorTok{,} \OperatorTok{\{}\NormalTok{GLI}\OperatorTok{[}\StringTok{"asyR3prop2Ltopo01310X11111N1"}\OperatorTok{,} \OperatorTok{\{}\DecValTok{1}\OperatorTok{,} \DecValTok{1}\OperatorTok{,} \DecValTok{1}\OperatorTok{,} \DecValTok{1}\OperatorTok{,} \DecValTok{2}\OperatorTok{\}],}\NormalTok{ GLI}\OperatorTok{[}\StringTok{"asyR1prop2Ltopo01310X11111N1"}\OperatorTok{,} \OperatorTok{\{}\DecValTok{1}\OperatorTok{,} \DecValTok{1}\OperatorTok{,} \DecValTok{1}\OperatorTok{,} \DecValTok{1}\OperatorTok{,} \DecValTok{2}\OperatorTok{\}]\},} 
   \FunctionTok{FileNameJoin}\OperatorTok{[\{}\NormalTok{$FeynCalcDirectory}\OperatorTok{,} \StringTok{"Database"}\OperatorTok{\}],}\NormalTok{ KiraMassDimensions }\OtherTok{{-}\textgreater{}} \OperatorTok{\{}\NormalTok{mb }\OtherTok{{-}\textgreater{}} \DecValTok{1}\OperatorTok{,}\NormalTok{ mg }\OtherTok{{-}\textgreater{}} \DecValTok{1}\OperatorTok{\}]}
\end{Highlighting}
\end{Shaded}

\begin{dmath*}\breakingcomma
\left(
\begin{array}{cc}
 \;\text{/home/vs/.Mathematica/Applications/FeynCalc/Database/asyR3prop2Ltopo01310X11111N1/config/integralfamilies.yaml} & \;\text{/home/vs/.Mathematica/Applications/FeynCalc/Database/asyR3prop2Ltopo01310X11111N1/config/kinematics.yaml} \\
 \;\text{/home/vs/.Mathematica/Applications/FeynCalc/Database/asyR1prop2Ltopo01310X11111N1/config/integralfamilies.yaml} & \;\text{/home/vs/.Mathematica/Applications/FeynCalc/Database/asyR1prop2Ltopo01310X11111N1/config/kinematics.yaml} \\
\end{array}
\right)
\end{dmath*}

\begin{Shaded}
\begin{Highlighting}[]
\FunctionTok{FilePrint}\OperatorTok{[}\FunctionTok{fileNames}\OperatorTok{[[}\DecValTok{1}\OperatorTok{]][[}\DecValTok{1}\OperatorTok{]]]}

\CommentTok{(*integralfamilies:}
\CommentTok{  {-} name: asyR3prop2Ltopo01310X11111N1}
\CommentTok{    loop\_momenta: [p1, p3]}
\CommentTok{    top\_level\_sectors: [b11111]}
\CommentTok{    propagators:}
\CommentTok{      {-} [ "p1", 0]}
\CommentTok{      {-} [ "p3", "mg\^{}2"]}
\CommentTok{      {-} \{ bilinear: [ [ "2*p3", "q"], 0] \}}
\CommentTok{      {-} \{ bilinear: [ [ "2*p1", "q"], 0] \}}
\CommentTok{      {-} [ "{-}p1 + p3", 0]*)}
\end{Highlighting}
\end{Shaded}

\begin{Shaded}
\begin{Highlighting}[]
\FunctionTok{FilePrint}\OperatorTok{[}\FunctionTok{fileNames}\OperatorTok{[[}\DecValTok{1}\OperatorTok{]][[}\DecValTok{2}\OperatorTok{]]]}

\CommentTok{(*kinematics:}
\CommentTok{  incoming\_momenta: [q]}
\CommentTok{  outgoing\_momenta: [q]}
\CommentTok{  momentum\_conservation: []}
\CommentTok{  kinematic\_invariants:}
\CommentTok{    {-} [mb, 1]}
\CommentTok{    {-} [mg, 1]}
\CommentTok{  scalarproduct\_rules:}
\CommentTok{    {-} [ [ q, q],mb\^{}2]*)}
\end{Highlighting}
\end{Shaded}

\begin{Shaded}
\begin{Highlighting}[]
\FunctionTok{FilePrint}\OperatorTok{[}\FunctionTok{fileNames}\OperatorTok{[[}\DecValTok{2}\OperatorTok{]][[}\DecValTok{1}\OperatorTok{]]]}

\CommentTok{(*integralfamilies:}
\CommentTok{  {-} name: asyR1prop2Ltopo01310X11111N1}
\CommentTok{    loop\_momenta: [p1, p3]}
\CommentTok{    top\_level\_sectors: [b11111]}
\CommentTok{    propagators:}
\CommentTok{      {-} [ "p1", 0]}
\CommentTok{      {-} [ "p3", "mg\^{}2"]}
\CommentTok{      {-} \{ bilinear: [ [ "2*p3", "q"], 0] \}}
\CommentTok{      {-} [ "{-}p1 {-} q", "mb\^{}2"]}
\CommentTok{      {-} \{ bilinear: [ [ "{-}p1", "p3"], 0] \}*)}
\end{Highlighting}
\end{Shaded}

\begin{Shaded}
\begin{Highlighting}[]
\FunctionTok{FilePrint}\OperatorTok{[}\FunctionTok{fileNames}\OperatorTok{[[}\DecValTok{2}\OperatorTok{]][[}\DecValTok{2}\OperatorTok{]]]}

\CommentTok{(*kinematics:}
\CommentTok{  incoming\_momenta: [q]}
\CommentTok{  outgoing\_momenta: [q]}
\CommentTok{  momentum\_conservation: []}
\CommentTok{  kinematic\_invariants:}
\CommentTok{    {-} [mb, 1]}
\CommentTok{    {-} [mg, 1]}
\CommentTok{  scalarproduct\_rules:}
\CommentTok{    {-} [ [ q, q],mb\^{}2]*)}
\end{Highlighting}
\end{Shaded}

\end{document}
