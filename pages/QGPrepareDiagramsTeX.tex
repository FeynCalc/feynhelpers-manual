% !TeX program = pdflatex
% !TeX root = QGPrepareDiagramsTeX.tex

\documentclass[../FeynHelpersManual.tex]{subfiles}
\begin{document}
\hypertarget{qgpreparediagramstex}{
\section{QGPrepareDiagramsTeX}\label{qgpreparediagramstex}\index{QGPrepareDiagramsTeX}}

\texttt{QGPrepareDiagramsTeX[\allowbreak{}input_,\ \allowbreak{}output_]}
processes the LaTeX representation of Feynman diagrams generated by
QGRAF using a supported style file. The input file must contain valid
LaTeX code. Following styles (to be set via the option \texttt{Style})
are supported:

\begin{itemize}
\tightlist
\item
  \texttt{"TikZ-Feynman"} - uses tikz-feynmann packages to visualize
  Feynman diagrams created with tikz-feynman.sty
\end{itemize}

The beginning and the end of each .tex file will be pasted from the
files specified by the options \texttt{QGTeXProlog} and
\texttt{QGTeXEpilog} respectively. The resulting .tex file will be saved
to \texttt{output}. If the option \texttt{Split} is set to
\texttt{True}, each diagram will be saved to a separate .tex file in the
directory output.

\subsection{See also}

\hyperlink{toc}{Overview}, \hyperlink{qgconverttofc}{QGConvertToFC},
\hyperlink{qgcreateamp}{QGCreateAmp}.

\subsection{Examples}
\end{document}
