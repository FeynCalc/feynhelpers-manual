% !TeX program = pdflatex
% !TeX root = FiestaUsageExamples.tex

\documentclass[../FeynHelpersManual.tex]{subfiles}
\begin{document}
\hypertarget{fiesta usage examples}{
\section{FIESTA usage examples}\label{fiesta usage examples}\index{FIESTA usage examples}}

The main idea behind the FeynHelpers interface to FIESTA is to
facilitate the generation of FIESTA scripts for integrals written in the
FeynCalc notation (i.e.~as \texttt{GLI}s with the corresponding lists of
\texttt{FCTopology} symbols). If needed, those scripts can be also
evaluated in background.

The main high-level function of this interface is called
\texttt{FSACreateMathematicaScripts}. In the simplest case we need two
provide following arguments and options

\begin{itemize}
\tightlist
\item
  the 1st argument is some \texttt{GLI}
\item
  the 2nd argument is the \texttt{FCTopology} to which this \texttt{GLI}
  belongs
\item
  the 3rd argument is where to put the directory with FIESTA scripts.
  For quick tests one can simply use
  \texttt{NotebookDirectory[\allowbreak{}]}
\item
  the option \texttt{FSAOrderInEps} specifies the order in
  \(\varepsilon\) to which the integral should be evaluated (default is
  \texttt{0})
\item
  the option \texttt{FSAParameterRules} is a list of rules for replacing
  kinematic invariants with numerical values which can be real or
  complex numbers.
\item
  if the script directory already exists, the function will by default
  refuse to overwrite it. Setting the option \texttt{OverwriteTarget} to
  \texttt{True} you can tell the code that you do not care about that
\end{itemize}

Here is a simple 1-loop example that incorporates all of the above.
Notice that it is crucial to switch the \(i \eta\) sign of propagators
from plus to minus, otherwise the result will be incorrect.

\begin{Shaded}
\begin{Highlighting}[]
\NormalTok{int }\ExtensionTok{=}\NormalTok{ GLI}\OperatorTok{[}\NormalTok{prop1L}\OperatorTok{,} \OperatorTok{\{}\DecValTok{1}\OperatorTok{,} \DecValTok{1}\OperatorTok{\}]}
\NormalTok{topo }\ExtensionTok{=}\NormalTok{ FCLoopSwitchEtaSign}\OperatorTok{[}\NormalTok{FCTopology}\OperatorTok{[}\NormalTok{prop1L}\OperatorTok{,} \OperatorTok{\{}\NormalTok{FAD}\OperatorTok{[\{}\NormalTok{p1}\OperatorTok{,}\NormalTok{ m1}\OperatorTok{\}],}\NormalTok{ FAD}\OperatorTok{[\{}\NormalTok{p1 }\SpecialCharTok{+} \FunctionTok{q}\OperatorTok{,}\NormalTok{ m2}\OperatorTok{\}]\},} \OperatorTok{\{}\NormalTok{p1}\OperatorTok{\},} \OperatorTok{\{}\FunctionTok{q}\OperatorTok{\},} \OperatorTok{\{\},} \OperatorTok{\{\}],} \SpecialCharTok{{-}}\DecValTok{1}\OperatorTok{]}
\NormalTok{files }\ExtensionTok{=}\NormalTok{ FSACreateMathematicaScripts}\OperatorTok{[}\NormalTok{int}\OperatorTok{,}\NormalTok{ topo}\OperatorTok{,} \FunctionTok{NotebookDirectory}\OperatorTok{[],}\NormalTok{ FinalSubstitutions }\OtherTok{{-}\textgreater{}} \OperatorTok{\{}\FunctionTok{Hold}\OperatorTok{[}\NormalTok{SPD}\OperatorTok{][}\FunctionTok{q}\OperatorTok{]} \OtherTok{{-}\textgreater{}}\NormalTok{ qq}\OperatorTok{\}} \OperatorTok{,} 
\NormalTok{  FSAParameterRules }\OtherTok{{-}\textgreater{}} \OperatorTok{\{}\NormalTok{qq }\OtherTok{{-}\textgreater{}} \FloatTok{30.}\OperatorTok{,}\NormalTok{ m1 }\OtherTok{{-}\textgreater{}} \FloatTok{2.}\OperatorTok{,}\NormalTok{ m2 }\OtherTok{{-}\textgreater{}} \FloatTok{3.}\OperatorTok{\},}\NormalTok{ OverwriteTarget }\OtherTok{{-}\textgreater{}} \ConstantTok{True}\OperatorTok{]}
\end{Highlighting}
\end{Shaded}

The output is a list containing two elements. The first one is the full
path to the Mathematica script file \texttt{FiestaScript.m}, while the
second give the name of the output file containing numerical result for
the given integral. For simple integrals you can evaluate the script
directly in your Mathematica session by running

\begin{Shaded}
\begin{Highlighting}[]
\NormalTok{FSARunIntegration}\OperatorTok{[}\NormalTok{files}\OperatorTok{[[}\DecValTok{1}\OperatorTok{]]]}
\end{Highlighting}
\end{Shaded}

Notice that the evaluation of sufficiently complicated integrals can
take hours or even days so in general it is not recommended to use
\texttt{FSARunIntegration}.

Here is a sample the script file

\begin{Shaded}
\begin{Highlighting}[]
\FunctionTok{Get}\OperatorTok{[}\StringTok{"/home/vs/.Mathematica/Applications/FIESTA5/FIESTA5.m"}\OperatorTok{]}\NormalTok{;}


\FunctionTok{If}\OperatorTok{[}\VariableTok{$FrontEnd}\ExtensionTok{===}\ConstantTok{Null}\OperatorTok{,}
\NormalTok{  projectDirectory}\ExtensionTok{=}\FunctionTok{DirectoryName}\OperatorTok{[}\VariableTok{$InputFileName}\OperatorTok{],}
\NormalTok{  projectDirectory}\ExtensionTok{=}\FunctionTok{NotebookDirectory}\OperatorTok{[]}
\OperatorTok{]}\NormalTok{;}
\FunctionTok{SetDirectory}\OperatorTok{[}\NormalTok{projectDirectory}\OperatorTok{]}\NormalTok{;}
\NormalTok{resFileName }\ExtensionTok{=} \StringTok{"numres\_"}\NormalTok{ \textless{}\textgreater{} StringRiffle}\OperatorTok{[}\FunctionTok{ToString}\OperatorTok{[}\NormalTok{\#}\OperatorTok{,} \FunctionTok{InputForm}\OperatorTok{]}\NormalTok{ \& }\SpecialCharTok{/}\NormalTok{@ }\OperatorTok{\{}\FloatTok{15.}\OperatorTok{,} \FloatTok{2.}\OperatorTok{,} \FloatTok{3.}\OperatorTok{\},} \StringTok{"\_"}\OperatorTok{]}\NormalTok{\textless{}\textgreater{}}\StringTok{"\_fiesta.m"}\NormalTok{;}
\FunctionTok{Print}\OperatorTok{[}\StringTok{"Working directory: "}\OperatorTok{,}\NormalTok{ projectDirectory}\OperatorTok{]}\NormalTok{;}
\FunctionTok{Print}\OperatorTok{[}\StringTok{"The results will be saved to: "}\OperatorTok{,}\NormalTok{ resFileName}\OperatorTok{]}\NormalTok{;}


\NormalTok{uf }\ExtensionTok{=}\NormalTok{ UF}\OperatorTok{[\{}\NormalTok{p1}\OperatorTok{\},\{}\NormalTok{m1}\SpecialCharTok{\^{}}\DecValTok{2} \SpecialCharTok{{-}}\NormalTok{ p1}\SpecialCharTok{\^{}}\DecValTok{2}\OperatorTok{,}\NormalTok{ m2}\SpecialCharTok{\^{}}\DecValTok{2} \SpecialCharTok{{-}}\NormalTok{ (p1 }\SpecialCharTok{+} \FunctionTok{q}\NormalTok{)}\SpecialCharTok{\^{}}\DecValTok{2}\OperatorTok{\},} \OperatorTok{\{}\FunctionTok{q}\SpecialCharTok{\^{}}\DecValTok{2} \OtherTok{{-}\textgreater{}}\NormalTok{ qq}\OperatorTok{,}\NormalTok{ qq }\OtherTok{{-}\textgreater{}} \FloatTok{15.}\OperatorTok{,}\NormalTok{ m1 }\OtherTok{{-}\textgreater{}} \FloatTok{2.}\OperatorTok{,}\NormalTok{ m2 }\OtherTok{{-}\textgreater{}} \FloatTok{3.}\OperatorTok{\}]}\NormalTok{;}
\FunctionTok{SetOptions}\OperatorTok{[}\NormalTok{FIESTA}\OperatorTok{,} \StringTok{"NumberOfSubkernels"} \OtherTok{{-}\textgreater{}} \DecValTok{4}\OperatorTok{,}\StringTok{"ComplexMode"} \OtherTok{{-}\textgreater{}} \ConstantTok{True}\OperatorTok{,}\StringTok{"ReturnErrorWithBrackets"} \OtherTok{{-}\textgreater{}} \ConstantTok{True}\OperatorTok{,}
\StringTok{"Integrator"} \OtherTok{{-}\textgreater{}} \StringTok{"quasiMonteCarlo"}\OperatorTok{,}\StringTok{"IntegratorOptions"} \OtherTok{{-}\textgreater{}} \OperatorTok{\{\{}\StringTok{"maxeval"}\OperatorTok{,} \StringTok{"50000"}\OperatorTok{\},} 
\OperatorTok{\{}\StringTok{"epsrel"}\OperatorTok{,} \StringTok{"1.000000E{-}05"}\OperatorTok{\},} \OperatorTok{\{}\StringTok{"epsabs"}\OperatorTok{,} \StringTok{"1.000000E{-}12"}\OperatorTok{\},} \OperatorTok{\{}\StringTok{"integralTransform"}\OperatorTok{,} \StringTok{"korobov"}\OperatorTok{\}\}]}\NormalTok{;}
\NormalTok{pref }\ExtensionTok{=} \DecValTok{1}\NormalTok{;}
\NormalTok{resRaw }\ExtensionTok{=}\NormalTok{ SDEvaluate}\OperatorTok{[}\NormalTok{uf}\OperatorTok{,\{}\DecValTok{1}\OperatorTok{,} \DecValTok{1}\OperatorTok{\},}\DecValTok{0}\OperatorTok{]}\NormalTok{;}
\NormalTok{res }\ExtensionTok{=}\NormalTok{ resRaw}\SpecialCharTok{*}\NormalTok{pref;}
\FunctionTok{Print}\OperatorTok{[}\StringTok{"Final result: "}\OperatorTok{,}\NormalTok{ res}\OperatorTok{]}\NormalTok{;}
\FunctionTok{Put}\OperatorTok{[}\NormalTok{res}\OperatorTok{,}\NormalTok{ resFileName}\OperatorTok{]}\NormalTok{;}
\end{Highlighting}
\end{Shaded}

To load the numerical results into your Mathematica session you can use
the function \texttt{FSALoadNumericalResults}. To that aim you just need
to give it \texttt{files} as input.

\begin{Shaded}
\begin{Highlighting}[]
\NormalTok{FSALoadNumericalResults}\OperatorTok{[}\NormalTok{files}\OperatorTok{]}
\end{Highlighting}
\end{Shaded}

If you want to perform an asymptotic expansion, you need to set the
option \texttt{FSASDExpandAsy} to \texttt{True}, specify the expansion
variable using \texttt{FSAExpandVar} and set the desired expansion order
via \texttt{FSASDExpandAsyOrder}. For example,

\begin{Shaded}
\begin{Highlighting}[]
\NormalTok{int }\ExtensionTok{=}\NormalTok{ GLI}\OperatorTok{[}\NormalTok{prop1L}\OperatorTok{,} \OperatorTok{\{}\DecValTok{1}\OperatorTok{,} \DecValTok{1}\OperatorTok{\}]}
\NormalTok{topo }\ExtensionTok{=}\NormalTok{ FCLoopSwitchEtaSign}\OperatorTok{[}\NormalTok{FCTopology}\OperatorTok{[}\NormalTok{prop1L}\OperatorTok{,} \OperatorTok{\{}\NormalTok{FAD}\OperatorTok{[\{}\NormalTok{p1}\OperatorTok{,}\NormalTok{ m1}\OperatorTok{\}],}\NormalTok{ FAD}\OperatorTok{[\{}\NormalTok{p1 }\SpecialCharTok{+} \FunctionTok{q}\OperatorTok{,}\NormalTok{ m2}\OperatorTok{\}]\},} \OperatorTok{\{}\NormalTok{p1}\OperatorTok{\},} \OperatorTok{\{}\FunctionTok{q}\OperatorTok{\},} \OperatorTok{\{\},} \OperatorTok{\{\}],} \SpecialCharTok{{-}}\DecValTok{1}\OperatorTok{]}
\NormalTok{files }\ExtensionTok{=}\NormalTok{ FSACreateMathematicaScripts}\OperatorTok{[}\NormalTok{int}\OperatorTok{,}\NormalTok{ topo}\OperatorTok{,} \FunctionTok{NotebookDirectory}\OperatorTok{[],}\NormalTok{ FinalSubstitutions }\OtherTok{{-}\textgreater{}} \OperatorTok{\{}\FunctionTok{Hold}\OperatorTok{[}\NormalTok{SPD}\OperatorTok{][}\FunctionTok{q}\OperatorTok{]} \OtherTok{{-}\textgreater{}}\NormalTok{ qq}\OperatorTok{\}} \OperatorTok{,} 
\NormalTok{  FSAParameterRules }\OtherTok{{-}\textgreater{}} \OperatorTok{\{}\NormalTok{qq }\OtherTok{{-}\textgreater{}} \FloatTok{30.}\OperatorTok{,}\NormalTok{ m1 }\OtherTok{{-}\textgreater{}} \FloatTok{2.}\OperatorTok{\},}\NormalTok{ OverwriteTarget }\OtherTok{{-}\textgreater{}} \ConstantTok{True}\OperatorTok{,}\NormalTok{ FSASDExpandAsy}\OtherTok{{-}\textgreater{}}\ConstantTok{True}\OperatorTok{,}\NormalTok{ FSAExpandVar }\OtherTok{{-}\textgreater{}}\NormalTok{ m2}\OperatorTok{,}\NormalTok{ FSASDExpandAsyOrder}\OtherTok{{-}\textgreater{}} \DecValTok{4}\OperatorTok{]}
\end{Highlighting}
\end{Shaded}

\end{document}
