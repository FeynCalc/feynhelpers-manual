% !TeX program = pdflatex
% !TeX root = FIREUsageExamples.tex

\documentclass[../FeynHelpersManual.tex]{subfiles}
\begin{document}
\hypertarget{fire usage examples}{
\section{FIRE usage examples}\label{fire usage examples}\index{FIRE usage examples}}

The main idea behind the FeynHelpers interface to FIRE is to facilitate
the generation of FIRE scripts for reducing the given sets of loop
integrals written in the FeynCalc notation (i.e.~as \texttt{GLI}s with
the corresponding lists of \texttt{FCTopology} symbols) to masters.

If needed, the reduction can be done by running the corresponding
scripts from a Mathematica notebook, although we do not recommend using
this approach for complicated calculations.

Let us assume that upon simplifying some amplitude we have a list of
topologies and the corresponding loop integrals that need to be reduced.

\begin{Shaded}
\begin{Highlighting}[]
\NormalTok{topos }\ExtensionTok{=} \OperatorTok{\{}\NormalTok{FCTopology}\OperatorTok{[}\StringTok{"fctopology1"}\OperatorTok{,} \OperatorTok{\{}\NormalTok{SFAD}\OperatorTok{[\{\{}\NormalTok{q2}\OperatorTok{,} \DecValTok{0}\OperatorTok{\},} \OperatorTok{\{}\DecValTok{0}\OperatorTok{,} \DecValTok{1}\OperatorTok{\},} \DecValTok{1}\OperatorTok{\}],}\NormalTok{ SFAD}\OperatorTok{[\{\{}\NormalTok{q1}\OperatorTok{,} \DecValTok{0}\OperatorTok{\},} \OperatorTok{\{}\DecValTok{0}\OperatorTok{,} \DecValTok{1}\OperatorTok{\},} \DecValTok{1}\OperatorTok{\}],}\NormalTok{ SFAD}\OperatorTok{[\{\{}\NormalTok{q1 }\SpecialCharTok{+}\NormalTok{ q2}\OperatorTok{,} \DecValTok{0}\OperatorTok{\},} \OperatorTok{\{}\DecValTok{0}\OperatorTok{,} \DecValTok{1}\OperatorTok{\},} \DecValTok{1}\OperatorTok{\}],}\NormalTok{ SFAD}\OperatorTok{[\{\{}\FunctionTok{p} \SpecialCharTok{+}\NormalTok{ q1}\OperatorTok{,} \DecValTok{0}\OperatorTok{\},} \OperatorTok{\{}\DecValTok{0}\OperatorTok{,} \DecValTok{1}\OperatorTok{\},} \DecValTok{1}\OperatorTok{\}],}\NormalTok{ SFAD}\OperatorTok{[\{\{}\FunctionTok{p} \SpecialCharTok{{-}}\NormalTok{ q2}\OperatorTok{,} \DecValTok{0}\OperatorTok{\},} \OperatorTok{\{}\DecValTok{0}\OperatorTok{,} \DecValTok{1}\OperatorTok{\},} \DecValTok{1}\OperatorTok{\}]\},} \OperatorTok{\{}\NormalTok{q1}\OperatorTok{,}\NormalTok{ q2}\OperatorTok{\},} \OperatorTok{\{}\FunctionTok{p}\OperatorTok{\},} \OperatorTok{\{}\FunctionTok{Hold}\OperatorTok{[}\NormalTok{Pair}\OperatorTok{][}\NormalTok{Momentum}\OperatorTok{[}\FunctionTok{p}\OperatorTok{,} \FunctionTok{D}\OperatorTok{],}\NormalTok{ Momentum}\OperatorTok{[}\FunctionTok{p}\OperatorTok{,} \FunctionTok{D}\OperatorTok{]]} \OtherTok{{-}\textgreater{}}\NormalTok{ pp}\OperatorTok{\},} \OperatorTok{\{\}]\}}\NormalTok{;}
\end{Highlighting}
\end{Shaded}

\begin{Shaded}
\begin{Highlighting}[]
\NormalTok{ints }\ExtensionTok{=} \OperatorTok{\{}\NormalTok{GLI}\OperatorTok{[}\StringTok{"fctopology1"}\OperatorTok{,} \OperatorTok{\{}\SpecialCharTok{{-}}\DecValTok{2}\OperatorTok{,} \DecValTok{1}\OperatorTok{,} \DecValTok{1}\OperatorTok{,} \DecValTok{2}\OperatorTok{,} \DecValTok{1}\OperatorTok{\}],}\NormalTok{ GLI}\OperatorTok{[}\StringTok{"fctopology1"}\OperatorTok{,} \OperatorTok{\{}\SpecialCharTok{{-}}\DecValTok{1}\OperatorTok{,} \DecValTok{0}\OperatorTok{,} \DecValTok{1}\OperatorTok{,} \DecValTok{2}\OperatorTok{,} \DecValTok{1}\OperatorTok{\}],}\NormalTok{ GLI}\OperatorTok{[}\StringTok{"fctopology1"}\OperatorTok{,} \OperatorTok{\{}\SpecialCharTok{{-}}\DecValTok{1}\OperatorTok{,} \DecValTok{1}\OperatorTok{,} \DecValTok{0}\OperatorTok{,} \DecValTok{2}\OperatorTok{,} \DecValTok{1}\OperatorTok{\}],} 
\NormalTok{ GLI}\OperatorTok{[}\StringTok{"fctopology1"}\OperatorTok{,} \OperatorTok{\{}\SpecialCharTok{{-}}\DecValTok{1}\OperatorTok{,} \DecValTok{1}\OperatorTok{,} \DecValTok{1}\OperatorTok{,} \DecValTok{1}\OperatorTok{,} \DecValTok{1}\OperatorTok{\}],}\NormalTok{ GLI}\OperatorTok{[}\StringTok{"fctopology1"}\OperatorTok{,} \OperatorTok{\{}\SpecialCharTok{{-}}\DecValTok{1}\OperatorTok{,} \DecValTok{1}\OperatorTok{,} \DecValTok{1}\OperatorTok{,} \DecValTok{2}\OperatorTok{,} \DecValTok{0}\OperatorTok{\}],}\NormalTok{ GLI}\OperatorTok{[}\StringTok{"fctopology1"}\OperatorTok{,} \OperatorTok{\{}\SpecialCharTok{{-}}\DecValTok{1}\OperatorTok{,} \DecValTok{1}\OperatorTok{,} \DecValTok{1}\OperatorTok{,} \DecValTok{2}\OperatorTok{,} \DecValTok{1}\OperatorTok{\}],} 
\NormalTok{ GLI}\OperatorTok{[}\StringTok{"fctopology1"}\OperatorTok{,} \OperatorTok{\{}\DecValTok{0}\OperatorTok{,} \SpecialCharTok{{-}}\DecValTok{1}\OperatorTok{,} \DecValTok{1}\OperatorTok{,} \DecValTok{2}\OperatorTok{,} \DecValTok{1}\OperatorTok{\}],}\NormalTok{ GLI}\OperatorTok{[}\StringTok{"fctopology1"}\OperatorTok{,} \OperatorTok{\{}\DecValTok{0}\OperatorTok{,} \DecValTok{0}\OperatorTok{,} \DecValTok{0}\OperatorTok{,} \DecValTok{2}\OperatorTok{,} \DecValTok{1}\OperatorTok{\}],}\NormalTok{ GLI}\OperatorTok{[}\StringTok{"fctopology1"}\OperatorTok{,} \OperatorTok{\{}\DecValTok{0}\OperatorTok{,} \DecValTok{0}\OperatorTok{,} \DecValTok{1}\OperatorTok{,} \DecValTok{1}\OperatorTok{,} \DecValTok{1}\OperatorTok{\}],} 
\NormalTok{ GLI}\OperatorTok{[}\StringTok{"fctopology1"}\OperatorTok{,} \OperatorTok{\{}\DecValTok{0}\OperatorTok{,} \DecValTok{0}\OperatorTok{,} \DecValTok{1}\OperatorTok{,} \DecValTok{2}\OperatorTok{,} \DecValTok{0}\OperatorTok{\}],}\NormalTok{ GLI}\OperatorTok{[}\StringTok{"fctopology1"}\OperatorTok{,} \OperatorTok{\{}\DecValTok{0}\OperatorTok{,} \DecValTok{0}\OperatorTok{,} \DecValTok{1}\OperatorTok{,} \DecValTok{2}\OperatorTok{,} \DecValTok{1}\OperatorTok{\}],}\NormalTok{ GLI}\OperatorTok{[}\StringTok{"fctopology1"}\OperatorTok{,} \OperatorTok{\{}\DecValTok{0}\OperatorTok{,} \DecValTok{1}\OperatorTok{,} \SpecialCharTok{{-}}\DecValTok{1}\OperatorTok{,} \DecValTok{2}\OperatorTok{,} \DecValTok{1}\OperatorTok{\}],} 
\NormalTok{ GLI}\OperatorTok{[}\StringTok{"fctopology1"}\OperatorTok{,} \OperatorTok{\{}\DecValTok{0}\OperatorTok{,} \DecValTok{1}\OperatorTok{,} \DecValTok{0}\OperatorTok{,} \DecValTok{1}\OperatorTok{,} \DecValTok{1}\OperatorTok{\}],}\NormalTok{ GLI}\OperatorTok{[}\StringTok{"fctopology1"}\OperatorTok{,} \OperatorTok{\{}\DecValTok{0}\OperatorTok{,} \DecValTok{1}\OperatorTok{,} \DecValTok{0}\OperatorTok{,} \DecValTok{2}\OperatorTok{,} \DecValTok{0}\OperatorTok{\}],}\NormalTok{ GLI}\OperatorTok{[}\StringTok{"fctopology1"}\OperatorTok{,} \OperatorTok{\{}\DecValTok{0}\OperatorTok{,} \DecValTok{1}\OperatorTok{,} \DecValTok{0}\OperatorTok{,} \DecValTok{2}\OperatorTok{,} \DecValTok{1}\OperatorTok{\}],} 
\NormalTok{ GLI}\OperatorTok{[}\StringTok{"fctopology1"}\OperatorTok{,} \OperatorTok{\{}\DecValTok{0}\OperatorTok{,} \DecValTok{1}\OperatorTok{,} \DecValTok{1}\OperatorTok{,} \DecValTok{0}\OperatorTok{,} \DecValTok{1}\OperatorTok{\}],}\NormalTok{ GLI}\OperatorTok{[}\StringTok{"fctopology1"}\OperatorTok{,} \OperatorTok{\{}\DecValTok{0}\OperatorTok{,} \DecValTok{1}\OperatorTok{,} \DecValTok{1}\OperatorTok{,} \DecValTok{1}\OperatorTok{,} \DecValTok{0}\OperatorTok{\}],}\NormalTok{ GLI}\OperatorTok{[}\StringTok{"fctopology1"}\OperatorTok{,} \OperatorTok{\{}\DecValTok{0}\OperatorTok{,} \DecValTok{1}\OperatorTok{,} \DecValTok{1}\OperatorTok{,} \DecValTok{1}\OperatorTok{,} \DecValTok{1}\OperatorTok{\}],} 
\NormalTok{ GLI}\OperatorTok{[}\StringTok{"fctopology1"}\OperatorTok{,} \OperatorTok{\{}\DecValTok{0}\OperatorTok{,} \DecValTok{1}\OperatorTok{,} \DecValTok{1}\OperatorTok{,} \DecValTok{2}\OperatorTok{,} \SpecialCharTok{{-}}\DecValTok{1}\OperatorTok{\}],}\NormalTok{ GLI}\OperatorTok{[}\StringTok{"fctopology1"}\OperatorTok{,} \OperatorTok{\{}\DecValTok{0}\OperatorTok{,} \DecValTok{1}\OperatorTok{,} \DecValTok{1}\OperatorTok{,} \DecValTok{2}\OperatorTok{,} \DecValTok{0}\OperatorTok{\}],}\NormalTok{ GLI}\OperatorTok{[}\StringTok{"fctopology1"}\OperatorTok{,} \OperatorTok{\{}\DecValTok{0}\OperatorTok{,} \DecValTok{1}\OperatorTok{,} \DecValTok{1}\OperatorTok{,} \DecValTok{2}\OperatorTok{,} \DecValTok{1}\OperatorTok{\}],} 
\NormalTok{ GLI}\OperatorTok{[}\StringTok{"fctopology1"}\OperatorTok{,} \OperatorTok{\{}\DecValTok{1}\OperatorTok{,} \SpecialCharTok{{-}}\DecValTok{1}\OperatorTok{,} \DecValTok{1}\OperatorTok{,} \DecValTok{1}\OperatorTok{,} \DecValTok{1}\OperatorTok{\}],}\NormalTok{ GLI}\OperatorTok{[}\StringTok{"fctopology1"}\OperatorTok{,} \OperatorTok{\{}\DecValTok{1}\OperatorTok{,} \DecValTok{0}\OperatorTok{,} \DecValTok{0}\OperatorTok{,} \DecValTok{1}\OperatorTok{,} \DecValTok{1}\OperatorTok{\}],}\NormalTok{ GLI}\OperatorTok{[}\StringTok{"fctopology1"}\OperatorTok{,} \OperatorTok{\{}\DecValTok{1}\OperatorTok{,} \DecValTok{0}\OperatorTok{,} \DecValTok{1}\OperatorTok{,} \DecValTok{0}\OperatorTok{,} \DecValTok{1}\OperatorTok{\}],} 
\NormalTok{ GLI}\OperatorTok{[}\StringTok{"fctopology1"}\OperatorTok{,} \OperatorTok{\{}\DecValTok{1}\OperatorTok{,} \DecValTok{0}\OperatorTok{,} \DecValTok{1}\OperatorTok{,} \DecValTok{1}\OperatorTok{,} \DecValTok{0}\OperatorTok{\}],}\NormalTok{ GLI}\OperatorTok{[}\StringTok{"fctopology1"}\OperatorTok{,} \OperatorTok{\{}\DecValTok{1}\OperatorTok{,} \DecValTok{0}\OperatorTok{,} \DecValTok{1}\OperatorTok{,} \DecValTok{1}\OperatorTok{,} \DecValTok{1}\OperatorTok{\}],}\NormalTok{ GLI}\OperatorTok{[}\StringTok{"fctopology1"}\OperatorTok{,} \OperatorTok{\{}\DecValTok{1}\OperatorTok{,} \DecValTok{1}\OperatorTok{,} \SpecialCharTok{{-}}\DecValTok{1}\OperatorTok{,} \DecValTok{1}\OperatorTok{,} \DecValTok{1}\OperatorTok{\}],} 
\NormalTok{ GLI}\OperatorTok{[}\StringTok{"fctopology1"}\OperatorTok{,} \OperatorTok{\{}\DecValTok{1}\OperatorTok{,} \DecValTok{1}\OperatorTok{,} \DecValTok{0}\OperatorTok{,} \DecValTok{0}\OperatorTok{,} \DecValTok{1}\OperatorTok{\}],}\NormalTok{ GLI}\OperatorTok{[}\StringTok{"fctopology1"}\OperatorTok{,} \OperatorTok{\{}\DecValTok{1}\OperatorTok{,} \DecValTok{1}\OperatorTok{,} \DecValTok{0}\OperatorTok{,} \DecValTok{1}\OperatorTok{,} \DecValTok{0}\OperatorTok{\}],}\NormalTok{ GLI}\OperatorTok{[}\StringTok{"fctopology1"}\OperatorTok{,} \OperatorTok{\{}\DecValTok{1}\OperatorTok{,} \DecValTok{1}\OperatorTok{,} \DecValTok{0}\OperatorTok{,} \DecValTok{1}\OperatorTok{,} \DecValTok{1}\OperatorTok{\}],} 
\NormalTok{ GLI}\OperatorTok{[}\StringTok{"fctopology1"}\OperatorTok{,} \OperatorTok{\{}\DecValTok{1}\OperatorTok{,} \DecValTok{1}\OperatorTok{,} \DecValTok{1}\OperatorTok{,} \SpecialCharTok{{-}}\DecValTok{1}\OperatorTok{,} \DecValTok{1}\OperatorTok{\}],}\NormalTok{ GLI}\OperatorTok{[}\StringTok{"fctopology1"}\OperatorTok{,} \OperatorTok{\{}\DecValTok{1}\OperatorTok{,} \DecValTok{1}\OperatorTok{,} \DecValTok{1}\OperatorTok{,} \DecValTok{0}\OperatorTok{,} \DecValTok{0}\OperatorTok{\}],}\NormalTok{ GLI}\OperatorTok{[}\StringTok{"fctopology1"}\OperatorTok{,} \OperatorTok{\{}\DecValTok{1}\OperatorTok{,} \DecValTok{1}\OperatorTok{,} \DecValTok{1}\OperatorTok{,} \DecValTok{0}\OperatorTok{,} \DecValTok{1}\OperatorTok{\}],} 
\NormalTok{ GLI}\OperatorTok{[}\StringTok{"fctopology1"}\OperatorTok{,} \OperatorTok{\{}\DecValTok{1}\OperatorTok{,} \DecValTok{1}\OperatorTok{,} \DecValTok{1}\OperatorTok{,} \DecValTok{1}\OperatorTok{,} \SpecialCharTok{{-}}\DecValTok{1}\OperatorTok{\}],}\NormalTok{ GLI}\OperatorTok{[}\StringTok{"fctopology1"}\OperatorTok{,} \OperatorTok{\{}\DecValTok{1}\OperatorTok{,} \DecValTok{1}\OperatorTok{,} \DecValTok{1}\OperatorTok{,} \DecValTok{1}\OperatorTok{,} \DecValTok{0}\OperatorTok{\}],}\NormalTok{ GLI}\OperatorTok{[}\StringTok{"fctopology1"}\OperatorTok{,} \OperatorTok{\{}\DecValTok{1}\OperatorTok{,} \DecValTok{1}\OperatorTok{,} \DecValTok{1}\OperatorTok{,} \DecValTok{1}\OperatorTok{,} \DecValTok{1}\OperatorTok{\}]\}}
\end{Highlighting}
\end{Shaded}

The first step is to generate scripts for creating .sbases and .lbases
files required by FIRE for the actual reduction. Those files are
specific to the given topology, which is why we need to provide the list
of the relevant topologies and as well as a path to the directory that
will contain reductions files. Each topology will receive a separate
subdirectory created from its name.

\begin{Shaded}
\begin{Highlighting}[]
\NormalTok{FIREPrepareStartFile}\OperatorTok{[}\NormalTok{topos}\OperatorTok{,}\FunctionTok{NotebookDirectory}\OperatorTok{[]]}
\end{Highlighting}
\end{Shaded}

If everything works as expected each topology directory will contain two
Mathematica scripts, \texttt{CreateLiteRedFiles.m} and
\texttt{CreateStartFile.m}. Both of them must be evaluated in the
correct order. The first script analyzes the topology using LiteRed,
while the second one uses this information to create .sbases and .lbases
files. In the case of complicated topologies, running those scripts can
take many minutes or even couple of hours. For simple topologies the
process is usually so fast, that one can conveniently run it during the
notebook evaluation. To this aim one may use

\begin{Shaded}
\begin{Highlighting}[]
\NormalTok{FIRECreateStartFile}\OperatorTok{[}\FunctionTok{NotebookDirectory}\OperatorTok{[],}\NormalTok{topos}\OperatorTok{]}
\end{Highlighting}
\end{Shaded}

The next step is to create a file containing the list of integrals and a
configuration file for the reduction.

\begin{Shaded}
\begin{Highlighting}[]
\NormalTok{FIRECreateConfigFile}\OperatorTok{[}\NormalTok{topos}\OperatorTok{,}\FunctionTok{NotebookDirectory}\OperatorTok{[]]}
\NormalTok{FIRECreateIntegralFile}\OperatorTok{[}\NormalTok{ints}\OperatorTok{,}\NormalTok{topos}\OperatorTok{,}\FunctionTok{NotebookDirectory}\OperatorTok{[]]}
\end{Highlighting}
\end{Shaded}

In particular, \texttt{FIRECreateConfigFile} offers a lot of options for
fine-tuning the reduction parameters as one normally do it when writing
such files by hand.

Again, in the case of complicated topologies the reduction must be done
on a powerful workstation or computer cluster and can take hours, days
or even weeks. But as long as everything is simple enough, there should
be no harm running it directly from the notebook via

\begin{Shaded}
\begin{Highlighting}[]
\NormalTok{FIRERunReduction}\OperatorTok{[}\FunctionTok{NotebookDirectory}\OperatorTok{[],}\NormalTok{topos}\OperatorTok{]}
\end{Highlighting}
\end{Shaded}

Notice, that in this case FeynHelpers runs the C++ version of FIRE via a
shell script. To that aim FIRE must be properly compiled from source.

Finally, in order to load the reduction tables into Mathematica it is
convenient to use

\begin{Shaded}
\begin{Highlighting}[]
\NormalTok{tables}\ExtensionTok{=}\NormalTok{FIREImportResults}\OperatorTok{[}\NormalTok{topos}\OperatorTok{,}\FunctionTok{NotebookDirectory}\OperatorTok{[]]}
\end{Highlighting}
\end{Shaded}

which returns a list of replacement rules that can be directly applied
to the given amplitude or perhaps exported to FORM.
\end{document}
