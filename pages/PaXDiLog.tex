% !TeX program = pdflatex
% !TeX root = PaXDiLog.tex

\documentclass[../FeynHelpersManual.tex]{subfiles}
\begin{document}
\hypertarget{paxdilog}{
\section{PaXDiLog}\label{paxdilog}\index{PaXDiLog}}

\texttt{PaXDiLog} corresponds to \texttt{DiLog} in Package-X.

\subsection{See also}

\hyperlink{toc}{Overview}.

\subsection{Examples}

\begin{Shaded}
\begin{Highlighting}[]
\CommentTok{(*Just to load Package{-}X*)}
\NormalTok{  PaXEvaluate}\OperatorTok{[}\NormalTok{A0}\OperatorTok{[}\DecValTok{1}\OperatorTok{]]}\NormalTok{;}
\end{Highlighting}
\end{Shaded}

\texttt{PaXDiLog} uses \texttt{X\textasciigrave DiLog} for numerical
evaluations

\begin{Shaded}
\begin{Highlighting}[]
\NormalTok{PaXDiLog}\OperatorTok{[}\DecValTok{1}\OperatorTok{,} \DecValTok{2}\OperatorTok{]}
\end{Highlighting}
\end{Shaded}

\begin{dmath*}\breakingcomma
\frac{\pi ^2}{6}
\end{dmath*}

\begin{Shaded}
\begin{Highlighting}[]
\FunctionTok{X}\NormalTok{\textasciigrave{}DiLog}\OperatorTok{[}\DecValTok{1}\OperatorTok{,} \DecValTok{2}\OperatorTok{]}
\end{Highlighting}
\end{Shaded}

\begin{dmath*}\breakingcomma
\frac{\pi ^2}{6}
\end{dmath*}

The same goes for derivatives and series expansions

\begin{Shaded}
\begin{Highlighting}[]
\FunctionTok{D}\OperatorTok{[}\NormalTok{PaXDiLog}\OperatorTok{[}\FunctionTok{x}\OperatorTok{,} \SpecialCharTok{\textbackslash{}}\OperatorTok{[}\NormalTok{Alpha}\OperatorTok{]],} \FunctionTok{x}\OperatorTok{]}
\end{Highlighting}
\end{Shaded}

\begin{dmath*}\breakingcomma
-\frac{\log (1-x+i \alpha \epsilon )}{x}
\end{dmath*}

\begin{Shaded}
\begin{Highlighting}[]
\FunctionTok{Series}\OperatorTok{[}\NormalTok{PaXDiLog}\OperatorTok{[}\FunctionTok{x}\OperatorTok{,} \SpecialCharTok{\textbackslash{}}\OperatorTok{[}\NormalTok{Alpha}\OperatorTok{]],} \OperatorTok{\{}\FunctionTok{x}\OperatorTok{,} \DecValTok{0}\OperatorTok{,} \DecValTok{5}\OperatorTok{\}]}
\end{Highlighting}
\end{Shaded}

\begin{dmath*}\breakingcomma
x+\frac{x^2}{4}+\frac{x^3}{9}+\frac{x^4}{16}+\frac{x^5}{25}+O\left(x^6\right)
\end{dmath*}
\end{document}
