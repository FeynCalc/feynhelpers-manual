% !TeX program = pdflatex
% Options for packages loaded elsewhere

\documentclass[11pt,a4paper,
parskip=half, % half a line vertical space between paragraphs
%headsepline,               % Kopfzeile mit horizontaler Liniel
%headings=big
]{scrreprt}
\usepackage[USenglish]{babel}
\usepackage[T1]{fontenc}
\usepackage[utf8]{inputenc}

\usepackage[a4paper, includefoot, left=1.5cm, right=1.5cm, top=2cm, bottom=2 cm]{geometry}

\author{}
\date{}

%\PassOptionsToPackage{unicode}{hyperref}
\PassOptionsToPackage{hyphens}{url}
%\usepackage{unicode-math}

\usepackage{amsmath,amssymb}

%https://tex.stackexchange.com/questions/60746/breqn-does-not-automatically-break-lines
\newcommand{\breakingcomma}{%
	\begingroup\lccode`~=`,
	\lowercase{\endgroup\expandafter\def\expandafter~\expandafter{~\penalty0 }}}


\usepackage{breqn}
\usepackage{fleqn}
%\usepackage{lmodern}
%\usepackage{newtxtext}
%\usepackage{PTSerif}
\usepackage{textcomp} % provide euro and other symbols

\usepackage{newpxtext}		% For now use Palatino as the default font
\usepackage{sourcecodepro}	% Default font for Mathematica 11 Input on Linux
%\usepackage{newtxmath}		% Default font for Mathematica 11 Output on Linux

\usepackage{upquote} % Use upquote if available, for straight quotes in verbatim environments

\usepackage[]{microtype}
\UseMicrotypeSet[protrusion]{basicmath} % disable protrusion for tt fonts

\usepackage{fancyvrb}
\usepackage{fvextra}
\usepackage{color}
\usepackage{framed}
\usepackage{xcolor}
\usepackage{subfiles}
\usepackage{xurl} % add URL line breaks if available
\usepackage{placeins}

%\usepackage[htt]{hyphenat}

\usepackage{orcidlink}

\usepackage{hyperref}

\hypersetup{
	pdftitle = {FeynHelpers Manual},
	pdfauthor = {Vladyslav Shtabovenko},
	colorlinks,
	filecolor=teal,
	linkcolor=teal,
	urlcolor=teal
}
\usepackage{bookmark}
\usepackage{makeidx}
\urlstyle{same} % disable monospaced font for URLs
%\usepackage{color}

\newcommand{\VerbBar}{|}
\newcommand{\VERB}{\Verb[commandchars=\\\{\}]}

\DefineVerbatimEnvironment
{Highlighting}{Verbatim}
{
%	frame=single,
	breaklines=true, 
	breakanywhere=true, breaksymbol=, 
	breakanywheresymbolpre=,
	framerule=0.1mm,
	fontfamily=tt,
%	framesep=3mm,
%	rulecolor=\color{red},
%	fillcolor=\color{yellow},
	commandchars=\\\{\}}

\DeclareUnicodeCharacter{03BB}{$\lambda$}
\DeclareUnicodeCharacter{03D5}{$\phi$}


%\newenvironment{Shaded}{}{}
%\definecolor{shadecolor}{RGB}{248,248,248}
\newenvironment{Shaded}{\begin{leftbar}}{\end{leftbar}}
\newcommand{\AlertTok}[1]{\textcolor[rgb]{1.00,0.00,0.00}{\textbf{#1}}}
\newcommand{\AnnotationTok}[1]{\textcolor[rgb]{0.38,0.63,0.69}{\textbf{\textit{#1}}}}
\newcommand{\AttributeTok}[1]{\textcolor[rgb]{0.49,0.56,0.16}{#1}}
\newcommand{\BaseNTok}[1]{\textcolor[rgb]{0.25,0.63,0.44}{#1}}
\newcommand{\BuiltInTok}[1]{#1}
\newcommand{\CharTok}[1]{\textcolor[rgb]{0.25,0.44,0.63}{#1}}
\newcommand{\CommentTok}[1]{\textcolor[rgb]{0.38,0.63,0.69}{\textit{#1}}}
\newcommand{\CommentVarTok}[1]{\textcolor[rgb]{0.38,0.63,0.69}{\textbf{\textit{#1}}}}
\newcommand{\ConstantTok}[1]{\textcolor[rgb]{0.53,0.00,0.00}{#1}}
\newcommand{\ControlFlowTok}[1]{\textcolor[rgb]{0.00,0.44,0.13}{\textbf{#1}}}
\newcommand{\DataTypeTok}[1]{\textcolor[rgb]{0.56,0.13,0.00}{#1}}
\newcommand{\DecValTok}[1]{\textcolor[rgb]{0.25,0.63,0.44}{#1}}
\newcommand{\DocumentationTok}[1]{\textcolor[rgb]{0.73,0.13,0.13}{\textit{#1}}}
\newcommand{\ErrorTok}[1]{\textcolor[rgb]{1.00,0.00,0.00}{\textbf{#1}}}
\newcommand{\ExtensionTok}[1]{#1}
\newcommand{\FloatTok}[1]{\textcolor[rgb]{0.25,0.63,0.44}{#1}}
\newcommand{\FunctionTok}[1]{\textcolor[rgb]{0.02,0.16,0.49}{#1}}
\newcommand{\ImportTok}[1]{#1}
\newcommand{\InformationTok}[1]{\textcolor[rgb]{0.38,0.63,0.69}{\textbf{\textit{#1}}}}
\newcommand{\KeywordTok}[1]{\textcolor[rgb]{0.00,0.44,0.13}{\textbf{#1}}}
\newcommand{\NormalTok}[1]{#1}
\newcommand{\OperatorTok}[1]{\textcolor[rgb]{0.40,0.40,0.40}{#1}}
\newcommand{\OtherTok}[1]{\textcolor[rgb]{0.00,0.44,0.13}{#1}}
\newcommand{\PreprocessorTok}[1]{\textcolor[rgb]{0.74,0.48,0.00}{#1}}
\newcommand{\RegionMarkerTok}[1]{#1}
\newcommand{\SpecialCharTok}[1]{\textcolor[rgb]{0.25,0.44,0.63}{#1}}
\newcommand{\SpecialStringTok}[1]{\textcolor[rgb]{0.73,0.40,0.53}{#1}}
\newcommand{\StringTok}[1]{\textcolor[rgb]{0.25,0.44,0.63}{#1}}
\newcommand{\VariableTok}[1]{\textcolor[rgb]{0.10,0.09,0.49}{#1}}
\newcommand{\VerbatimStringTok}[1]{\textcolor[rgb]{0.25,0.44,0.63}{#1}}
\newcommand{\WarningTok}[1]{\textcolor[rgb]{0.38,0.63,0.69}{\textbf{\textit{#1}}}}

% This kind of reproduces the stlyle of Mathematica output
\SetSymbolFont{operators}   {normal}{OT1}{cmr} {m}{n}
%\SetSymbolFont{letters}     {normal}{OML}{cmm} {m}{it}
%\SetSymbolFont{symbols}     {normal}{OMS}{cmsy}{m}{n}
\SetSymbolFont{letters}{normal}{OML}{ntxmi}{m}{it}
\SetSymbolFont{symbols}{normal}{OMS}{ntxsy}{m}{n}
\SetSymbolFont{largesymbols}{normal}{OMX}{cmex}{m}{n}
\SetMathAlphabet{\mathbf}{normal}{OT1}{cmr}{bx}{n}
\SetMathAlphabet{\mathsf}{normal}{OT1}{cmss}{m}{n}
\SetMathAlphabet{\mathit}{normal}{OT1}{cmr}{m}{it}
\SetMathAlphabet{\mathtt}{normal}{OT1}{cmtt}{m}{n}


\renewcommand{\texttt}[1]{{\bfseries\ttfamily #1}}

\setlength{\emergencystretch}{3em} % prevent overfull lines
\providecommand{\tightlist}{%
	\setlength{\itemsep}{0pt}\setlength{\parskip}{0pt}}
%\setcounter{secnumdepth}{-\maxdimen} % remove section numbering


% allow page breaks 
\allowdisplaybreaks

% active index generation
\makeindex

%%%%%%%%%%%%%%%%%%% TOC CUSTOMIZATION %%%%%%%%%%%%%%%%%%%%%%%%%%%%%%%
\addtocontents{toc}{\protect\hypertarget{toc}{}} % Add a link target to the TOC itself
\renewcommand\TOCLineLeaderFill[1][]{\hfill} % get rid of the dots between the entry name and the page number
\addtokomafont{partentry}{\Huge} % Part Überschriften
\addtokomafont{disposition}{\sffamily\Large} % Chapter Überschriften
% Let us have more space between subsection numbers and the titles, for things like 3.1.17. XXX ; 
\DeclareTOCStyleEntry[numwidth=2.5em]{tocline}{part}
\DeclareTOCStyleEntry[numwidth=2.5em]{tocline}{chapter}
\DeclareTOCStyleEntry[numwidth=3.5em]{tocline}{section}
\DeclareTOCStyleEntry[numwidth=4.5em]{tocline}{subsection}
\DeclareTOCStyleEntry[numwidth=6.5em]{tocline}{subsubsection}

\setcounter{tocdepth}{1}

% enable numbering of subsubsections
%\setcounter{secnumdepth}{1}
%%%%%%%%%%%%%%%%%%%%%%%%%%%%%%%%%%%%%%%%%%%%%%%%%%%%%%%%%%%%%%%%%%%%%


\begin{document}

\title{\Huge Guide to FeynHelpers}
\subtitle{\LARGE A collection of interfaces between \textsc{FeynCalc} and other HEP-related tools}


\author{Vladyslav Shtabovenko\,\orcidlink{0000-0002-2782-3694}}
\date{\today}
\maketitle

This documentation snapshot was generated on \texttt{DATE_TIME} from the commit \texttt{GITHUB_SHA_SHORT}

% Here comes the TOC!
{\hypersetup{hidelinks}
\tableofcontents
}	

\chapter{Useful information}

\subfile{pages/Cite.tex}
\subfile{pages/Install.tex}
\subfile{pages/TensorReductionWithFermat.tex}

\chapter{Tutorials}

\subfile{pages/FiestaUsageExamples.tex}
\subfile{pages/PSDUsageExamples.tex}
\subfile{pages/QGRAFUsageExamples.tex}

\chapter{Generic functions and symbols}

\subfile{pages/FeynHelpersHowToCite.tex}
\subfile{pages/DollarFeynHelpersDirectory.tex}
\subfile{pages/DollarFeynHelpersLastCommitDateHash.tex}
\subfile{pages/DollarFeynHelpersLoadInterfaces.tex}
\subfile{pages/DollarFeynHelpersVersion.tex}

\chapter{Fermat interface}

\subfile{pages/FerImportArrayAsSparseMatrix.tex}
\subfile{pages/FerMatrixToFermatArray.tex}
\subfile{pages/FerCommand.tex}
\subfile{pages/FerRunScript.tex}
\subfile{pages/FerRowReduce.tex}
\subfile{pages/FerSolve.tex}
\subfile{pages/FerInputFile.tex}
\subfile{pages/FerOutputFile.tex}
\subfile{pages/FerPath.tex}
\subfile{pages/FerScriptFile.tex}

\chapter{Package-X interface}

\subfile{pages/PaXEvaluate.tex}
\subfile{pages/PaXEvaluateUV.tex}
\subfile{pages/PaXEvaluateIR.tex}
\subfile{pages/PaXEvaluateUVIRSplit.tex}
\subfile{pages/PaXContinuedDiLog.tex}
\subfile{pages/PaXDiLog.tex}
\subfile{pages/PaXDiscB.tex}
\subfile{pages/PaXEpsilonBar.tex}
\subfile{pages/PaXKallenLambda.tex}
\subfile{pages/PaXKibblePhi.tex}
\subfile{pages/PaXLn.tex}
\subfile{pages/PaXpvA.tex}
\subfile{pages/PaXpvB.tex}
\subfile{pages/PaXpvC.tex}
\subfile{pages/PaXpvD.tex}
\subfile{pages/PaXAnalytic.tex}
\subfile{pages/PaXC0Expand.tex}
\subfile{pages/PaXD0Expand.tex}
\subfile{pages/PaXDiscExpand.tex}
\subfile{pages/PaXExpandInEpsilon.tex}
\subfile{pages/PaXImplicitPrefactor.tex}
\subfile{pages/PaXKallenExpand.tex}
\subfile{pages/PaXKibbleExpand.tex}
\subfile{pages/PaXLoopRefineOptions.tex}
\subfile{pages/PaXSeries.tex}
\subfile{pages/PaXSimplifyEpsilon.tex}
\subfile{pages/PaXSubstituteEpsilon.tex}

\chapter{FIESTA interface}

\subfile{pages/FSACreateMathematicaScripts.tex}
\subfile{pages/FSARunIntegration.tex}
\subfile{pages/FSALoadNumericalResults.tex}
\subfile{pages/FSAAdditionalPrefactor.tex}
\subfile{pages/FSAAnalyticIntegration.tex}
\subfile{pages/FSAAnalyzeWorstPower.tex}
\subfile{pages/FSAAssemblyIntegration.tex}
\subfile{pages/FSAAsyLP.tex}
\subfile{pages/FSABalanceMode.tex}
\subfile{pages/FSABalancePower.tex}
\subfile{pages/FSABalanceSamplingPoints.tex}
\subfile{pages/FSABucketSize.tex}
\subfile{pages/FSAChunkSize.tex}
\subfile{pages/FSACIntegratePath.tex}
\subfile{pages/FSAComplexMode.tex}
\subfile{pages/FSAContourShiftCoefficient.tex}
\subfile{pages/FSAContourShiftIgnoreFail.tex}
\subfile{pages/FSAContourShiftShape.tex}
\subfile{pages/FSAd0.tex}
\subfile{pages/FSADataPath.tex}
\subfile{pages/FSADebugAllEntries.tex}
\subfile{pages/FSADebugMemory.tex}
\subfile{pages/FSADebugParallel.tex}
\subfile{pages/FSADebugSector.tex}
\subfile{pages/FSAEpVarNegativeTermsHandling.tex}
\subfile{pages/FSAExactIntegrationOrder.tex}
\subfile{pages/FSAExactIntegrationTimeout.tex}
\subfile{pages/FSAExpandResult.tex}
\subfile{pages/FSAExpandVar.tex}
\subfile{pages/FSAFixedContourShift.tex}
\subfile{pages/FSAFixSectors.tex}
\subfile{pages/FSAGPUIntegration.tex}
\subfile{pages/FSAGraph.tex}
\subfile{pages/FSAIntegrator.tex}
\subfile{pages/FSAIntegratorOptions.tex}
\subfile{pages/FSALambdaIterations.tex}
\subfile{pages/FSALambdaSplit.tex}
\subfile{pages/FSAMathematicaBinary.tex}
\subfile{pages/FSAMathematicaKernelPath.tex}
\subfile{pages/FSAMinimizeContourTransformation.tex}
\subfile{pages/FSAMixSectors.tex}
\subfile{pages/FSAMPMin.tex}
\subfile{pages/FSAMPPrecisionShift.tex}
\subfile{pages/FSAMPSmallX.tex}
\subfile{pages/FSAMPThreshold.tex}
\subfile{pages/FSANoAVX.tex}
\subfile{pages/FSANoDatabaseLock.tex}
\subfile{pages/FSANumberOfLinks.tex}
\subfile{pages/FSANumberOfSubkernels.tex}
\subfile{pages/FSAOnlyPrepare.tex}
\subfile{pages/FSAOnlyPrepareRegions.tex}
\subfile{pages/FSAOptimizeIntegrationStrings.tex}
\subfile{pages/FSAOrderInEps.tex}
\subfile{pages/FSAParameterRules.tex}
\subfile{pages/FSAPath.tex}
\subfile{pages/FSAPMVar.tex}
\subfile{pages/FSAPolesMultiplicity.tex}
\subfile{pages/FSAPrecision.tex}
\subfile{pages/FSAPrimarySectorCoefficients.tex}
\subfile{pages/FSAQHullPath.tex}
\subfile{pages/FSARegionNumber.tex}
\subfile{pages/FSARegVar.tex}
\subfile{pages/FSARemoveDatabases.tex}
\subfile{pages/FSAResolutionMode.tex}
\subfile{pages/FSAReturnErrorWithBrackets.tex}
\subfile{pages/FSAScriptFileName.tex}
\subfile{pages/FSASDExpandAsy.tex}
\subfile{pages/FSASDExpandAsyOrder.tex}
\subfile{pages/FSASectorSplitting.tex}
\subfile{pages/FSASectorSymmetries.tex}
\subfile{pages/FSASeparateTerms.tex}
\subfile{pages/FSAShowOutput.tex}
\subfile{pages/FSAStrategy.tex}
\subfile{pages/FSAUsingC.tex}
\subfile{pages/FSAXVar.tex}
\subfile{pages/FSAZeroCheckCount.tex}

\chapter{C++ FIRE interface}

\subfile{pages/FIRECreateConfigFile.tex}
\subfile{pages/FIRECreateIntegralFile.tex}
\subfile{pages/FIRECreateLiteRedFiles.tex}
\subfile{pages/FIRECreateStartFile.tex}
\subfile{pages/FIREImportResults.tex}
\subfile{pages/FIREPrepareStartFile.tex}
\subfile{pages/FIRERunReduction.tex}
\subfile{pages/FIREToFCTopology.tex}
\subfile{pages/FIREAutoDetectRestrictions.tex}
\subfile{pages/FIREBinaryPath.tex}
\subfile{pages/FIREBucket.tex}
\subfile{pages/FIRECompressor.tex}
\subfile{pages/FIREDatabase.tex}
\subfile{pages/FIREFthreads.tex}
\subfile{pages/FIREIntegrals.tex}
\subfile{pages/FIRELI.tex}
\subfile{pages/FIRELthreads.tex}
\subfile{pages/FIREMathematicaKernelPath.tex}
\subfile{pages/FIREParallel.tex}
\subfile{pages/FIREProblemId.tex}
\subfile{pages/FIREPosPref.tex}
\subfile{pages/FIREShowOutput.tex}
\subfile{pages/FIRESthreads.tex}
\subfile{pages/FIREThreads.tex}
\subfile{pages/FIREUseLiteRed.tex}

\chapter{Mathematica FIRE interface}

\subfile{pages/FIREBurn.tex}
\subfile{pages/FIREAddPropagators.tex}
\subfile{pages/FIREConfigFiles.tex}
\subfile{pages/FIREPath.tex}
\subfile{pages/FIRERun.tex}
\subfile{pages/FIRESilentMode.tex}
\subfile{pages/FIREStartFile.tex}
\subfile{pages/FIREUsingFermat.tex}

\chapter{Kira interface}

\subfile{pages/KiraCreateConfigFiles.tex}
\subfile{pages/KiraCreateIntegralFile.tex}
\subfile{pages/KiraCreateJobFile.tex}
\subfile{pages/KiraGetRS.tex}
\subfile{pages/KiraImportResults.tex}
\subfile{pages/KiraLabelSector.tex}
\subfile{pages/KiraRunReduction.tex}
\subfile{pages/KiraBinaryPath.tex}
\subfile{pages/KiraFermatPath.tex}
\subfile{pages/KiraIncomingMomenta.tex}
\subfile{pages/KiraIntegrals.tex}
\subfile{pages/KiraJobFileName.tex}
\subfile{pages/KiraMassDimensions.tex}
\subfile{pages/KiraMomentumConservation.tex}
\subfile{pages/KiraOutgoingMomenta.tex}
\subfile{pages/KiraShowOutput.tex}

\chapter{LoopTools interface}

\subfile{pages/LToolsEvaluate.tex}
\subfile{pages/LToolsLoadLibrary.tex}
\subfile{pages/LToolsUnLoadLibrary.tex}
\subfile{pages/LToolsExpandInEpsilon.tex}
\subfile{pages/LToolsFullResult.tex}
\subfile{pages/LToolsImplicitPrefactor.tex}
\subfile{pages/LToolsSetLambda.tex}
\subfile{pages/LToolsSetMudim.tex}
\subfile{pages/LToolsPath.tex}
\subfile{pages/LToolsA0.tex}
\subfile{pages/LToolsA00.tex}
\subfile{pages/LToolsA0i.tex}
\subfile{pages/LToolsB0.tex}
\subfile{pages/LToolsB00.tex}
\subfile{pages/LToolsB0i.tex}
\subfile{pages/LToolsB1.tex}
\subfile{pages/LToolsB001.tex}
\subfile{pages/LToolsB11.tex}
\subfile{pages/LToolsB111.tex}
\subfile{pages/LToolsC0.tex}
\subfile{pages/LToolsC0i.tex}
\subfile{pages/LToolsClearCache.tex}
\subfile{pages/LToolsD0.tex}
\subfile{pages/LToolsD0i.tex}
\subfile{pages/LToolsDB0.tex}
\subfile{pages/LToolsDB00.tex}
\subfile{pages/LToolsDB1.tex}
\subfile{pages/LToolsDB11.tex}
\subfile{pages/LToolsDebugA.tex}
\subfile{pages/LToolsDebugAll.tex}
\subfile{pages/LToolsDebugB.tex}
\subfile{pages/LToolsDebugC.tex}
\subfile{pages/LToolsDebugD.tex}
\subfile{pages/LToolsDebugE.tex}
\subfile{pages/LToolsDR1eps.tex}
\subfile{pages/LToolsDRResult.tex}
\subfile{pages/LToolsE0.tex}
\subfile{pages/LToolsE0i.tex}
\subfile{pages/LToolsGetCmpBits.tex}
\subfile{pages/LToolsGetDebugKey.tex}
\subfile{pages/LToolsGetDelta.tex}
\subfile{pages/LToolsGetDiffEps.tex}
\subfile{pages/LToolsGetErrDigits.tex}
\subfile{pages/LToolsGetLambda.tex}
\subfile{pages/LToolsGetMaxDev.tex}
\subfile{pages/LToolsGetMinMass.tex}
\subfile{pages/LToolsGetMudim.tex}
\subfile{pages/LToolsGetUVDiv.tex}
\subfile{pages/LToolsGetVersionKey.tex}
\subfile{pages/LToolsGetWarnDigits.tex}
\subfile{pages/LToolsGetZeroEps.tex}
\subfile{pages/LToolsKeyA0.tex}
\subfile{pages/LToolsKeyAll.tex}
\subfile{pages/LToolsKeyBget.tex}
\subfile{pages/LToolsKeyC0.tex}
\subfile{pages/LToolsKeyCEget.tex}
\subfile{pages/LToolsKeyD0.tex}
\subfile{pages/LToolsKeyE0.tex}
\subfile{pages/LToolsKeyEget.tex}
\subfile{pages/LToolsLi2.tex}
\subfile{pages/LToolsLi2omx.tex}
\subfile{pages/LToolsMarkCache.tex}
\subfile{pages/LToolsPaVe.tex}
\subfile{pages/LToolsRestoreCache.tex}
\subfile{pages/LToolsSetCmpBits.tex}
\subfile{pages/LToolsSetDebugKey.tex}
\subfile{pages/LToolsSetDebugRange.tex}
\subfile{pages/LToolsSetDelta.tex}
\subfile{pages/LToolsSetDiffEps.tex}
\subfile{pages/LToolsSetErrDigits.tex}
\subfile{pages/LToolsSetMaxDev.tex}
\subfile{pages/LToolsSetMinMass.tex}
\subfile{pages/LToolsSetUVDiv.tex}
\subfile{pages/LToolsSetVersionKey.tex}
\subfile{pages/LToolsSetWarnDigits.tex}
\subfile{pages/DollarLTools.tex}

\chapter{pySecDec interface}

\subfile{pages/PSDCreatePythonScripts.tex}
\subfile{pages/PSDIntegrate.tex}
\subfile{pages/PSDLoopIntegralFromPropagators.tex}
\subfile{pages/PSDLoadNumericalResults.tex}
\subfile{pages/PSDLoopPackage.tex}
\subfile{pages/PSDLoopRegions.tex}
\subfile{pages/PSDSumPackage.tex}
\subfile{pages/PSDAdditionalPrefactor.tex}
\subfile{pages/PSDAddMonomialRegulatorPower.tex}
\subfile{pages/PSDCoefficients.tex}
\subfile{pages/PSDComplexParameterRules.tex}
\subfile{pages/PSDComplexParameters.tex}
\subfile{pages/PSDComplexParameterValues.tex}
\subfile{pages/PSDContourDeformation.tex}
\subfile{pages/PSDCPUThreads.tex}
\subfile{pages/PSDDecompositionMethod.tex}
\subfile{pages/PSDDecreaseToPercentage.tex}
\subfile{pages/PSDDeformationParametersDecreaseFactor.tex}
\subfile{pages/PSDDeformationParametersMaximum.tex}
\subfile{pages/PSDDeformationParametersMinimum.tex}
\subfile{pages/PSDEnforceComplex.tex}
\subfile{pages/PSDEpsAbs.tex}
\subfile{pages/PSDEpsRel.tex}
\subfile{pages/PSDErrorMode.tex}
\subfile{pages/PSDErrorModeQmc.tex}
\subfile{pages/PSDEvaluateMinn.tex}
\subfile{pages/PSDExpansionByRegionsOrder.tex}
\subfile{pages/PSDExpansionByRegionsParameter.tex}
\subfile{pages/PSDFitFunction.tex}
\subfile{pages/PSDFlags.tex}
\subfile{pages/PSDFormExecutable.tex}
\subfile{pages/PSDFormMemoryUse.tex}
\subfile{pages/PSDFormOptimizationLevel.tex}
\subfile{pages/PSDFormThreads.tex}
\subfile{pages/PSDFormWorkSpace.tex}
\subfile{pages/PSDGenerateFileName.tex}
\subfile{pages/PSDGeneratingVectors.tex}
\subfile{pages/PSDIntegrateFileName.tex}
\subfile{pages/PSDIntegrator.tex}
\subfile{pages/PSDLoopIntegralName.tex}
\subfile{pages/PSDMaxEpsAbs.tex}
\subfile{pages/PSDMaxEpsRel.tex}
\subfile{pages/PSDMaxEval.tex}
\subfile{pages/PSDMaxIncreaseFac.tex}
\subfile{pages/PSDMinDecreaseFactor.tex}
\subfile{pages/PSDMinEpsAbs.tex}
\subfile{pages/PSDMinEpsRel.tex}
\subfile{pages/PSDMinEval.tex}
\subfile{pages/PSDMinm.tex}
\subfile{pages/PSDMinn.tex}
\subfile{pages/PSDNormalizExecutable.tex}
\subfile{pages/PSDNumberOfPresamples.tex}
\subfile{pages/PSDNumberOfThreads.tex}
\subfile{pages/PSDOutputDirectory.tex}
\subfile{pages/PSDOverwritePackageDirectory.tex}
\subfile{pages/PSDPyLinkQMCTransforms.tex}
\subfile{pages/PSDRealParameterRules.tex}
\subfile{pages/PSDRealParameters.tex}
\subfile{pages/PSDRealParameterValues.tex}
\subfile{pages/PSDRegulators.tex}
\subfile{pages/PSDRequestedOrder.tex}
\subfile{pages/PSDResetCudaAfter.tex}
\subfile{pages/PSDSplit.tex}
\subfile{pages/PSDTransform.tex}
\subfile{pages/PSDVerbose.tex}
\subfile{pages/PSDVerbosity.tex}

\chapter{QGRAF interface}

\subfile{pages/QGConvertToFC.tex}
\subfile{pages/QGCreateAmp.tex}
\subfile{pages/QGLoadInsertions.tex}
\subfile{pages/QGTZFCreateFieldStyles.tex}
\subfile{pages/QGTZFCreateTeXFiles.tex}
\subfile{pages/QGPolarization.tex}
\subfile{pages/QGPropagator.tex}
\subfile{pages/QGTruncatedPolarization.tex}
\subfile{pages/QGVertex.tex}
\subfile{pages/DollarQGScriptsDirectory.tex}
\subfile{pages/DollarQGInsertionsDirectory.tex}
\subfile{pages/DollarQGLogOutputAmplitudes.tex}
\subfile{pages/DollarQGLogOutputDiagrams.tex}
\subfile{pages/DollarQGModelsDirectory.tex}
\subfile{pages/DollarQGStylesDirectory.tex}
\subfile{pages/DollarQGTeXDirectory.tex}
\subfile{pages/QGAmplitudeStyle.tex}
\subfile{pages/QGBinaryFile.tex}
\subfile{pages/QGCleanUpOutputDirectory.tex}
\subfile{pages/QGDiagramStyle.tex}
\subfile{pages/QGInsertionRule.tex}
\subfile{pages/QGLoopMomentum.tex}
\subfile{pages/QGModel.tex}
\subfile{pages/QGOptionalStatements.tex}
\subfile{pages/QGOptions.tex}
\subfile{pages/QGOutputAmplitudes.tex}
\subfile{pages/QGOutputDiagrams.tex}
\subfile{pages/QGOutputDirectory.tex}
\subfile{pages/QGOverwriteExistingAmplitudes.tex}
\subfile{pages/QGOverwriteExistingDiagrams.tex}
\subfile{pages/QGSaveInputFile.tex}
\subfile{pages/QGShowOutput.tex}
\subfile{pages/QGFieldStyles.tex}

\printindex

\end{document}
